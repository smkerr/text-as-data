% Options for packages loaded elsewhere
\PassOptionsToPackage{unicode}{hyperref}
\PassOptionsToPackage{hyphens}{url}
\PassOptionsToPackage{dvipsnames,svgnames,x11names}{xcolor}
%
\documentclass[
]{article}
\usepackage{amsmath,amssymb}
\usepackage{iftex}
\ifPDFTeX
  \usepackage[T1]{fontenc}
  \usepackage[utf8]{inputenc}
  \usepackage{textcomp} % provide euro and other symbols
\else % if luatex or xetex
  \usepackage{unicode-math} % this also loads fontspec
  \defaultfontfeatures{Scale=MatchLowercase}
  \defaultfontfeatures[\rmfamily]{Ligatures=TeX,Scale=1}
\fi
\usepackage{lmodern}
\ifPDFTeX\else
  % xetex/luatex font selection
\fi
% Use upquote if available, for straight quotes in verbatim environments
\IfFileExists{upquote.sty}{\usepackage{upquote}}{}
\IfFileExists{microtype.sty}{% use microtype if available
  \usepackage[]{microtype}
  \UseMicrotypeSet[protrusion]{basicmath} % disable protrusion for tt fonts
}{}
\makeatletter
\@ifundefined{KOMAClassName}{% if non-KOMA class
  \IfFileExists{parskip.sty}{%
    \usepackage{parskip}
  }{% else
    \setlength{\parindent}{0pt}
    \setlength{\parskip}{6pt plus 2pt minus 1pt}}
}{% if KOMA class
  \KOMAoptions{parskip=half}}
\makeatother
\usepackage{xcolor}
\usepackage[margin=1in]{geometry}
\usepackage{graphicx}
\makeatletter
\def\maxwidth{\ifdim\Gin@nat@width>\linewidth\linewidth\else\Gin@nat@width\fi}
\def\maxheight{\ifdim\Gin@nat@height>\textheight\textheight\else\Gin@nat@height\fi}
\makeatother
% Scale images if necessary, so that they will not overflow the page
% margins by default, and it is still possible to overwrite the defaults
% using explicit options in \includegraphics[width, height, ...]{}
\setkeys{Gin}{width=\maxwidth,height=\maxheight,keepaspectratio}
% Set default figure placement to htbp
\makeatletter
\def\fps@figure{htbp}
\makeatother
\setlength{\emergencystretch}{3em} % prevent overfull lines
\providecommand{\tightlist}{%
  \setlength{\itemsep}{0pt}\setlength{\parskip}{0pt}}
\setcounter{secnumdepth}{-\maxdimen} % remove section numbering
\usepackage{booktabs}
\usepackage{xcolor}
\ifLuaTeX
  \usepackage{selnolig}  % disable illegal ligatures
\fi
\usepackage[]{natbib}
\bibliographystyle{plainnat}
\IfFileExists{bookmark.sty}{\usepackage{bookmark}}{\usepackage{hyperref}}
\IfFileExists{xurl.sty}{\usepackage{xurl}}{} % add URL line breaks if available
\urlstyle{same}
\hypersetup{
  pdftitle={Assignment 3},
  pdfauthor={Text as Data},
  colorlinks=true,
  linkcolor={Maroon},
  filecolor={Maroon},
  citecolor={Blue},
  urlcolor={blue},
  pdfcreator={LaTeX via pandoc}}

\title{Assignment 3}
\author{Text as Data}
\date{2023-10-02}

\begin{document}
\maketitle

\hypertarget{string-manipulation}{%
\subsection{String manipulation}\label{string-manipulation}}

The following homework sets you a challenging task to give structure to
raw text. You will need to look through the text to discover text
patterns that will allow you to split, join, and extract in order to
create the structure we are looking for. If you struggle to implement
this, describe \emph{in words} how you would plan to go about the task,
thinking about how to split this into subtasks and how to describe
these.

To start with, you are asked to retrieve \emph{Songs of Innocence and of
Experience} by William Blake from Project Gutenberg. It is located at
\url{https://www.gutenberg.org/cache/epub/1934/pg1934.txt}. This is a
collection of poems in two books: \emph{Songs of Innocence} and
\emph{Songs of Experience}.

The goal of the task is to parse this into a dataframe where each row is
a line of a poem (there should be no empty lines). The following columns
should describe where each line was found:

\begin{itemize}
\tightlist
\item
  line\_number
\item
  stanza\_number
\item
  poem\_title
\item
  book\_title
\end{itemize}

\hypertarget{substeps}{%
\subsection{Substeps}\label{substeps}}

Think about how to split this up into smaller tasks before bringing this
all together. Remember that when working on loops it is often easier to
work with a single item in the list first, before putting this into a
loop. There will be other ways to approach this, but a step-by-step
approach will do the following:

\begin{itemize}
\tightlist
\item
  Get the content of the book (removing publisher information, contents,
  and the copyright notice)
\item
  Split the book into the two sub-books (Songs of Innocence and Songs of
  Experience)
\item
  Split each book into poems
\item
  Split each poem into stanzas (verses)
\item
  Split each stanza into lines
\end{itemize}

  \bibliography{../presentation-resources/MyLibrary.bib}

\end{document}
