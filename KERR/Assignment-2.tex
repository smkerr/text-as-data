% Options for packages loaded elsewhere
\PassOptionsToPackage{unicode}{hyperref}
\PassOptionsToPackage{hyphens}{url}
%
\documentclass[
]{article}
\usepackage{amsmath,amssymb}
\usepackage{iftex}
\ifPDFTeX
  \usepackage[T1]{fontenc}
  \usepackage[utf8]{inputenc}
  \usepackage{textcomp} % provide euro and other symbols
\else % if luatex or xetex
  \usepackage{unicode-math} % this also loads fontspec
  \defaultfontfeatures{Scale=MatchLowercase}
  \defaultfontfeatures[\rmfamily]{Ligatures=TeX,Scale=1}
\fi
\usepackage{lmodern}
\ifPDFTeX\else
  % xetex/luatex font selection
\fi
% Use upquote if available, for straight quotes in verbatim environments
\IfFileExists{upquote.sty}{\usepackage{upquote}}{}
\IfFileExists{microtype.sty}{% use microtype if available
  \usepackage[]{microtype}
  \UseMicrotypeSet[protrusion]{basicmath} % disable protrusion for tt fonts
}{}
\makeatletter
\@ifundefined{KOMAClassName}{% if non-KOMA class
  \IfFileExists{parskip.sty}{%
    \usepackage{parskip}
  }{% else
    \setlength{\parindent}{0pt}
    \setlength{\parskip}{6pt plus 2pt minus 1pt}}
}{% if KOMA class
  \KOMAoptions{parskip=half}}
\makeatother
\usepackage{xcolor}
\usepackage[margin=1in]{geometry}
\usepackage{color}
\usepackage{fancyvrb}
\newcommand{\VerbBar}{|}
\newcommand{\VERB}{\Verb[commandchars=\\\{\}]}
\DefineVerbatimEnvironment{Highlighting}{Verbatim}{commandchars=\\\{\}}
% Add ',fontsize=\small' for more characters per line
\usepackage{framed}
\definecolor{shadecolor}{RGB}{248,248,248}
\newenvironment{Shaded}{\begin{snugshade}}{\end{snugshade}}
\newcommand{\AlertTok}[1]{\textcolor[rgb]{0.94,0.16,0.16}{#1}}
\newcommand{\AnnotationTok}[1]{\textcolor[rgb]{0.56,0.35,0.01}{\textbf{\textit{#1}}}}
\newcommand{\AttributeTok}[1]{\textcolor[rgb]{0.13,0.29,0.53}{#1}}
\newcommand{\BaseNTok}[1]{\textcolor[rgb]{0.00,0.00,0.81}{#1}}
\newcommand{\BuiltInTok}[1]{#1}
\newcommand{\CharTok}[1]{\textcolor[rgb]{0.31,0.60,0.02}{#1}}
\newcommand{\CommentTok}[1]{\textcolor[rgb]{0.56,0.35,0.01}{\textit{#1}}}
\newcommand{\CommentVarTok}[1]{\textcolor[rgb]{0.56,0.35,0.01}{\textbf{\textit{#1}}}}
\newcommand{\ConstantTok}[1]{\textcolor[rgb]{0.56,0.35,0.01}{#1}}
\newcommand{\ControlFlowTok}[1]{\textcolor[rgb]{0.13,0.29,0.53}{\textbf{#1}}}
\newcommand{\DataTypeTok}[1]{\textcolor[rgb]{0.13,0.29,0.53}{#1}}
\newcommand{\DecValTok}[1]{\textcolor[rgb]{0.00,0.00,0.81}{#1}}
\newcommand{\DocumentationTok}[1]{\textcolor[rgb]{0.56,0.35,0.01}{\textbf{\textit{#1}}}}
\newcommand{\ErrorTok}[1]{\textcolor[rgb]{0.64,0.00,0.00}{\textbf{#1}}}
\newcommand{\ExtensionTok}[1]{#1}
\newcommand{\FloatTok}[1]{\textcolor[rgb]{0.00,0.00,0.81}{#1}}
\newcommand{\FunctionTok}[1]{\textcolor[rgb]{0.13,0.29,0.53}{\textbf{#1}}}
\newcommand{\ImportTok}[1]{#1}
\newcommand{\InformationTok}[1]{\textcolor[rgb]{0.56,0.35,0.01}{\textbf{\textit{#1}}}}
\newcommand{\KeywordTok}[1]{\textcolor[rgb]{0.13,0.29,0.53}{\textbf{#1}}}
\newcommand{\NormalTok}[1]{#1}
\newcommand{\OperatorTok}[1]{\textcolor[rgb]{0.81,0.36,0.00}{\textbf{#1}}}
\newcommand{\OtherTok}[1]{\textcolor[rgb]{0.56,0.35,0.01}{#1}}
\newcommand{\PreprocessorTok}[1]{\textcolor[rgb]{0.56,0.35,0.01}{\textit{#1}}}
\newcommand{\RegionMarkerTok}[1]{#1}
\newcommand{\SpecialCharTok}[1]{\textcolor[rgb]{0.81,0.36,0.00}{\textbf{#1}}}
\newcommand{\SpecialStringTok}[1]{\textcolor[rgb]{0.31,0.60,0.02}{#1}}
\newcommand{\StringTok}[1]{\textcolor[rgb]{0.31,0.60,0.02}{#1}}
\newcommand{\VariableTok}[1]{\textcolor[rgb]{0.00,0.00,0.00}{#1}}
\newcommand{\VerbatimStringTok}[1]{\textcolor[rgb]{0.31,0.60,0.02}{#1}}
\newcommand{\WarningTok}[1]{\textcolor[rgb]{0.56,0.35,0.01}{\textbf{\textit{#1}}}}
\usepackage{graphicx}
\makeatletter
\def\maxwidth{\ifdim\Gin@nat@width>\linewidth\linewidth\else\Gin@nat@width\fi}
\def\maxheight{\ifdim\Gin@nat@height>\textheight\textheight\else\Gin@nat@height\fi}
\makeatother
% Scale images if necessary, so that they will not overflow the page
% margins by default, and it is still possible to overwrite the defaults
% using explicit options in \includegraphics[width, height, ...]{}
\setkeys{Gin}{width=\maxwidth,height=\maxheight,keepaspectratio}
% Set default figure placement to htbp
\makeatletter
\def\fps@figure{htbp}
\makeatother
\setlength{\emergencystretch}{3em} % prevent overfull lines
\providecommand{\tightlist}{%
  \setlength{\itemsep}{0pt}\setlength{\parskip}{0pt}}
\setcounter{secnumdepth}{-\maxdimen} % remove section numbering
\usepackage{booktabs}
\usepackage{longtable}
\usepackage{array}
\usepackage{multirow}
\usepackage{wrapfig}
\usepackage{float}
\usepackage{colortbl}
\usepackage{pdflscape}
\usepackage{tabu}
\usepackage{threeparttable}
\usepackage{threeparttablex}
\usepackage[normalem]{ulem}
\usepackage{makecell}
\usepackage{xcolor}
\ifLuaTeX
  \usepackage{selnolig}  % disable illegal ligatures
\fi
\IfFileExists{bookmark.sty}{\usepackage{bookmark}}{\usepackage{hyperref}}
\IfFileExists{xurl.sty}{\usepackage{xurl}}{} % add URL line breaks if available
\urlstyle{same}
\hypersetup{
  pdftitle={Assignment 1},
  pdfauthor={Steve Kerr (211924)},
  hidelinks,
  pdfcreator={LaTeX via pandoc}}

\title{Assignment 1}
\author{Steve Kerr (211924)}
\date{2023-09-25}

\begin{document}
\maketitle

\hypertarget{parsing-xml-text-data}{%
\subsection{Parsing XML text data}\label{parsing-xml-text-data}}

In this assignment we will access and work with German Parliamentary
data, which is available in XML format
\href{https://www.bundestag.de/services/opendata}{here} (scroll down)
for the last two parliamentary periods. Remember XML format is very like
HTML format, and we can parse it using a scraper and CSS selectors.
Speeches are contained in \texttt{\textless{}rede\textgreater{}}
elements, which each contain a paragraph element describing the speaker,
and paragraph elements recording what they said.

\hypertarget{section}{%
\subsubsection{1.1}\label{section}}

Choose one of the sessions, and retrieve it using R or Python.

\begin{quote}
Using R, we retrieve the latest available session from 22 September
2023.
\end{quote}

\begin{Shaded}
\begin{Highlighting}[]
\CommentTok{\# load packages}
\NormalTok{pacman}\SpecialCharTok{::}\FunctionTok{p\_load}\NormalTok{(}
\NormalTok{  dplyr,}
\NormalTok{  kableExtra,}
\NormalTok{  rvest,}
\NormalTok{  xml2}
\NormalTok{)}

\NormalTok{url }\OtherTok{\textless{}{-}} \StringTok{"https://www.bundestag.de/resource/blob/967706/381c8be4289b1cff2ca33f8d641d2333/20123{-}data.xml"}
\NormalTok{xml }\OtherTok{\textless{}{-}}\NormalTok{ xml2}\SpecialCharTok{::}\FunctionTok{read\_xml}\NormalTok{(url)}
\end{Highlighting}
\end{Shaded}

\hypertarget{section-1}{%
\subsubsection{1.2}\label{section-1}}

Using a scraper, get a list of all the elements.

\begin{Shaded}
\begin{Highlighting}[]
\NormalTok{rede\_list }\OtherTok{\textless{}{-}}\NormalTok{ xml }\SpecialCharTok{|\textgreater{}} 
  \FunctionTok{html\_elements}\NormalTok{(}\StringTok{"rede"}\NormalTok{)}

\NormalTok{rede\_list }\SpecialCharTok{|\textgreater{}} 
  \FunctionTok{head}\NormalTok{(}\DecValTok{5}\NormalTok{)}
\end{Highlighting}
\end{Shaded}

\begin{verbatim}
## {xml_nodeset (5)}
## [1] <rede id="ID2012300100">\n  <p klasse="redner"><a id="r1"/><redner id="11 ...
## [2] <rede id="ID2012300200">\n  <p klasse="redner"><a id="r2"/><redner id="11 ...
## [3] <rede id="ID2012300300">\n  <p klasse="redner"><a id="r3"/><redner id="11 ...
## [4] <rede id="ID2012300400">\n  <p klasse="redner"><a id="r4"/><redner id="11 ...
## [5] <rede id="ID2012300500">\n  <p klasse="redner"><a id="r5"/><redner id="11 ...
\end{verbatim}

\hypertarget{section-2}{%
\subsubsection{1.3}\label{section-2}}

For each element, get the name of the speaker, and a single string
containing everything that they said. Put this into a dataframe.

\begin{Shaded}
\begin{Highlighting}[]
\NormalTok{rede\_df }\OtherTok{\textless{}{-}}\NormalTok{ tibble}\SpecialCharTok{::}\FunctionTok{tibble}\NormalTok{(}
  \AttributeTok{speaker =} \FunctionTok{character}\NormalTok{(),}
  \AttributeTok{text =} \FunctionTok{character}\NormalTok{()}
\NormalTok{  )}

\ControlFlowTok{for}\NormalTok{ (i }\ControlFlowTok{in} \DecValTok{1}\SpecialCharTok{:}\FunctionTok{length}\NormalTok{(rede\_list)) \{}
  
  \CommentTok{\# extract speaker name}
\NormalTok{  speaker }\OtherTok{\textless{}{-}}\NormalTok{ rede\_list[i] }\SpecialCharTok{|\textgreater{}} 
    \FunctionTok{html\_elements}\NormalTok{(}\StringTok{"titel, vorname, nachname"}\NormalTok{) }\SpecialCharTok{|\textgreater{}} 
\NormalTok{    xml2}\SpecialCharTok{::}\FunctionTok{xml\_text}\NormalTok{() }\SpecialCharTok{|\textgreater{}} 
\NormalTok{    stringr}\SpecialCharTok{::}\FunctionTok{str\_c}\NormalTok{(}\AttributeTok{collapse =} \StringTok{" "}\NormalTok{)}
  
  \CommentTok{\# extract text}
\NormalTok{  text }\OtherTok{\textless{}{-}}\NormalTok{ rede\_list[i] }\SpecialCharTok{|\textgreater{}} 
    \FunctionTok{xml\_find\_all}\NormalTok{(}\StringTok{".//p[@klasse=\textquotesingle{}J\textquotesingle{}] | .//p[@klasse=\textquotesingle{}J\_1\textquotesingle{}][position() \textless{} last()]"}\NormalTok{) }\SpecialCharTok{|\textgreater{}}
    \FunctionTok{html\_text2}\NormalTok{() }\SpecialCharTok{|\textgreater{}} 
\NormalTok{    stringr}\SpecialCharTok{::}\FunctionTok{str\_c}\NormalTok{(}\AttributeTok{collapse =} \StringTok{"}\SpecialCharTok{\textbackslash{}n}\StringTok{"}\NormalTok{)}
  
  \CommentTok{\# combine into dataframe}
\NormalTok{  rede\_df }\OtherTok{\textless{}{-}}\NormalTok{ rede\_df }\SpecialCharTok{|\textgreater{}} 
\NormalTok{    dplyr}\SpecialCharTok{::}\FunctionTok{add\_row}\NormalTok{(}\AttributeTok{speaker =}\NormalTok{ speaker, }\AttributeTok{text =}\NormalTok{ text)}
  
\NormalTok{\}}

\CommentTok{\# print the top row of the data frame}
\NormalTok{rede\_df }\SpecialCharTok{|\textgreater{}} 
  \FunctionTok{slice}\NormalTok{(}\DecValTok{1}\NormalTok{) }\SpecialCharTok{|\textgreater{}} 
\NormalTok{  kableExtra}\SpecialCharTok{::}\FunctionTok{kbl}\NormalTok{() }\SpecialCharTok{|\textgreater{}} 
\NormalTok{  kableExtra}\SpecialCharTok{::}\FunctionTok{column\_spec}\NormalTok{(}\DecValTok{2}\NormalTok{, }\AttributeTok{width =} \StringTok{"300in"}\NormalTok{) }\SpecialCharTok{|\textgreater{}}  \CommentTok{\# set max width}
\NormalTok{  kableExtra}\SpecialCharTok{::}\FunctionTok{kable\_paper}\NormalTok{(}\AttributeTok{full\_width =} \ConstantTok{FALSE}\NormalTok{) }
\end{Highlighting}
\end{Shaded}

\begin{table}
\centering
\begin{tabular}[t]{l|>{\raggedright\arraybackslash}p{300in}}
\hline
speaker & text\\
\hline
Dr. Robert Habeck & Frau Präsidentin! Guten Morgen! Sehr geehrte Damen und Herren! Die Nationale Wasserstoffstrategie, vor den Sommerferien vom Kabinett verabschiedet und jetzt hier zur Beratung dem Bundestag vorgelegt, schreibt fort, was erst 2020 von der Vorgängerregierung aufs Gleis gesetzt wurde, nämlich die Entwicklung eines neuen Marktes, eines Marktes, den es im Moment nur in Ansätzen in Deutschland gibt. Nur drei Jahre später geht es jetzt darum, diese Marktentwicklung mit höherem Ambitionsniveau, mit größerer Geschwindigkeit voranzubringen.
Wenn man sich anschaut, was im Moment passiert im Bereich von Wasserstoff, so muss man sagen, dass der Zug den Bahnhof verlassen hat. Überall sind Investitionen unterwegs, überall sind Planungen im Entstehen
Die Nationale Wasserstoffstrategie steigert das Ambitionsniveau der Produktion in Deutschland bis 2030 um 100 Prozent von 5 Gigawatt auf 10 Gigawatt. Dafür ist es notwendig, die Produktionsorte zu definieren. Die Ausschreibungen für die Elektrolyse sollen noch im Jahr 2023 starten.
Dann muss der Wasserstoff transportiert werden. Die Planungen für das Wasserstoff-Kernnetz mit den Netzbetreibern befinden sich in den letzten Abstimmungsphasen und werden innerhalb von, ich würde fast sagen, Tagen, es werden vielleicht noch wenige Wochen sein, dann auch öffentlich vorgestellt. Die gesetzlichen Grundlagen dafür sind schon geschaffen.
Der regulatorische Rahmen ist nicht trivial. Man kann sich das ja leicht vorstellen: Es muss ein Netz gebaut werden für einen Energieträger, der noch nicht da ist. Trotzdem müssen die Investitionen getätigt werden; es braucht also eine Sicherheit dafür. Und umgekehrt soll das Ganze natürlich ein wirtschaftliches Projekt sein. Aber auch diese Gespräche sind im Grunde abgeschlossen und müssen nur noch formalisiert werden.
Dann haben wir das Wasserstoffnetz und die Produktion. Dann brauchen wir die Abnahme. Diese Abnahme wiederum bedeutet Bautätigkeit und Investitionen in deutsche Infrastruktur, in Unternehmen. Der Hochlauf beginnt mit großer Fahrt. – Ich sehe, dass einige klatschen wollen. Ich will das aber hier kurz und konzentriert ausführen. Ich glaube, dieses Thema ist Grund für sachliche Auseinandersetzungen. Vielen Dank.für die Handbewegungen dazu.
Was ich sagen will, ist: Die im IPCEI-Rahmen anvisierten Projekte werden mit Förderbescheiden jetzt aufs Gleis gebracht. Die Leitprojekte sind natürlich die großen Abnehmer. Das ist vor allem der Stahlmarkt. Da es öffentlich bekannt ist, darf ich das sagen: Salzgitter und thyssenkrupp haben ihre Förderbescheide bekommen. ArcelorMittal und Saarstahl werden sie sicherlich bekommen; auch da laufen die Gespräche mit der Kommission. Das sind aber nur die ganz großen Projekte. Andere mittelständische und kleinere sind ebenfalls dabei, die Förderbescheide zu bekommen.
Darüber hinaus bereiten wir die Klimaschutzverträge vor. Um es zu systematisieren: Diese IPCEI-Projekte sind die alte Fördersäule. Die Klimaschutzverträge als zusätzliches Instrument sollen dann im Rahmen von Contracts for Difference auch die OpEx-Kosten mit abdecken. Wir haben ein Interessenbekundungsverfahren sehr erfolgreich durchgeführt. Wir wissen, welche Unternehmen bis tief in den deutschen Mittelstand hinein Interesse daran haben. Wir bereiten jetzt die Ausschreibung vor. Die Ausschreibung soll noch in diesem Jahr starten. In 2024 werden dann diese Verträge abgeschlossen werden.
Wir werden, wenn wir kalkulieren, wie hoch der Wasserstoffbedarf in Deutschland im Industrie- und im Stromsektor ist, ungefähr ein Drittel dessen, was wir verbrauchen, in Deutschland produzieren können. Das heißt umgekehrt: Wir brauchen auch Importe; das ist nicht verwerflich. Ein Drittel ist weit mehr als das, was wir hier im Moment in Deutschland an Energie selber erzeugen können. Also: Die Energiesouveränität, die Unabhängigkeit wächst durch diese Strategie gegenüber der Abhängigkeit von Importen von Gas oder Öl oder Kohle, die wir jetzt von anderen Ländern bekommen.
Aber trotzdem brauchen wir Importe. Also werden wir auch in diesem Jahr noch eine Importstrategie vorlegen. Die Strategie erklärt noch einmal, was längst stattfindet. Das nationale Programm H2Global, von der Vorgängerregierung klug aufgesetzt, ist so etwas wie die Benchmark des globalen Wasserstoffeinkaufs geworden. Wir haben eine Stiftung gegründet. Die Stiftung kauft auf dem globalen Markt
Es wird übernommen von der Europäischen Union, die jetzt ein ähnliches Instrument auflegt, die sogenannte European Hydrogen Bank. Die Chancen stehen nicht schlecht, dass wir die Kräfte über H2Global bündeln, sodass diese Benchmark, die geschaffen wurde, tatsächlich in der Skalierung größer wird. Ich glaube, das ist eine reine Win-win-Situation. Europa profitiert von der Vorreiterrolle, die Deutschland da eingenommen hat. Deutschland profitiert, wenn der Markt größer wird. Wir werden durch den Skaleneffekt natürlich ein noch bedeutenderer Marktplayer werden.
Nimmt man das alles zusammen, sieht man, dass über die Wasserstoffstrategie, die Verteilung, die Produktion und die Abnahme ein großer industrie- und wirtschaftspolitischer Impuls ausgelöst wird. Die Elektrolyseure zu bauen und zu exportieren, ist für den deutschen Maschinenbau ein Riesengeschäftsfeld. Kraftwerke zu bauen, die nicht nur Gas verbrennen, sondern auch Wasserstoff-ready sind oder reine Wasserstoffkraftwerke sind, ist Benchmark für die Kraftwerksstrategien auch anderer Länder für die Zukunft. Stahl zu produzieren, der durch Wasserstoff grün wird, Chemie zu produzieren, die grünen Wasserstoff nutzt und sich damit dekarbonisiert, all das wird die Wertschöpfung in diesem Land – vom Maschinenbau bis zur Produktion – enorm steigern.
Die Wasserstoffstrategie startet als Beitrag zum Klimaschutz in den Bereichen, die nicht elektrisch durchdrungen werden können, und sie wird enden als großes Wirtschaftsimpulsprogramm für diese Republik und für Europa.
Vielen Dank.\\
\hline
\end{tabular}
\end{table}

\hypertarget{section-3}{%
\subsubsection{2.1}\label{section-3}}

Choose a politician, and print the number of speeches they made in this
session

\begin{Shaded}
\begin{Highlighting}[]
\CommentTok{\# let\textquotesingle{}s check out Steffi Lemke}
\CommentTok{\#politician \textless{}{-} "Steffi Lemke"}

\CommentTok{\# calculate the number of speeches given by Steffi Lemke in this particular session}
\NormalTok{rede\_df }\SpecialCharTok{|\textgreater{}} 
  \FunctionTok{filter}\NormalTok{(stringr}\SpecialCharTok{::}\FunctionTok{str\_detect}\NormalTok{(speaker, }\StringTok{"Steffi Lemke"}\NormalTok{)) }\SpecialCharTok{|\textgreater{}} 
  \FunctionTok{nrow}\NormalTok{()}
\end{Highlighting}
\end{Shaded}

\begin{verbatim}
## [1] 1
\end{verbatim}

\hypertarget{section-4}{%
\subsubsection{2.2}\label{section-4}}

Print the content of the first speech by the politician you choose.

\begin{Shaded}
\begin{Highlighting}[]
\NormalTok{rede\_df }\SpecialCharTok{|\textgreater{}} 
  \FunctionTok{filter}\NormalTok{(stringr}\SpecialCharTok{::}\FunctionTok{str\_detect}\NormalTok{(speaker, }\StringTok{"Steffi Lemke"}\NormalTok{)) }\SpecialCharTok{|\textgreater{}} 
  \FunctionTok{slice}\NormalTok{(}\DecValTok{1}\NormalTok{) }\SpecialCharTok{|\textgreater{}} 
  \FunctionTok{pull}\NormalTok{(text) }\SpecialCharTok{|\textgreater{}} 
  \FunctionTok{cat}\NormalTok{()}
\end{Highlighting}
\end{Shaded}

\begin{verbatim}
## Liebe Kolleginnen und Kollegen! Sehr geehrter Herr Präsident! Man kann über den Schutz von Weidetierhaltung und den Schutz des Wolfes auf zweierlei Art diskutieren. Man kann es populistisch tun, man kann Anlehnung an Rechtsextremisten dabei nehmen,
## Die Weidetierhalter haben in der Realität aufgrund der Zunahme der Wolfszahlen in den letzten Jahren ein Problem, das wir lösen müssen. Ich finde, das sollten wir im Bundestag auch gemeinsam feststellen können, um auf dieser Grundlage dann über die Lösungen zu diskutieren.
## Um es noch hinzuzufügen: Ich lebe in einer ländlichen Region, in der der Wolf vorkommt und wo Spaziergänger sagen, dass sie ihn gesichtet haben. Ich bin im Sommer zu einem Schäfer hinausgefahren und habe handfest ausprobiert, wie sich die verschiedenen Zaunhöhen anfühlen, wenn man den Zaun dort tatsächlich setzt, weil ich verstehen will, was es bedeutet, wenn wir von wolfssicheren Zäunen reden.
## Die Lösungen, die wir brauchen, müssen vor allem praxistauglich sein.
## Diejenigen, die etwas vom Bundesnaturschutzgesetz und von Naturschutzregelungen verstehen, wissen, dass es sehr viel Zeit in Anspruch nehmen wird, wenn wir nach Brüssel gehen und dort auf Lösungen zu diesem Thema warten. Ich möchte, dass wir den Weidetierhaltern schneller helfen, als so lange auf Brüssel zu warten.
## Außerdem möchte ich, dass wir einheitliche Regelungen schaffen und den Wirrwarr zwischen den 16 Bundesländern aufheben, der da zum Teil existiert, den die Bundesländer aber nicht zu verschulden haben. Deshalb rede ich mit allen 16 Bundesländern über diese praxistauglichen Lösungen. Wir brauchen unkomplizierte Lösungen. Regionen großflächig einzuzäunen, um wolfsfreie Zonen zu schaffen, das halte ich für keine gute Lösung. Es wäre weder unkompliziert noch realisierbar.
## Es ist richtig, dass es für alle Weidetierhalter eine schlimme Belastung ist, wenn sie auf die Weide kommen und dort gerissene Tiere vorfinden. Das ist wahr. Aber es ist auch wahr, dass der Wolf ein Säugetier ist, ein Tier, das Schmerzen empfindet, das in einem Familienverband lebt.
## Vielen Dank.
## Nächster Redner ist der Kollege Klaus Mack, CDU/CSU-Fraktion.
\end{verbatim}

\hypertarget{section-5}{%
\subsubsection{2.3}\label{section-5}}

Process the list of speeches into a TFIDF matrix. What are the highest
scoring terms in this matrix for the first speech by the politician you
have chosen?

\begin{Shaded}
\begin{Highlighting}[]
\FunctionTok{library}\NormalTok{(quanteda)}
\end{Highlighting}
\end{Shaded}

\begin{verbatim}
## Warning in .recacheSubclasses(def@className, def, env): undefined subclass
## "pcorMatrix" of class "replValueSp"; definition not updated
\end{verbatim}

\begin{verbatim}
## Warning in .recacheSubclasses(def@className, def, env): undefined subclass
## "pcorMatrix" of class "xMatrix"; definition not updated
\end{verbatim}

\begin{verbatim}
## Warning in .recacheSubclasses(def@className, def, env): undefined subclass
## "pcorMatrix" of class "mMatrix"; definition not updated
\end{verbatim}

\begin{verbatim}
## Package version: 3.3.1
## Unicode version: 14.0
## ICU version: 71.1
\end{verbatim}

\begin{verbatim}
## Parallel computing: 8 of 8 threads used.
\end{verbatim}

\begin{verbatim}
## See https://quanteda.io for tutorials and examples.
\end{verbatim}

\begin{Shaded}
\begin{Highlighting}[]
\FunctionTok{library}\NormalTok{(tidytext)}

\NormalTok{dfm\_tidy }\OtherTok{\textless{}{-}}\NormalTok{ rede\_df }\SpecialCharTok{|\textgreater{}} 
  \FunctionTok{corpus}\NormalTok{() }\SpecialCharTok{|\textgreater{}} 
  \FunctionTok{tokens}\NormalTok{() }\SpecialCharTok{|\textgreater{}} 
  \FunctionTok{tokens\_remove}\NormalTok{(}\AttributeTok{pattern =} \FunctionTok{stopwords}\NormalTok{(}\StringTok{"de"}\NormalTok{)) }\SpecialCharTok{|\textgreater{}} 
  \FunctionTok{tokens\_wordstem}\NormalTok{(}\AttributeTok{language =} \StringTok{"german"}\NormalTok{) }\SpecialCharTok{|\textgreater{}} 
  \FunctionTok{dfm}\NormalTok{() }\SpecialCharTok{|\textgreater{}} 
  \FunctionTok{dfm\_tfidf}\NormalTok{() }\SpecialCharTok{|\textgreater{}} 
\NormalTok{  tidytext}\SpecialCharTok{::}\FunctionTok{tidy}\NormalTok{()}

\NormalTok{dfm\_tidy }\SpecialCharTok{|\textgreater{}} 
  \FunctionTok{filter}\NormalTok{(document }\SpecialCharTok{==} \FunctionTok{paste0}\NormalTok{(}\StringTok{"text"}\NormalTok{, }\FunctionTok{which}\NormalTok{(rede\_df}\SpecialCharTok{$}\NormalTok{speaker }\SpecialCharTok{==} \StringTok{"Steffi Lemke"}\NormalTok{))) }\SpecialCharTok{|\textgreater{}} \CommentTok{\# 2x check her speech to make sure it worked correctly}
  \FunctionTok{select}\NormalTok{(}\SpecialCharTok{{-}}\NormalTok{document) }\SpecialCharTok{|\textgreater{}} 
  \FunctionTok{arrange}\NormalTok{(}\FunctionTok{desc}\NormalTok{(count)) }\SpecialCharTok{|\textgreater{}} 
  \FunctionTok{head}\NormalTok{(}\DecValTok{10}\NormalTok{) }\SpecialCharTok{|\textgreater{}} 
\NormalTok{  kableExtra}\SpecialCharTok{::}\FunctionTok{kbl}\NormalTok{() }\SpecialCharTok{|\textgreater{}} 
\NormalTok{  kableExtra}\SpecialCharTok{::}\FunctionTok{kable\_styling}\NormalTok{()}
\end{Highlighting}
\end{Shaded}

\begin{table}
\centering
\begin{tabular}[t]{l|r}
\hline
term & count\\
\hline
weidetierhalt & 4.357602\\
\hline
losung & 3.934469\\
\hline
praxistaug & 3.868997\\
\hline
wahr & 3.868997\\
\hline
bundesland & 3.706585\\
\hline
unkompliziert & 3.266937\\
\hline
brussel & 3.266937\\
\hline
tier & 2.914754\\
\hline
wolf & 2.803495\\
\hline
wart & 2.664877\\
\hline
\end{tabular}
\end{table}

\hypertarget{section-6}{%
\subsubsection{2.4}\label{section-6}}

Using the resource ``Stammdaten aller Abgeordneten seit 1949 im
XML-Format'', retrieve the records pertaining to your chosen politician
and print the information they contain.

\begin{Shaded}
\begin{Highlighting}[]
\FunctionTok{library}\NormalTok{(tidyr)}
\DocumentationTok{\#\#\# how to read in XML file?}
\CommentTok{\# url \textless{}{-} "https://www.bundestag.de/resource/blob/472878/a4859899e44a7cab1a8233e5dd69f2f3/MdB{-}Stammdaten{-}data.zip"}
\NormalTok{mdb\_raw }\OtherTok{\textless{}{-}}\NormalTok{ here}\SpecialCharTok{::}\FunctionTok{here}\NormalTok{(}\StringTok{"Assignments{-}COMPLETED/MDB\_STAMMDATEN.XML"}\NormalTok{) }\SpecialCharTok{|\textgreater{}}  \CommentTok{\# this wont work if he tries to run the code...}
\NormalTok{  xml2}\SpecialCharTok{::}\FunctionTok{read\_xml}\NormalTok{() }\SpecialCharTok{|\textgreater{}} 
\NormalTok{  xml2}\SpecialCharTok{::}\FunctionTok{xml\_find\_all}\NormalTok{(}\StringTok{"//MDB[NAMEN/NAME/NACHNAME/text() = \textquotesingle{}Lemke\textquotesingle{}]"}\NormalTok{) }\SpecialCharTok{|\textgreater{}} 
\NormalTok{  xml2}\SpecialCharTok{::}\FunctionTok{as\_list}\NormalTok{()}

\NormalTok{mdb\_list }\OtherTok{\textless{}{-}}\NormalTok{ mdb\_raw }\SpecialCharTok{|\textgreater{}} 
\NormalTok{  tibble}\SpecialCharTok{::}\FunctionTok{as\_tibble\_col}\NormalTok{() }\SpecialCharTok{|\textgreater{}} 
\NormalTok{  tidyr}\SpecialCharTok{::}\FunctionTok{unnest\_wider}\NormalTok{(}\StringTok{"value"}\NormalTok{)}

\NormalTok{unlist\_all }\OtherTok{\textless{}{-}} \ControlFlowTok{function}\NormalTok{(x) \{ }\CommentTok{\# found this function online}
\NormalTok{  x[}\FunctionTok{sapply}\NormalTok{(x, is.null)] }\OtherTok{\textless{}{-}} \ConstantTok{NA}
  \FunctionTok{unlist}\NormalTok{(x)}
\NormalTok{\}}

\CommentTok{\# name info}
\NormalTok{name\_df }\OtherTok{\textless{}{-}}\NormalTok{ mdb\_list }\SpecialCharTok{|\textgreater{}} 
  \FunctionTok{select}\NormalTok{(ID, NAMEN) }\SpecialCharTok{|\textgreater{}} 
\NormalTok{  tidyr}\SpecialCharTok{::}\FunctionTok{unnest\_longer}\NormalTok{(}\StringTok{"NAMEN"}\NormalTok{, }\AttributeTok{indices\_include =} \ConstantTok{FALSE}\NormalTok{) }\SpecialCharTok{|\textgreater{}} 
\NormalTok{  tidyr}\SpecialCharTok{::}\FunctionTok{unnest\_wider}\NormalTok{(}\StringTok{"NAMEN"}\NormalTok{) }\SpecialCharTok{|\textgreater{}} 
\NormalTok{  dplyr}\SpecialCharTok{::}\FunctionTok{mutate}\NormalTok{(dplyr}\SpecialCharTok{::}\FunctionTok{across}\NormalTok{(}\AttributeTok{.cols =}\NormalTok{ tidyselect}\SpecialCharTok{::}\FunctionTok{everything}\NormalTok{(),}
                              \AttributeTok{.fns =}\NormalTok{ unlist\_all)) }\SpecialCharTok{|\textgreater{}} 
\NormalTok{  janitor}\SpecialCharTok{::}\FunctionTok{clean\_names}\NormalTok{() }\SpecialCharTok{|\textgreater{}} 
  \FunctionTok{select\_if}\NormalTok{(}\SpecialCharTok{\textasciitilde{}!}\FunctionTok{all}\NormalTok{(}\FunctionTok{is.na}\NormalTok{(.)))}

\CommentTok{\# print table}
\NormalTok{name\_df }\SpecialCharTok{|\textgreater{}} 
  \FunctionTok{kbl}\NormalTok{() }\SpecialCharTok{|\textgreater{}} 
  \FunctionTok{kable\_paper}\NormalTok{()}
\end{Highlighting}
\end{Shaded}

\begin{table}
\centering
\begin{tabular}[t]{l|l|l|l}
\hline
id & nachname & vorname & historie\_von\\
\hline
11002720 & Lemke & Steffi & 10.11.1994\\
\hline
\end{tabular}
\end{table}

\begin{Shaded}
\begin{Highlighting}[]
\CommentTok{\# bio info}
\NormalTok{bio\_df }\OtherTok{\textless{}{-}}\NormalTok{ mdb\_list }\SpecialCharTok{|\textgreater{}} 
  \FunctionTok{select}\NormalTok{(ID, BIOGRAFISCHE\_ANGABEN) }\SpecialCharTok{|\textgreater{}} 
\NormalTok{  tidyr}\SpecialCharTok{::}\FunctionTok{unnest\_wider}\NormalTok{(}\StringTok{"BIOGRAFISCHE\_ANGABEN"}\NormalTok{) }\SpecialCharTok{|\textgreater{}} 
\NormalTok{  dplyr}\SpecialCharTok{::}\FunctionTok{mutate}\NormalTok{(dplyr}\SpecialCharTok{::}\FunctionTok{across}\NormalTok{(}\AttributeTok{.cols =}\NormalTok{ tidyselect}\SpecialCharTok{::}\FunctionTok{everything}\NormalTok{(),}
                              \AttributeTok{.fns =}\NormalTok{ unlist\_all)) }\SpecialCharTok{|\textgreater{}} 
\NormalTok{  janitor}\SpecialCharTok{::}\FunctionTok{clean\_names}\NormalTok{() }\SpecialCharTok{|\textgreater{}} 
  \FunctionTok{select\_if}\NormalTok{(}\SpecialCharTok{\textasciitilde{}!}\FunctionTok{all}\NormalTok{(}\FunctionTok{is.na}\NormalTok{(.)))}

\CommentTok{\# print table}
\NormalTok{bio\_df }\SpecialCharTok{|\textgreater{}} 
  \FunctionTok{kbl}\NormalTok{() }\SpecialCharTok{|\textgreater{}} 
  \FunctionTok{kable\_paper}\NormalTok{()}
\end{Highlighting}
\end{Shaded}

\begin{table}
\centering
\begin{tabular}[t]{l|l|l|l|l|l|l|l|l}
\hline
id & geburtsdatum & geburtsort & geschlecht & familienstand & religion & beruf & partei\_kurz & vita\_kurz\\
\hline
11002720 & 19.01.1968 & Dessau & weiblich & geschieden, 1 Kind & ohne Angaben & Dipl.-Agraringenieurin & BÜNDNIS 90/DIE GRÜNEN & 1974/84 POS Dessau. 1984/86 Ausbildung zur Zootechnikerin. 1986/88 Tätigkeit als Briefträgerin. 1986/88 Abendschule und Abitur an der Kreisvolkshochschule, Dessau. 1988/93 Studium der Agrarwissenschaften Humboldt-Univ. Berlin. 1993/94 Fraktionsgeschäftsführerin der Stadtratsfraktion Bürger/Forum/Grüne in Dessau. Seit 1990 Mitgl. BÜNDNIS 90/DIE GRÜNEN, 1993/94 Mitgl. Landesvorst. BÜNDNIS 90/DIE GRÜNEN Sachsen-Anhalt; 2002/13 Politische Bundesgeschäftsführerin von BÜNDNIS 90/DIE GRÜNEN. Seit Dez. 2021 Bundesministerin für Umwelt, Naturschutz, nukleare Sicherheit und Verbraucherschutz. - MdB 1994/2002 und seit Okt. 2013, 1998/2002 und 2013/21 Parl. Geschäftsführerin der Fraktion BÜNDNIS 90/DIE GRÜNEN und Sprecherin für Naturschutz, 2017/21 Mitgl. Ältestenrat.\\
\hline
\end{tabular}
\end{table}

\begin{Shaded}
\begin{Highlighting}[]
\CommentTok{\# election info}
\NormalTok{elec\_df }\OtherTok{\textless{}{-}}\NormalTok{ mdb\_list }\SpecialCharTok{\%\textgreater{}\%}
\NormalTok{    dplyr}\SpecialCharTok{::}\FunctionTok{select}\NormalTok{(}\StringTok{"ID"}\NormalTok{, }\StringTok{"WAHLPERIODEN"}\NormalTok{) }\SpecialCharTok{\%\textgreater{}\%} \DocumentationTok{\#\#\# update pipes for consistency}
\NormalTok{    tidyr}\SpecialCharTok{::}\FunctionTok{unnest\_longer}\NormalTok{(}\StringTok{"WAHLPERIODEN"}\NormalTok{, }\AttributeTok{indices\_include =} \ConstantTok{FALSE}\NormalTok{) }\SpecialCharTok{\%\textgreater{}\%}
\NormalTok{    tidyr}\SpecialCharTok{::}\FunctionTok{unnest\_wider}\NormalTok{(}\StringTok{"WAHLPERIODEN"}\NormalTok{) }\SpecialCharTok{\%\textgreater{}\%}
\NormalTok{    dplyr}\SpecialCharTok{::}\FunctionTok{mutate}\NormalTok{(dplyr}\SpecialCharTok{::}\FunctionTok{across}\NormalTok{(}\AttributeTok{.cols =} \FunctionTok{c}\NormalTok{(}\StringTok{"ID"}\NormalTok{,}
                                          \StringTok{"WP"}\NormalTok{,}
                                          \StringTok{"MDBWP\_VON"}\NormalTok{,}
                                          \StringTok{"MDBWP\_BIS"}\NormalTok{,}
                                          \StringTok{"WKR\_NUMMER"}\NormalTok{,}
                                          \StringTok{"WKR\_NAME"}\NormalTok{,}
                                          \StringTok{"WKR\_LAND"}\NormalTok{,}
                                          \StringTok{"LISTE"}\NormalTok{,}
                                          \StringTok{"MANDATSART"}\NormalTok{),}
                                \AttributeTok{.fns =}\NormalTok{ unlist\_all)) }\SpecialCharTok{|\textgreater{}} 
\NormalTok{    janitor}\SpecialCharTok{::}\FunctionTok{clean\_names}\NormalTok{() }\SpecialCharTok{|\textgreater{}} 
    \FunctionTok{select\_if}\NormalTok{(}\SpecialCharTok{\textasciitilde{}!}\FunctionTok{all}\NormalTok{(}\FunctionTok{is.na}\NormalTok{(.)))}

\CommentTok{\# print table}
\NormalTok{elec\_df }\SpecialCharTok{|\textgreater{}} 
  \FunctionTok{kbl}\NormalTok{() }\SpecialCharTok{|\textgreater{}} 
  \FunctionTok{kable\_paper}\NormalTok{()}
\end{Highlighting}
\end{Shaded}

\begin{table}
\centering
\begin{tabular}[t]{l|l|l|l|l|l|l|l|l|l}
\hline
id & wp & mdbwp\_von & mdbwp\_bis & wkr\_nummer & wkr\_name & wkr\_land & liste & mandatsart & institutionen\\
\hline
11002720 & 13 & 10.11.1994 & 26.10.1998 & 289 & Dessau - Bitterfeld & SAA & SAA & Landesliste & Fraktion/Gruppe               , Fraktion BÜNDNIS 90/DIE GRÜNEN, 10.11.1994\\
\hline
11002720 & 14 & 26.10.1998 & 17.10.2002 & 289 & Dessau - Bitterfeld & SAA & SAA & Landesliste & Fraktion/Gruppe                   , Fraktion BÜNDNIS 90/DIE GRÜNEN    , 26.10.1998                        , Parlamentarische Geschäftsführerin, 27.10.1998\\
\hline
11002720 & 18 & 22.10.2013 & 24.10.2017 & NA & NA & NA & ST & Landesliste & Fraktion/Gruppe                   , Fraktion BÜNDNIS 90/DIE GRÜNEN    , 22.10.2013                        , Parlamentarische Geschäftsführerin, 22.10.2013\\
\hline
11002720 & 19 & 24.10.2017 & 26.10.2021 & NA & NA & NA & ST & Landesliste & Fraktion/Gruppe                   , Fraktion BÜNDNIS 90/DIE GRÜNEN    , 24.10.2017                        , Parlamentarische Geschäftsführerin, 24.01.2018\\
\hline
11002720 & 20 & 26.10.2021 & NA & NA & NA & NA & ST & Landesliste & Fraktion/Gruppe                                                                     , Fraktion BÜNDNIS 90/DIE GRÜNEN                                                      , 26.10.2021                                                                          , Parlamentarische Geschäftsführerin                                                  , 26.10.2021                                                                          , 07.12.2021                                                                          , Ministerium                                                                         , Bundesministerium für Umwelt, Naturschutz, nukleare Sicherheit und Verbraucherschutz, 08.12.2021                                                                          , Bundesministerin                                                                    , 08.12.2021\\
\hline
\end{tabular}
\end{table}

\end{document}
