% Options for packages loaded elsewhere
\PassOptionsToPackage{unicode}{hyperref}
\PassOptionsToPackage{hyphens}{url}
\PassOptionsToPackage{dvipsnames,svgnames,x11names}{xcolor}
%
\documentclass[
]{article}
\usepackage{amsmath,amssymb}
\usepackage{iftex}
\ifPDFTeX
  \usepackage[T1]{fontenc}
  \usepackage[utf8]{inputenc}
  \usepackage{textcomp} % provide euro and other symbols
\else % if luatex or xetex
  \usepackage{unicode-math} % this also loads fontspec
  \defaultfontfeatures{Scale=MatchLowercase}
  \defaultfontfeatures[\rmfamily]{Ligatures=TeX,Scale=1}
\fi
\usepackage{lmodern}
\ifPDFTeX\else
  % xetex/luatex font selection
\fi
% Use upquote if available, for straight quotes in verbatim environments
\IfFileExists{upquote.sty}{\usepackage{upquote}}{}
\IfFileExists{microtype.sty}{% use microtype if available
  \usepackage[]{microtype}
  \UseMicrotypeSet[protrusion]{basicmath} % disable protrusion for tt fonts
}{}
\makeatletter
\@ifundefined{KOMAClassName}{% if non-KOMA class
  \IfFileExists{parskip.sty}{%
    \usepackage{parskip}
  }{% else
    \setlength{\parindent}{0pt}
    \setlength{\parskip}{6pt plus 2pt minus 1pt}}
}{% if KOMA class
  \KOMAoptions{parskip=half}}
\makeatother
\usepackage{xcolor}
\usepackage[margin=1in]{geometry}
\usepackage{color}
\usepackage{fancyvrb}
\newcommand{\VerbBar}{|}
\newcommand{\VERB}{\Verb[commandchars=\\\{\}]}
\DefineVerbatimEnvironment{Highlighting}{Verbatim}{commandchars=\\\{\}}
% Add ',fontsize=\small' for more characters per line
\usepackage{framed}
\definecolor{shadecolor}{RGB}{248,248,248}
\newenvironment{Shaded}{\begin{snugshade}}{\end{snugshade}}
\newcommand{\AlertTok}[1]{\textcolor[rgb]{0.94,0.16,0.16}{#1}}
\newcommand{\AnnotationTok}[1]{\textcolor[rgb]{0.56,0.35,0.01}{\textbf{\textit{#1}}}}
\newcommand{\AttributeTok}[1]{\textcolor[rgb]{0.13,0.29,0.53}{#1}}
\newcommand{\BaseNTok}[1]{\textcolor[rgb]{0.00,0.00,0.81}{#1}}
\newcommand{\BuiltInTok}[1]{#1}
\newcommand{\CharTok}[1]{\textcolor[rgb]{0.31,0.60,0.02}{#1}}
\newcommand{\CommentTok}[1]{\textcolor[rgb]{0.56,0.35,0.01}{\textit{#1}}}
\newcommand{\CommentVarTok}[1]{\textcolor[rgb]{0.56,0.35,0.01}{\textbf{\textit{#1}}}}
\newcommand{\ConstantTok}[1]{\textcolor[rgb]{0.56,0.35,0.01}{#1}}
\newcommand{\ControlFlowTok}[1]{\textcolor[rgb]{0.13,0.29,0.53}{\textbf{#1}}}
\newcommand{\DataTypeTok}[1]{\textcolor[rgb]{0.13,0.29,0.53}{#1}}
\newcommand{\DecValTok}[1]{\textcolor[rgb]{0.00,0.00,0.81}{#1}}
\newcommand{\DocumentationTok}[1]{\textcolor[rgb]{0.56,0.35,0.01}{\textbf{\textit{#1}}}}
\newcommand{\ErrorTok}[1]{\textcolor[rgb]{0.64,0.00,0.00}{\textbf{#1}}}
\newcommand{\ExtensionTok}[1]{#1}
\newcommand{\FloatTok}[1]{\textcolor[rgb]{0.00,0.00,0.81}{#1}}
\newcommand{\FunctionTok}[1]{\textcolor[rgb]{0.13,0.29,0.53}{\textbf{#1}}}
\newcommand{\ImportTok}[1]{#1}
\newcommand{\InformationTok}[1]{\textcolor[rgb]{0.56,0.35,0.01}{\textbf{\textit{#1}}}}
\newcommand{\KeywordTok}[1]{\textcolor[rgb]{0.13,0.29,0.53}{\textbf{#1}}}
\newcommand{\NormalTok}[1]{#1}
\newcommand{\OperatorTok}[1]{\textcolor[rgb]{0.81,0.36,0.00}{\textbf{#1}}}
\newcommand{\OtherTok}[1]{\textcolor[rgb]{0.56,0.35,0.01}{#1}}
\newcommand{\PreprocessorTok}[1]{\textcolor[rgb]{0.56,0.35,0.01}{\textit{#1}}}
\newcommand{\RegionMarkerTok}[1]{#1}
\newcommand{\SpecialCharTok}[1]{\textcolor[rgb]{0.81,0.36,0.00}{\textbf{#1}}}
\newcommand{\SpecialStringTok}[1]{\textcolor[rgb]{0.31,0.60,0.02}{#1}}
\newcommand{\StringTok}[1]{\textcolor[rgb]{0.31,0.60,0.02}{#1}}
\newcommand{\VariableTok}[1]{\textcolor[rgb]{0.00,0.00,0.00}{#1}}
\newcommand{\VerbatimStringTok}[1]{\textcolor[rgb]{0.31,0.60,0.02}{#1}}
\newcommand{\WarningTok}[1]{\textcolor[rgb]{0.56,0.35,0.01}{\textbf{\textit{#1}}}}
\usepackage{graphicx}
\makeatletter
\def\maxwidth{\ifdim\Gin@nat@width>\linewidth\linewidth\else\Gin@nat@width\fi}
\def\maxheight{\ifdim\Gin@nat@height>\textheight\textheight\else\Gin@nat@height\fi}
\makeatother
% Scale images if necessary, so that they will not overflow the page
% margins by default, and it is still possible to overwrite the defaults
% using explicit options in \includegraphics[width, height, ...]{}
\setkeys{Gin}{width=\maxwidth,height=\maxheight,keepaspectratio}
% Set default figure placement to htbp
\makeatletter
\def\fps@figure{htbp}
\makeatother
\setlength{\emergencystretch}{3em} % prevent overfull lines
\providecommand{\tightlist}{%
  \setlength{\itemsep}{0pt}\setlength{\parskip}{0pt}}
\setcounter{secnumdepth}{-\maxdimen} % remove section numbering
\usepackage{booktabs}
\usepackage{xcolor}
\ifLuaTeX
  \usepackage{selnolig}  % disable illegal ligatures
\fi
\usepackage[]{natbib}
\bibliographystyle{plainnat}
\IfFileExists{bookmark.sty}{\usepackage{bookmark}}{\usepackage{hyperref}}
\IfFileExists{xurl.sty}{\usepackage{xurl}}{} % add URL line breaks if available
\urlstyle{same}
\hypersetup{
  pdftitle={Assignment 4},
  pdfauthor={Steve Kerr (211924)},
  colorlinks=true,
  linkcolor={Maroon},
  filecolor={Maroon},
  citecolor={Blue},
  urlcolor={blue},
  pdfcreator={LaTeX via pandoc}}

\title{Assignment 4}
\author{Steve Kerr (211924)}
\date{2023-10-23}

\begin{document}
\maketitle

\hypertarget{introduction}{%
\section{Introduction}\label{introduction}}

In this assignment, you are asked to use topic modelling to investigate
manifestos from the manifesto project maintained by
\href{https://manifesto-project.wzb.eu/}{WZB}. You can either use the UK
manifestos we looked at together in class, or collect your own set of
manifestos by choosing the country/countries, year/years and
party/parties you are interested in. You should produce a report which
includes your code, that addresses the following aspects of creating a
topic model, making sure to answer the questions below.

This time, you will be assessed not only on whether the code gets the
right result, but on how you understand and communicate your
understanding of the modelling process and how this can answer your
research question. The best research question is one that is interesting
and answerable, but the most important thing is that the research
question is answerable with the methods you choose.

You will also be assessed on the presentation of your results, and on
the concision and readability of your code.

\begin{Shaded}
\begin{Highlighting}[]
\CommentTok{\# load packages}
\NormalTok{pacman}\SpecialCharTok{::}\FunctionTok{p\_load}\NormalTok{(}
\NormalTok{  here,}
\NormalTok{  lubridate,}
\NormalTok{  manifestoR,}
\NormalTok{  readr}
\NormalTok{  )}
\end{Highlighting}
\end{Shaded}

\hypertarget{data-acquisition-description-and-preparation}{%
\subsection{1. Data acquisition, description, and
preparation}\label{data-acquisition-description-and-preparation}}

Bring together a dataset from the WZB.

\begin{Shaded}
\begin{Highlighting}[]
\CommentTok{\# define file path for data}
\NormalTok{data\_path }\OtherTok{\textless{}{-}} \FunctionTok{here}\NormalTok{(}\StringTok{"Assignments\_SK/Assignment{-}4/data/corpus.rds"}\NormalTok{) }

\CommentTok{\# download data if not yet downloaded}
\ControlFlowTok{if}\NormalTok{ (}\SpecialCharTok{!}\FunctionTok{file.exists}\NormalTok{(data\_path)) \{}
  
  \CommentTok{\# set API key}
  \FunctionTok{mp\_setapikey}\NormalTok{(}\FunctionTok{here}\NormalTok{(}\StringTok{"Assignments\_SK/Assignment{-}4/manifesto\_apikey.txt"}\NormalTok{))}
  
  \CommentTok{\# query data}
\NormalTok{  corpus }\OtherTok{\textless{}{-}} \FunctionTok{mp\_corpus}\NormalTok{(countryname }\SpecialCharTok{==} \StringTok{"United States"} \SpecialCharTok{\&}\NormalTok{ edate }\SpecialCharTok{\textgreater{}} \FunctionTok{as\_date}\NormalTok{(}\StringTok{"2000{-}01{-}01"}\NormalTok{))}
  
  \CommentTok{\# save data}
  \FunctionTok{saveRDS}\NormalTok{(corpus, }\FunctionTok{here}\NormalTok{(}\StringTok{"Assignments\_SK/Assignment{-}4/data/corpus.rds"}\NormalTok{))}
  
\NormalTok{\} }\ControlFlowTok{else}\NormalTok{ \{}
  
  \CommentTok{\# load data}
\NormalTok{  corpus }\OtherTok{\textless{}{-}} \FunctionTok{read\_rds}\NormalTok{(data\_path)}
  
\NormalTok{\}}

\CommentTok{\# inspect corupus}
\NormalTok{corpus}
\end{Highlighting}
\end{Shaded}

\begin{verbatim}
## <<ManifestoCorpus>>
## Metadata:  corpus specific: 0, document level (indexed): 0
## Content:  documents: 12
\end{verbatim}

\begin{Shaded}
\begin{Highlighting}[]
\CommentTok{\# explore data}
\FunctionTok{head}\NormalTok{(}\FunctionTok{content}\NormalTok{(corpus[[}\DecValTok{1}\NormalTok{]])) }\CommentTok{\# view beginning of text of first manifesto}
\end{Highlighting}
\end{Shaded}

\begin{verbatim}
## [1] "The 2000 Democratic National Platform: Prosperity, Progress, and Peace  as adopted by the 2000 Democratic National Convention on August 15, 2000  INTRODUCTION Today, America finds itself in the midst of prosperity, progress, and peace. We have arrived at this moment because of the hard work of the American people. This election will be about the big choices we have to make to secure prosperity that is broadly shared and progress that reaches all families in this new American century. In the year 2000, the Democratic Party stands ready to meet that challenge and to build on our achievements. When Thomas Jefferson was elected as our Party's first president in 1800, America was a young country trying to find its place in the world. Two hundred years later, Democrats gather at a moment of vast possibility to nominate Al Gore as America's next president. A new economy founded on the force of new technologies and traditional values of work is giving rise to new industries and transforming old ones. Biological breakthroughs give us the chance to unlock the mysteries of humanity's deadliest plagues. While the globe is still beset with tragedies and difficulties, more people live under governments of freedom, liberty, and democracy than ever before in history. America enjoys unparalleled affluence at home and influence abroad. Yet this moment is clearly one of possibility, not absolute guarantees. We must remember that our achievements were accomplished only with creativity, courage, and conscience; with a willingness to innovate and imagine; and with a recommitment to our basic American values of hard work, community, embracing diversity, faith, family, and personal responsibility. And all of it can be imperiled again. Let us not forget that America's future did not always seem so bright. Under the Bush-Quayle administration, America was suffering through economic stagnation. Businesses were failing. Jobs were disappearing. The welfare rolls swelled. Crime exploded in the streets. Hope and optimism were scarce. Most Americans felt that the American Dream was endangered - if not extinct. But in 1992, Americans elected Bill Clinton and Al Gore with a mandate to turn America around. And that's just what they did. They took on the old thinking that had come to dominate politics and offered new ideas - new ideas that met the challenges of the day, new ideas that kept faith with America's oldest values, new ideas that worked. Eight years later the record is clear: the longest economic expansion in American history. The most jobs ever created under a single administration. The first real wage growth in 20 years. The highest home ownership rate ever. The lowest African-American and Hispanic-American unemployment rates in American history. The lowest crime rate in 25 years. The lowest number of people on welfare since the 1960's. The largest drop in poverty in nearly 30 years. The lowest level of child poverty in 20 years. And after 15 painful years when the rich were getting richer and the poor were getting poorer, America is finally growing together instead of growing apart. These are accomplishments, not accidents. They came about because Democrats - from the White House, to the Congress, to State Houses all across America - brought new thinking and new action to our most pressing challenges. We used government as a catalyst to engage the best ideas and energies of the American people. We asked citizens to get involved and they did. They tutored in their children's schools, patrolled on neighborhood crime watches, volunteered in local hospitals, and voiced their opinion on every issue. They shaped effective solutions to real problems. It will take more of this brand of new thinking if we are to build on this record of achievement. During our nation's darkest hours, Americans have strived mightily and succeeded in meeting the challenges of their times. The question before us is whether we will do the same during this bright moment; whether we will seize this moment to bring more prosperity and progress to more Americans than ever before; whether, having finally conquered our financial deficits, we will have the courage to conquer the other deficits - in health care, in education, in the environment - that challenge us today. In this Platform, today's Democratic Party lays out its plans to do just that. This platform was not written in a dark backroom, but in the light of day; in an open, democratic process that was interactive and inclusive. It was developed both with the guidance of the brightest Democratic leaders and with the voices of thousands of ordinary Americans around the country who contributed their thoughts, ideas, beliefs, and dreams to this platform in person, on paper, and over the Internet. This is a 21st century platform for the 21st century's party. A people's platform for the people's party. If one theme runs through this 2000 Democratic platform, it is this: if America is to secure prosperity, progress, peace and security for all, we cannot afford to go back. We must move forward together and we must not leave anyone behind. PROSPERITY Eight years ago, America was facing a big challenge. Under the Bush-Quayle Administration, the American economy was floundering. Slow growth had turned into no growth and into a jobless recovery. Americans in all walks of life were facing a future of less prosperity and more resignation. In 1992, Bill Clinton and Al Gore were elected to turn the American economy around and point upward toward the future. They took office with a new set of ideas about how to get the economy moving again. They knew that the private sector is the engine of economic growth, but they also knew that, in Franklin Roosevelt's phrase, \"the national community\" - acting through government - can make a big difference. Today, the success of these new ideas is clear. After a generation of stagnation for many and decline for some, real wages for all working families have started to rise again. America has the lowest unemployment and fastest economic growth in more than 30 years. The American people have created 22 million new jobs. We have the lowest inflation rate in decades. More Americans own their home than ever before. Looking back on 1992, this much is clear: Americans are better off than we were eight years ago. But ours is a record to build on, not to rest on. That's because eight years later, we face a new challenge: how to keep prosperity alive - and how to deepen it - in a fast-moving, fast-changing economy. We can never take our economic prosperity for granted nor can we afford to go back to either tax-and-spend or cut-and-run - the failed policies of the past. It took innovative, new Democratic policies to create the environment where prosperity could bloom. It will take more such policies to allow prosperity to blossom - to forge a prosperity that does not leave anyone out and does not leave anyone behind. During the past decade, the birth of the global, information-based new economy has changed most every aspect of Americans' lives. As we move inexorably from the Industrial Age to the Information Age, the transition will be difficult for some. In the decade to come, Democrats must lead the way in equipping all Americans with new tools for economic success and security. This is the only sure means of ensuring that America's prosperity is one that is broadly shared. Time after time, Republicans opposed the ideas that brought prosperity to America. Time after time, they have been proven wrong. But their sorry record does not give them pause, it does not even slow them down. Despite a Democratic record of success, the Republicans now propose to rewind to the policies that brought America the days of deficits, doubt, debt, and decline; a retreat to the thinking of the era of recessions, repossessions, and retrenchment. Democrats believe that to further our prosperity and make sure all Americans are ready to reap the rewards of the new economy we need thinking as innovative as the moment in which we live. First, we must continue the fiscal discipline that has been the hallmark of the past eight years - that means paying down the debt and offering the right kind of tax cuts. Second, we must use our unprecedented prosperity to secure Social Security and Medicare for future generations. Third, we must invest in the most precious resource we have - the American people and their skills and ability to innovate. Fourth, we must continue to reinvent government so that it works better and costs less and is in line with the on-line world. Fifth, we must open new markets to American products at home and around the world. Finally, we must reinforce the basic American bargain of requiring and rewarding hard work and we must provide Americans with the opportunity to participate in key decisions at work and in their communities. FISCAL DISCIPLINE For the 12 years before Bill Clinton and Al Gore took office, Republicans talked about fiscal discipline while they quadrupled the national debt. They ran up monstrous yearly deficits and nearly ran the American economy into the ground. In 1992, Democrats promised to cut the deficit in half in four years. They did - and went even further. It took Al Gore's tie-breaking vote in the Senate to overcome unanimous Republican opposition to deficit reduction. Today, America has gone from the biggest deficits in history to the biggest surpluses in history. Fiscal discipline keeps interest rates low and investment rates high - and it has helped fuel America's remarkable prosperity. We must not go back. That's why Democrats now vow to balance the budget every year, barring a national emergency. But even this is not enough. In the 160 years since the very first Democratic Platform, America has always struggled under a national debt. Today's Democrats believe we should pay down the debt every year until we can give our children the independence, self-sufficiency, and prosperity that will come from an America that is debt-free. In 12 years of rule, Republicans quadrupled the national debt. In the next 12 years, Democrats vow to wipe out the publicly-held national debt. Today, because of the success of the Clinton-Gore Administration, a debt-free America is within reach. This would free businesses to invest and innovate, it would provide an ever more sturdy foundation for future economic growth, and it would create good jobs. That's why Al Gore is determined to completely eliminate the publicly-held national debt by the year 2012. The Right Kind of Tax Cuts. The road to long term prosperity starts with embracing fiscal discipline. Unfortunately, the Republicans eschew fiscal discipline and offer up nothing less than fiscal disaster. They would squander the surplus on a more than trillion-dollar federal government tax giveaway for the well-off and well-connected, while failing to eliminate the national debt, neglecting to shore up Social Security and Medicare, and shirking the need to invest in the education of America's children and the skills of her workers. For the past eight years, Democrats have been working to offer tax relief to the Americans who need it the most where they need it the most. We cut taxes for working parents who were struggling to make ends meet. We cut taxes for parents who were working hard and trying to raise good kids. We cut taxes for Americans who had studied hard and made it to college. We cut taxes for Americans who were continuing their educations and gaining new skills to stay on the cutting-edge of the economy. We cut taxes for companies that were helping Americans make the transition from welfare to work. We cut taxes for more than 90% of America's dynamic small businesses. Today, for most families, the federal tax burden is the lowest it has been in twenty years. The Bush tax slash takes a different course. It is bigger than any cut Newt Gingrich ever dreamed of. It would let the richest one percent of Americans afford a new sports car and middle class Americans afford a warm soda. It is so out-of-step with reality that the Republican Congress refused to enact it. It would undermine the American economy and undercut our prosperity. Under the leadership of Al Gore, Democrats want to give middle class families tax cuts they can use - tax cuts that will put their own values into action and that will not injure the economic vitality they rely on. Democrats seek the right kind of tax relief - tax cuts that are specifically targeted to help those who need them the most. These tax cuts would let families live their values by helping them save for college, invest in their job skills and lifelong learning, pay for health insurance, afford child care, eliminate the marriage penalty for working families, care for elderly or disabled loved ones, invest in clean cars and clean homes, and build additional security for their retirement. RETIREMENT SECURITY Americans' golden years should be times of calm and security, not concern and stress. Few achievements testify more to the ability of government to do good than Social Security. It has lifted millions of elderly Americans out of poverty and helped them make ends meet. Social Security is more than a government program. It is a solemn compact between the generations. It is our nation's most important family protection. The choice for Americans on this vital part of our national heritage has never been more clear: Democrats believe in using our prosperity to save Social Security; the Republicans' tax cut would prevent America from ensuring our senior citizens have a secure retirement. We owe it to America's children and their children to make the strength and solvency of Social Security a major national priority. That's why Al Gore is committed to making Social Security safe and secure for more than half a century by using the savings from our current unprecedented prosperity to strengthen the Social Security Trust Fund in preparation for the retirement of the Baby Boom generation. We now have an extraordinary opportunity to maintain Social Security. In addition, we can reform it - not the wrong way, with proposals such as raising the retirement age, but the right way - with fiscal discipline and by making it fairer for widows, widowers, and mothers. Retirement security comes on many fronts. Democrats have successfully passed reforms to simplify the pension process for small businesses, expand pension portability, and protect employee pension funds. Democrats believe that workers' pensions should be protected and more portable. We also believe that changes in every American's pension rights should be fully disclosed. This is becoming increasingly important today, as pensions are progressively being shifted from a workers' benefit plan to a workers' contribution plan. We believe these changes need to be carefully examined by independent agencies to make sure they abide by current federal law. Democrats support President Clinton's veto of the Republican tax scheme that would have diminished anti-discrimination protections for middle-class and lower-income workers. To build on the success of Social Security, Al Gore has proposed the creation of Retirement Savings Plus - voluntary, tax-free, personally-controlled, privately-managed savings accounts with a government match that would help couples build a nest egg of up to $400,000. Separate from Social Security, Retirement Savings Plus accounts would let Americans save and invest on top of the foundation of Social Security's guaranteed benefit. Under this plan, the federal government would match individual contributions with tax credits, with the hardest-pressed working families getting the most assistance. The Republicans have a far different idea - a scheme that would come not in addition to Social Security but at the expense of it. Their Social Security privatization plot would siphon $1 trillion in payroll taxes away from the Social Security trust fund, take 14 years off the life of Social Security, eliminate the fundamental guarantee of retirement security, and raise the specter of massive government bail-outs. And, according to independent analyses, the Republicans' privatization plan would cut the guaranteed benefits for young workers by as much as 54 percent. It would take the \"security\" out of Social Security. Retirement Savings Plus does not threaten Social Security's guaranteed benefit. Social Security may be 65 years old - but it is not ready to be retired. Taken together George W. Bush's $2 trillion tax cut, his campaign-season spending proposals, his support for an unspecified but unprecedented missile defense system, and his support for privatizing Social Security add up to an assault on the surplus - causing Americans to have to choose between drastic cuts in education and health care or a return to the days of deficit spending. This is not a choice Americans should have to make. With fiscal discipline and a commitment to honoring our values, we can both save Social Security and give Americans the ability to create a nest egg without turning back the clock on our prosperity. INVESTING IN AMERICANS Democrats know that today, more than ever before, we need the right kinds of investments - in education, lifelong learning, skill development, and research and development - to take advantage of the vast opportunities of the Information Age. We need to make sure Americans have the skills and tools they need to compete and win in the new knowledge-based, global economy. A Revolution in American Education Democrats understand that ensuring every child the highest quality education is essential if America is to remain strong and competitive in today's economy. That's why Al Gore's very first campaign speech was about education and that's why Al Gore will make education his top domestic priority. Nine out of every ten children in this country attend a public school. Public education already allows the United States to have one of the highest standards of living in the world, providing equality of opportunity for all regardless of socio-economic status. The success stories coming from public schools are greater than at any time in their history: higher graduation rates, increasing test scores, and higher student achievement - with especially substantial gains among our neediest students. We must continue to build on this record of success that Democrats have compiled in the last eight years. We have helped states and communities set high academic standards for students and called for an end to social promotion. We have started hiring 100,000 qualified teachers. We have increased accountability. We have opened the gates of college to millions of Americans. Now we must do more. Democrats understand that America will not long remain first in the world economically unless we become first in the world educationally. We cannot continue to generate a fifth of the world's economic output if a third of our students do not meet basic reading standards. We cannot stay number one in high technology jobs if we remain last in the percentage of degrees awarded in science. In today's knowledge-based economy, it's just that simple. Education leads to the future success and security of our country and citizenry. Americans have been told they must choose between investing in education and demanding accountability. This is the type of false choice that drives our government into stalemate and drives Americans up the wall. Americans believe that we need to invest more in our children's educations - and they're right. Americans also believe that we should not be pouring more money into a system that is producing bad results - and they're right about that too. We should do more of what we're doing right and less of what we're doing wrong. Al Gore and the Democratic Party know that investments without accountability are a waste of money and that accountability without investments are a waste of time. George W. Bush and the Republican Party offer neither real accountability nor reasonable investment. What they do offer are soothing sound-bites and bite-sized solutions. They refuse to invest in America's crumbling schools and crowded classrooms - spending 100 times more on tax cuts than on education. They don't help pay teachers like professionals nor do they insist on higher standards for teachers. They propose blank check block grants without accountability. Their version of accountability relies on private school vouchers that would offer too few dollars to too few children to escape their failing schools. These vouchers would pass the buck on accountability while pulling bucks out of the schools that need them most. When it comes to education, Democrats want to invest more and aim higher, the Republicans invest too little and aim too low. We cannot afford - materially or morally - to let another generation of American children pass through inadequate schools before we make needed changes that will save them from a lifetime of frustration and limited horizons. The time for action is now. By the end of the next presidential term, we should have a fully qualified, well trained teacher in every classroom in every school in every part of this country - and every teacher should pass a rigorous test to get there. By the end of the next presidential term, every failing school in America should be turned around - or shut down and reopened under new public leadership. By the end of the next presidential term, we should ensure that no high school student graduates unless they have mastered the basics of reading and math - so that the diploma they receive really means something. By the end of the next presidential term, parents across the nation ought to be able to choose the best public school for their children. By the end of the next presidential term, every eighth grader in America should be computer literate. By the end of the next presidential term, high-quality, affordable pre-school should be fully available to every family, for every child, in every community in America. By the end of the next presidential term, every child should learn in a safe, modern classroom with the most up-to-date technology. By the end of the next presidential term, the achievement gap between students of color and the rest of America's students should be eliminated. All this we pledge - and more. The time for tinkering around the edges has long passed. We need revolutionary improvements in our public schools. This requires a major national investment; a demand of accountability from all; a genuine expansion of public school choice; and a renewed focus on discipline, character, and safety in our schools. Discipline, Character, and Safety. Education is not just about test scores, but about passing on our values to the next generation of American citizens. Our children and teachers deserve schools of safety and classrooms free of fear. We should have a zero-tolerance policy towards guns in schools. Each school should institute strict, firm, and fair discipline policies that are agreed upon on the first day of the school year at a meeting of teachers, parents, and students. We should expand the Family Leave Law to make sure parents can attend these meetings and all parent-teacher conferences without being scared they will lose their jobs. We must do all we can to encourage active parental involvement in our schools - after all, parents are a child's first and best teachers. A parent's job does not end when they drop their child off at the schools front door. They have a responsibility to actively participate in their childrens' education, to read to their children, and to help their children with their homework. Schools need to do their part by welcoming parents into the education process and giving them a voice in the education of their children. Democrats believe in \"second-chance schools\" where kids expelled from school and those headed for trouble can get the concentrated help, services, and guidance they need to get back on the path to success. If we are serious about fighting school violence, we need a dramatic increase in after-school care for America's children. The average two-parent family works 500 more hours a year than they did a generation ago. Children often come home from school to empty houses. We know that the most dangerous hours for children are those between the end of the school day and the end of the work day. It is in these afternoon hours that children are most likely to get into trouble and fall under bad influences. Democrats have increased after-school assistance 500 times over in the last four years. Al Gore believes in expanding after-school programs and providing Americans with an after-school tax credit so that children have a safe, supervised after-school environment where they can continue to learn and learn right from wrong. Too often, our culture offers our children a virtual crash course in violence and degradation. It is sometimes a culture of too much meanness and not enough meaning. That's why character education is so important in our schools. Education should not be a morals-free zone. Schools can teach our kids about honesty, hard work, openness to new information, strong discipline, willingness to reason, personal responsibility, and tolerance for different points of view. Teachers can help children develop the values and the character - as well as the intellectual tools - it takes to succeed and contribute to their communities. The traditional three R's are not enough. Schools need to make sure they teach kids respect, reliability, and responsibility as well. We must also remember that our schools are not just training the next generation of workers, they are also educating the next generation of citizens. That's why Democrats support democracy education, civic education, and service requirements in our schools. Strict Accountability for Results, Strong Incentives for Success. Democrats believe that everyone involved in the education system should be held accountable. Accountability means we will no longer tolerate mediocrity and no longer allow failure. Accountability applies to states, school districts, schools, teachers, students, and parents. Everyone must do their part. Nobody can shirk their responsibility. Consistently bad schools should be shut down. No excuses. No exceptions. Every state and school district should identify failing schools and turn them around with all necessary measures and all necessary resources. Students in those schools should get first priority in transferring to a better-performing public school in the district and getting intensive after-school academic help to make sure they are not left behind while their school is being turned around. Failing schools that do not improve should be quickly shut down and reopened with a new principal and new teachers. States should be held accountable for reducing drop-out rates, increasing graduation rates, and raising student achievement. Working together with teachers, school principals should be able to hire on the basis of qualifications and fit, not just on the basis of seniority. Teachers should be answerable for what goes on in their classroom. New teachers who answer the call to join this honorable profession should get the mentors and professional support they need to make the transition into teaching - and then should have to pass a rigorous and fair test before they step foot into a classroom. Teaching is no easy job and we should not expect that everyone is able to make it in the classroom. New teachers should receive ongoing support and mentoring from their more experienced colleagues. Current teachers should receive continuing quality professional development to ensure that their skills and knowledge reflect the most up-to-date information and research. Those teachers who do not meet the highest quality standards should not be allowed to sully the reputation of the teaching profession. That's why teachers who are not teaching well should receive help in getting up to standards. At its best, teaching is the job of a lifetime. But teaching contracts and licenses should not be an automatic lifetime job guarantee. That's why we need regular evaluations to determine whether a teacher's license should be renewed. Democrats urge faster but fair ways, with due process, to identify, help - and when necessary - speedily remove low performing teachers. Every student must be given the opportunity to learn. But students have to take responsibility and be accountable for their own educations, as well. We need measurements to make sure students are getting the preparation they require - including voluntary national tests in 4th grade reading and 8th grade math. Democrats insist that no student should graduate with a diploma they cannot read. The federal government needs to be held accountable, too. In states that do not make progress in improving student performance, the federal government should redirect money from state bureaucrats and transfer it directly to schools that need it. States that do succeed in raising student success should receive bonuses - and schools that are making a positive difference should receive bonuses, as well. In addition, teachers who earn a National Board Certification should be especially rewarded. Investing in Our Schools. We cannot expect our children to learn all that they need to know in classrooms that are overcrowded, with teachers that are overburdened, and with textbooks and technology that are out-of-date. We need to invest in our schools and our childrens' futures. High-quality pre-school should no longer be a luxury. Research - and the experience of path breaking states such as North Carolina and Georgia - shows that giving kids a smart start can lead to higher reading and achievement levels, higher graduation rates, and greater success in the workplace. We need an aggressive national campaign to put one million new well-trained teachers in our classrooms. We must start reducing class size by finishing the job of hiring 100,000 new qualified teachers. In addition, Al Gore has proposed the creation of a new 21st Century Teacher Corps - open to talented people around the country who agree to teach in a school that needs their help. In return, they would get help paying their college tuition, assistance in paying off their student loans, or a hiring bonus for those willing to switch careers. And we need alternative certification so that those who choose to switch into teaching don't have to start their education all over again. Far too many teachers are overstressed and overworked, underpaid and underappreciated. We need to treat teachers like professionals - pay them like professionals and hold them to professional standards. All qualified teachers should get a raise and master teachers should get the biggest raise. We need to provide professional development, training, and support so that all teachers can succeed. We should rebuild and modernize our school buildings to assure students can attend schools that are modern, safe, and well-equipped for learning. And we need to construct more new schools to meet the needs of the largest generation of students in American history. We cannot convince our children to value education when they are packed into crammed classrooms like sardines in a can and when their facilities are falling down. Al Gore and the Democrats believe we need smaller classes, smaller schools, and \"schools within schools\" so that impressionable children do not get lost in the shuffle. We must ensure that children with disabilities are not blocked from having access to free, appropriate education and that the doors to our public schools are not closed to children with special needs. We must, finally, live up to the Federal government's promise to communities to help them defray the expenses of educating children with special needs. We must assure that schools have the resources to meet the challenges of an increasingly diverse student population with programs for English language learners, including bilingual education, to close the achievement gap. We oppose language-based discrimination in all its forms, including in the provision of education services, and encourage so-called English-plus initiatives because multilingualism is increasingly valuable in the global economy. We should create new Opportunity Academies around the nation between high school and college where disadvantaged students can get the intensive academic preparation in math, reading, writing, and study skills that will improve their likelihood for success in college and beyond. Supporting Schools of Innovation. In order to create a world-class educational system for all our students, we must allow experimentation in our public schools to find out what works. The Democratic Party supports expansion of charter schools, magnet schools, site-based schools, year-round schools, and other nontraditional public school options. Charter schools and other nontraditional public school options can free school leaders, teachers, parents, and community leaders to use their creativity and innovation to help all students meet the highest academic standards. The Democratic Party will triple the number of charter schools in the nation. And, we will ensure that these charter schools are fully accountable - financially and academically - to students and the communities they serve, and that they are indeed making progress in maximizing student achievement. All public schools should have the freedom to design their curriculum within high standards and all public schools should compete for students - and we should start by bringing universal public school choice and competition to our lowest-performing public schools. Let there be no mistake: what America needs are public schools that compete with one another and are held accountable for results, not private school vouchers that drain resources from public schools and hand over the public's hard-earned tax dollars to private schools with no accountability. Closing the Opportunity Gap Forty years ago, the Democratic platform discussed a Missile Gap as a measurement of America's competitiveness around the world and our security here at home. Today, too many Americans face an Opportunity Gap - a lack of the skills they need to be competitive in the global economy and have career security in the workplace. The Opportunity Gap is also a chasm created by income disparity, discrimination by race and gender, and the abandonment of our inner cities. Many of today's workers will need retraining over the next decade. Nearly ninety percent of companies say they already face a shortage of skilled workers. The Opportunity Gap is costing American workers good jobs at good wages - and it must be closed. Al Gore has proposed a broad set of initiatives to provide college education, lifelong learning, and ongoing skill development for all Americans. College Education and Lifelong Learning for All. With Democratic leadership over the past eight years, the percentage of young people who are entering college has gone up by nearly 20 percent. In the Information Age, it is clear that a college education is more important than ever. The HOPE Scholarship and Lifetime Learning Tax Credit have opened the gates of college wider than ever before. Pell grants are at their highest level ever. Now we need to do more. We should make a college education as universal as high school is today. Al Gore has proposed a new National Tuition Savings program to tie together state tuition savings programs in more than 30 states so that parents can save for college tax-free and inflation-free. We propose a tax cut for tuition and fees for post-high school education and training that allows families to choose either a $10,000 a year tax deduction or a $2,800 tax credit. In today's economy, education should not be a time in a person's life but a way of life. To keep up with the fast-moving, fast-changing economy, workers must have the ability to continue learning and upgrading their skills for a lifetime. The next great frontier in American education is dramatically expanding opportunities for lifelong learning, skill development, and training. Democrats believe that every hard-working American should have the chance to use their best talents. That is why we support a major new commitment to expanding worker training and skill development, including the creation of national skills standards. Al Gore has called on companies and workers to build more partnerships for skill development. He has proposed incentives for states and employers to expand worker training. We should fund partnerships of employers, colleges, unions, and others that will connect workers to the training they need. We should create a new tax credit for employers who train their workers in the skills needed in the New Economy. We must also give new training allowances that will extend unemployment insurance for those who need time to finish their training courses. Al Gore has called for new 401(j) accounts - like the 401(k)'s which so many Americans use - that would let employers help their employees save tax free and use those savings for the lifelong learning for the employee or their spouse, or their children's college education. Al Gore has also called for a permanent tax exemption to encourage employers to provide tuition assistance benefits to their workers, and for expanding this exemption so that entire families can benefit from these tuition benefits as well. Bridging the Digital Divide. Democrats believe that every American - regardless of income, geography, race, or disability - should be able to reach across a computer keyboard, and reach the vast new worlds of knowledge, commerce, and communication that are available at the touch of a fingertip. That is why Democrats fought for the e-rate to wire every classroom and library to the Internet. In the next four years, we must finish connecting the job and then go further. We must launch a new crusade - calling on the resources of government, employers, the high-tech industry, community organizations, and unions - to move toward full Internet access in every home, for every family, all across the United States. We must make sure that no family or community is left out. We must not rest until Internet access is universal. We must also launch a new national effort to provide basic skills in the newest technology. Al Gore has proposed a major initiative to set and achieve a national goal of computer literacy for every child by the time they finish the eighth grade. He has also called for expanded technology training for workers, and supports incentives for employers to provide home computers and Internet access to their workers. And we must do more than merely teaching technology in the classroom and the workplace. We must dramatically expand teacher training in how to use the power of the Internet. We should also use our AmeriCorps national service corps members to teach and promote the Internet in the schools, libraries, and technology centers that need them the most. America was the pioneer of universal education; now America must become the pioneer of universal computer literacy. Investing in Innovation Technology is no longer just wondrous gadgets, it is an ever more integral part of our economy - and an enormous part of what has been driving economic growth. We need to harness technology's power and make sure America stays on the cutting-edge. That means continuing to invest in experimentation, exploration, and innovation. Democrats recognize that a sustained public investment in long term basic research has been the foundation for America's scientific and technological leadership. That's why both public and private investment in research and development is crucial to sustaining our prosperity. On the public side, Democrats believe in doubling the current levels of investment in information technology research and biomedical research and supporting the continued development of the Next Generation Internet - moving 1,000 times faster than today's Internet. We believe in helping universities and federal laboratories become centers of innovation that support and catalyze private sector growth. We also believe in the use of creative public-private partnerships that will, when appropriate, help bring new products to market faster. We continue to support technology transfer - forming partnerships between industry and government that can help ensure that American companies and workers develop the technological tools needed to compete in tomorrow's global markets. In the private sector, Democrats believe in supporting the startups, the small businesses, and the entrepreneurs that are making the New Economy go. This means making permanent the Research and Experimentation tax credit and expanding it to make it partially refundable so that small businesses can use it more easily. It also means keeping cyberspace a duty-free zone so that American companies can sell goods around the world and insist that other countries refrain from actions that impede commerce. To expand technology's worldwide potential as a force for good, Al Gore has advanced a bold vision for a new Global Information Infrastructure - a network of networks that sends messages and images at the speed of light, across every continent - to expand access to phone service and communications, further improve the delivery of education and health care, and create new jobs and industries. Strengthening small business is a vital component of economic innovation, job creation, and supporting entrepreneurship. Small businesses have accounted for more than 90 percent of the 22 million new jobs created with Democratic leadership. The Democratic Party is committed to sustaining and adding to that level of growth of small businesses, including home based businesses. Democrats believe that strengthening small businesses is a vital component of strategies to create opportunity and community economic development. We will build on the tremendous progress of the Clinton-Gore Administration in modernizing the Small Business Administration and improving access to the Federal marketplace. We will fight to reform and strengthen programs to combat discrimination against women and minority entrepreneurs, including federal procurement, because the playing field is still not level. Americans generate more new technologies, new inventions, and more creative works of software and entertainment than the citizens of any other country in the world. American creativity contributes greatly to improving the quality of daily life, helps us work more efficiently, and enriches our national culture. America's laws and policies must be tailored and equipped to nurture and advance this unique aspect of our national character. This means we must ensure that sound patent and copyright laws motivate our inventors and creators to pursue their vision. Internationally, we must work to build support for strong intellectual property laws among the community of nations, including in trade agreements. We must take all steps necessary to secure effective enforcement of those laws - at home and abroad - to ensure that others do not steal intellectual property through piracy and other forms of theft. Democrats know that technological innovation is critical to maintaining a strong manufacturing sector as we enter the Information Age. Manufacturing is a principal engine of productivity growth, a provider of jobs that pay family-supportive wages, and a significant source of exports for paying our way in the world economy. Al Gore and the Democratic Party will fight to keep America's basic industries the most competitive in the world. Protecting American Consumers As our science and technology advance we must work hard to protect our oldest and most cherished values. That's why Al Gore, while supporting the completion of the Human Genome Project, has championed legislation to ban genetic discrimination. While fighting to expand Internet access, he has led the Administration's efforts to give parents, schools, and communities effective tools to protect children from inappropriate content on-line. In particular, Al Gore has focused on the challenge of protecting Americans' personal privacy on-line as well as the medical and financial information that can all too easily be intercepted and abused by others. Al Gore has called for an Electronic Bill of Rights for this electronic age - including the right to choose whether personal information is disclosed; the right to know how, when, and how much of that information is being used; the right to see it yourself; and the right to know if is accurate. We must protect not only our privacy, but the food we eat, the air we breathe, and the water we drink. That's why Democrats believe we ought to have a modern, science-based food safety system, including meaningful food labeling that also discloses where our food comes from, and that communities should have the right to know about toxins that are released into the air and water. INVESTING IN COMMUNITIES Democrats believe that in building upon the record-breaking prosperity and growth achieved in the past eight years, we must not leave any community behind. Under the leadership of Al Gore, the Empowerment Zones and Enterprise Communities programs have brought new hope to cities and rural areas all across America. Now we need a new round of Empowerment Zones to spread prosperity even further. The Clinton-Gore New Markets Initiative is shining a spotlight on the untapped potential for commerce, tourism, and investment in many communities, and Al Gore will extend these efforts to see that the prosperity of the mainstream economy flows to the Main Streets everywhere. The Clinton-Gore Administration fought to strengthen the Community Reinvestment Act and to create a network of Community Development Banks, and Al Gore will continue that fight. Democrats are committed to building an America in which no neighborhood or town see joblessness and shuttered businesses commonplace or inevitable, and where no families or young adults surrender their God-given right to work hard and live the American dream. Part of that dream is home ownership. Under Democratic leadership, we have achieved an all-time high in home ownership, including among groups that have historically been left out. We are committed to continuing this progress, because home ownership is a foundation for building wealth and economic security for families, and it provides a vital anchor enabling neighborhoods to thrive. In too many communities, however, owning or renting an affordable home seems an impossible dream. Al Gore and Democrats have long defended the mortgage interest deduction and the Low Income Housing Tax Credit, and believe we must reinvigorate our communities and support our families through partnerships and targeted investments and eliminating community redlining by lenders that will better harness the power of markets to create the housing we need. We must pay down the debt to keep interest rates low. We need to create a continuum of care for homeless people so that they get help in getting themselves off the streets and back on their feet. We must ensure that housing costs in thriving communities do not outpace the income of middle class families. We must expand the supply of life cycle housing. We must encourage the renovation and construction of affordable housing closer to places of work and to mass transit so workers can get to their jobs without being tied up in traffic for hours. In rural America, we have the opportunity to create a rural renewal on our nation's farms with improved transportation and infrastructure, better access to capital and technology, reduced concentration in agribusiness, and an expansion of new markets for our crops, and strengthening our ability to compete in world markets. The Internet can break down barriers of geography and isolation and bring the rural economy into the new economy. Farmers should receive incentives to conserve soil and improving farming and forestry techniques. The Republican Freedom to Farm Act has resulted in years of low prices and necessitated billion dollar bailouts. It is misguided and must be changed. Family farmers who work hard and smart should be able not only to survive but to thrive. Democrats will strengthen, not shred, the safety net for family farmers; we will open markets abroad for them. And we will not turn our backs on rural communities; we will work to ensure that they share in the new prosperity we are building for all of America. Livable Communities. Across America a new movement is emerging as citizens work together to build more livable communities. These are communities where the streets are safe and schools are good, where high wage jobs are not hours away from home, where people can get to work and run their errands without spending hours stuck in traffic, where they can breathe clean air and drink clean water, where the spirit of community reigns. Democrats believe communities know best and that they should have the resources and tools they need to act on their decisions, to have the ability to create communities of which families can be proud. We want to transform out-of-control sprawl to well-planned smart growth. That is why we support the \"Better America Bonds\" - tax credits for state and local bonds to build more livable communities. We must help communities reconnect to the land around them, preserve open spaces, build parks, improve water quality, and redevelop rusty old brown fields. We need to help save farms from being turned into strip malls and parks from being paved over. We should acquire new lands for urban and suburban forests and recreation sites and set aside wetlands, coastal and wildlife preserves. And it is time we enhanced our quality of life by unclogging our nation's roads and airports. Al Gore and the Democratic Party support the building of high-speed rail systems in major transportation corridors across the nation. High-speed rail reduces highway and airport congestion, improves air quality, stimulates the economy, and broadens the scope of personal choice for traveling between our communities. We support new grants to Amtrak and the states for improving existing and for expanding and completing passenger rail routes and corridors. OPENING MARKETS AROUND THE WORLD Exports sustain about 1 in 5 American factory jobs - jobs that pay more than jobs not tied to the global economy. Open markets spur innovation, speed the growth of new industries, and make our businesses more competitive. We must work to knock down barriers to fair trade so other nation's markets are as open as our own. Trade has been an important part of our economic expansion - about a third of our economic growth in recent years has come from selling American goods and services overseas. There is no doubt that with trade - and with investments in giving American workers the skills they need - we can out-compete workers anywhere in the world. It's clear we live in a globalized world - and that there is no turning back. But globalization is neither good nor evil. It is a fact - and we have to deal with it. Democrats believe we must be leaders in the new global economy, not followers. We believe that globalization will work for all Americans only if there are rules of the road, as in the domestic economy, that promote both a strong economy and our basic American values. We need to make the global economy work for all. That means making sure that all trade agreements contain provisions that will protect the environment and labor standards, as well as open markets in other countries. Al Gore will insist on and use the authority to enforce worker rights, human rights, and environmental protections in those agreements. We should use trade to lift up standards around the world not drag down standards here at home. True open trade is not just about profits, but about people; not a race to the bottom, but a dash to the top; about a rising tide lifting the boats of workers here and abroad; about reinforcing the values of freedom and liberty and the rule of law in the hearts and minds of people everywhere. The test of open trade in the years ahead is whether it empowers the many and not just the few, whether its blessings are widely shared, whether it helps to lift the poor out of poverty; and whether it works for working people. Democrats know that to build a new consensus for more open trade, we must give workers the tools they need to compete in the global economy and support rules that will protect workers' rights, human rights, and environmental protections. That's why our lifelong learning and skill development proposals are so important. American workers need access to ongoing skills development so that they have the tools they need to succeed in the New Economy. In addition, our trade adjustment assistance programs should be improved so that all affected workers receive timely and adequate assistance, including measures to address health care coverage and pension protections. With the leadership of Al Gore, Democrats helped America's steel industry weather the effects of the Asian financial crisis. As President, Al Gore will move aggressively to reduce our overall trade deficit and stop the erosion of good paying manufacturing jobs. This includes negotiating tough agreements to reduce our persistent automotive trade imbalances with our major trading partners. We must continue to monitor imports and, consistent with the World Trade Organization, ensure that the United States utilizes all of its trade laws and other mechanisms, including product specific safeguards, to stop quickly and effectively any import surges when they threaten our workers and communities. The President should be able to negotiate trade agreements with the nations of the world and should include worker rights, human rights, and environmental protections in those agreements, as well as market opening initiatives. At the same time, Al Gore will challenge American companies to ensure labor protections and worker safety at their overseas operations. And U.S. representatives at the International Monetary Fund and the World Bank should also seek to advance fair treatment for workers internationally. We should create an environment in which electronic commerce can flourish globally as it has here in America. We are committed to supporting the rights of workers around the world. And we should vigorously monitor trade agreements to make sure other nations are not shirking their responsibilities. Democrats are committed to addressing the problem of manipulative corporate tax shelters, including in the international context, that undermine the public's faith in the fairness of our voluntary tax system. At the same time, we must ensure no tax provision has the effect of encouraging corporations to locate in other countries at the expense of American workers. BUILDING A 21ST CENTURY GOVERNMENT Since he took office, Al Gore has led the way in reinventing government - making government more effective in its mission of service to the public. Under his leadership the federal workforce has been cut by 377,000, making it the smallest government since Dwight D. Eisenhower was president. This has been accomplished through cooperation and partnership. Sixteen thousand pages of regulations were scrapped. From tea testers to mohair subsidies to the Navy's own dairy farm, over 200 outdated and unnecessary government programs have been eliminated. As a percentage of the workforce, the federal government is the smallest it has been since the New Deal. We have saved over $135 billion - contributing to the surplus and our prosperity. But we have saved something much more precious as well. We have begun to earn back the faith and trust of the American people in their democratic institutions. Trust in government has almost doubled. The first customer survey ever taken of American's satisfaction with the services government delivers found that fully 60 percent felt service had improved in the last two years and rated government services at levels almost as high as services in the private sector. Today, our government is focused on emphasizing results over red tape, offering Americans quality service, old-fashioned common sense, and working in partnership with the private sector to achieve common goals. Republicans attack public workers and tear down public services. We have empowered government workers and improved public services. Now we need to go much further. We have ended the era of big government; its time to end the era of old government. We need to create a government where Americans can easily find the services they need; one that is on-line all the time with no need to wait in line, an open government that's always open. On the Internet, citizens will be able to help cut crime in their neighborhood, notify government of potentially dangerous environmental hazards, or sign up for a clinical trial of the latest advances in medicine. And all of this will be done while protecting everyone's personal privacy and with the highest levels of universal access and security. This new e-government will break down barriers to service, reduce costs, and make government accessible for all. We must forge partnerships between labor and management that recognize the interests of both sides while uniting both front-line government workers and managers in a common crusade to improve government performance. We must ensure that government has the tools and expertise necessary to provide high-quality services. Democrats do not believe that privatization is a panacea. Some services are inherently public. Democrats also believe that, to ensure government works better and costs less, public employees must be allowed to compete both for their current work and for new work. When government work is contracted out to private companies, they should adhere to same level of accountability as public agencies and those arrangements must incorporate labor, safety, health, civil rights, and other important safeguards. We must also continue to decentralize our government, to make it more flexible and responsive towards communities and individuals, and to turn its focus towards empowering Americans to take charge of their own lives. Faith-based and community-based organizations have always been at the forefront in combating the hardships facing families and communities. Democrats believe it is time that government found ways to harness the power of faith-based organizations in tackling social ills such as drug addiction, juvenile violence, and homelessness. However, in contrast to the Republicans, Democrats believe that partnerships with faith-based organizations should augment - not replace - government programs, should respect First Amendment protections, and should never use taxpayer funds to proselytize or to support discrimination. VALUING WORK Democrats believe in hard work and we believe that work must pay. It is what has made America great. There is a basic bargain at the heart of the American story - hard work should be both required and rewarded. Democrats also believe that those who do work hard should not be stuck in place - they should get ahead. And those who work hard should have a voice in their workplace. Supporting Working Families. Democrats know that workers' freedom to choose a voice at work is a fundamental American right that must never be threatened, never be obstructed, never be taken away. From the Industrial Age to the Information Age, unions have given working people the chance to improve their living standards and have a voice on the job. The Clinton-Gore Administration stopped the Team Act, defeated a national right-to-work law, and fought for the resources to enforce worker protections. Al Gore will protect our wage and hour laws, including the forty-hour workweek and overtime requirements, and stand firm in support of the Davis-Bacon act and the Service Contract act. He has also proposed reforming government contracting rules to ensure that taxpayer dollars do not go to companies that break basic labor laws. Democrats have always believed in making work pay. We are fighting for a new ergonomic standard and whistle-blower protections. We have stood up for the National Labor Relations Board and fought to protect the right of working families to participate in the political process when it was under attack. Now we must go further - not just playing defense against misguided Republican attempts to set back the cause of worker's rights, but moving the ball forward. We need a new national law banning permanent striker replacement workers - so that workers' right to organize into a union and bargain with their employers are never compromised. While we have made the workplace the safest ever, we need to further increase workplace safety. We should stiffen penalties for employer interference with the right to organize and violations of other worker rights. We must also reform labor laws to protect workers' rights to exercise their voices and organize into unions by providing for a more level playing field between management and labor during organizing drives, and facilitating the ability of workers to organize and to bargain collectively. Rewarding Work for All. Democrats believe in an economy that works for everyone and gives everyone a chance to work. We have made a good start by fighting for the Earned Income Tax Credit which has helped millions of American families work their way out of poverty. We won the battle for increasing the minimum wage. Now we must do more. We must bring all Americans who are willing to work hard into the circle of prosperity by more fully extend the benefit of the Earned Income Tax Credit to working families, again raising the minimum wage, and giving American workers the skills they need to make it in today's economy. We will vigorously enforce protections against on-the-job discrimination, reassert our belief in an equal day's pay for an equal day's work, seek to prevent the exploitation of workers, and ensure that the nation's worker protection laws are enforced. Democrats believe that one way we value and reward hard work is to modernize, strengthen, and sustain the nation's unemployment compensation system - a bedrock protection against poverty for millions of workers and their families. Today, the system serves far fewer working families than in the past and many especially vulnerable workers - such as low wage workers, seasonal employees, contingent workers, and women - are especially likely to fall outside the system's protective safety net. Democrats believe we must fight to update and upgrade the nation's unemployment system, to stabilize its funding, extend eligibility to more workers, and improve benefits. We know that even as the economy changes and expands, millions of workers will continue to labor in jobs that pay low wages and may not require significant education or skills. Many of these workers are women, people of color, or recent immigrants. These workers provide invaluable services to American society and their work has great dignity. Democrats are committed to ensuring that these workers - no less than their counterparts in more highly-skilled, better paid positions - are treated with dignity, respect, and fairness on the job. Democrats also believe that workers in temporary, part-time, and contract jobs should be treated fairly and earn the wages and benefits they deserve because of the jobs they do. Requiring Work from All. With Bill Clinton and Al Gore in the White House, we changed the nation's welfare system - transforming the program into one that encourages and promotes work. Since 1993, the welfare rolls have fallen to their lowest levels in over 30 years. Today, millions of parents now have the dignity of a paycheck, rather than the stigma of a welfare check. The next step is to help these new workers move into the economic mainstream so that they can support their families. It is part of our vision of abolishing poverty. Al Gore is committed to helping new workers and those still on the rolls get help with childcare, transportation and other supports to ensure that anyone who can work, does work. Democrats also believe that we must continue the fight to restore fairness to legal immigrants - these Americans also deserve access to the American dream. Our fundamental mission is to expand prosperity, not government. But the choices government makes can help or hurt prosperity. For the past eight years, Americans have counted on Democrats to make the right choices. The resulting prosperity is clear. Now, in another moment of big choices, Democrats stand ready to lead again - with a record of results and a vision for the future. PROGRESS Eight years ago, many citizens had come to accept the idea that America's best days were behind her: that crime, welfare, teen births, divisiveness and irresponsibility would continue to rise; that our air and water would keep getting dirtier; and that our essential social safety net programs were fated to go broke. Instead, with the leadership of today's Democratic Party, the past decade has seen not just a rebirth of American prosperity, but a new season of progress in meeting our challenges and living up to our obligations. Crime is down to its lowest levels in a generation - the longest decline on record, teen births are down seven years in a row, adoptions are up by 30 percent, millions of Americans have moved off the welfare rolls and onto the payrolls. America is not just better off, it is better. But Democrats know that it must be better still. So we want to use this moment to bring even more progress to America. To make America safer, healthier, more secure. To clean up our environment and our politics. To make the job of parents easier and to bring us together as one America. FIGHTING CRIME Democrats believe government's most basic duty is to establish law, order, and freedom and keep citizens safe from crime. When crime is rampant, families are forced off the streets and behind closed doors. When children are ducking for cover, they have a hard time reaching for their dreams. When people are afraid to walk in their own neighborhood, communities are robbed of the basic sense of decency and togetherness. When an overburdened justice system lets thugs off easy, good parents have a harder time teaching their children right from wrong. Bill Clinton and Al Gore took office determined to turn the tide in the battle against crime, drugs, and disorder in our communities. They put in place a tougher more comprehensive strategy than anything tried before, a strategy to fight crime on every single front: more police on the streets to thicken the thin blue line between order and disorder, tougher punishments - including the death penalty - for those that dare to terrorize the innocent, and smarter prevention to stop crime before it even starts. They stood up to the gun lobby, to pass the Brady Bill and ban deadly assault weapons - and stopped nearly half a million felons, fugitives, and stalkers from buying guns. They fought for and won the biggest anti-drug budgets in history, every single year. They funded new prison cells, and expanded the death penalty for cop killers and terrorists. Here are the results of that strategy: serious crime is down seven years in a row, to its lowest level in a quarter-century. Violent crime is down by 24 percent. The murder rate is down to levels unseen since the mid-1960's. The number of juveniles committing homicides with guns is down by nearly 60 percent. But we have just begun to fight the forces of lawlessness and violence. We cannot go back to the finger-pointing and failed strategies that led to that steep rise in crime in the Bush-Quayle years. We can't surrender to the right-wing Republicans who threatened funding for new police, who tried to gut crime prevention, and who would invite the NRA into the Oval Office. Nor will we go back to the old approach which was tough on the causes of crime, but not tough enough on crime itself. With Al Gore as President, America won't go back. We will move forward. We will fight to increase the number of community police on our streets. We will fight to give police the high-tech tools and the training they need to keep our streets safe and our families secure. We will toughen the laws against serious and violent crime to restore the sense of order that says to children as well as to criminals: don't even think about committing a crime here. We will reform a justice system that spills half a million prisoners back onto our streets each year - many of them addicted to drugs, unrehabilitated, and just waiting to commit another crime. We will make schools safe havens for students to learn and teachers to teach. We believe that in death penalty cases, DNA testing should be used in all appropriate circumstances, and defendants should have effective assistance of counsel. In all death row cases, we encourage thorough post-conviction reviews. We will put the rights of victims and families first again. And we will push for more crime prevention, to stop the next generation of crime before it's too late. Victims' Rights. We need a criminal justice system that both upholds our Constitution and reflects our values. Too often, we bend over backward to protect the right of criminals, but pay no attention to those who are hurt the most. Al Gore believes in a Victims' Rights Amendment to the United States Constitution - one that is consistent with fundamental Constitutional protections. Victims must have a voice in trial and other proceedings, their safety must be a factor in the sentencing and release of their attackers, they must be notified when an offender is released back into their community, they must have a right to compensation from their attacker. Our justice system should place victims and their families in their rightful place. Ending the Revolving Door. We have to test prisoners for drugs while they are in jail, treat them for addictions, and break up the drug rings inside our prison system. Drug and alcohol abuse are implicated in the crimes of 80 percent of the criminals behind bars. Al Gore believes we should make prisoners a simple deal: get clean to get out, stay clean to stay out. And this deal should be non-negotiable. We should do even more to make sure that when criminals leave jail, they leave a life of crime behind. We should impose strict supervision of those who have just been released on parole - and insist that they obey the law and stay off drugs. In return, we should help them make it in the workplace. Al Gore believes that ending the revolving door, in combination with more determined efforts at prevention, will both combat crime and ultimately reduce rates of incarceration that are so tragically high in many communities. Fighting the Scourge of Drugs and Gangs. We should send a strong message to every American child: drugs are wrong, and drugs can kill you. We need to dry up drug demand, hold up drugs at the border, and break up the drug rings that are spreading poison on our streets. We should open more drug courts, to speed justice for drug-related crimes; double the number of drug hot-spots where we aggressively target our enforcement efforts; expand drug treatment for at-risk youth; and make sure that all of our school zones are drug-free zones - by stiffening the penalties to those who would use children to peddle drugs, and those who would sell drugs anywhere near our schools. We know that to dry up drug demand, we must provide drug treatment upon demand. To empower communities protect themselves from organized criminal conduct, the Democrats support giving communities relief against gang related crimes. We should be tough on drugs no matter which form they take and should not discriminate in sentencing. Strong and Sensible Gun Laws. A shocking level of gun violence on our streets and in our schools has shown America the need to keep guns away from those who shouldn't have them - in ways that respect the rights of hunters, sportsmen, and legitimate gun owners. The Columbine tragedy struck America's heart, but in its wake Republicans have done nothing to keep guns away from those who should not have them. Democrats believe that we should fight gun crime on all fronts - with stronger laws and stronger enforcement. That's why Democrats fought and passed the Brady Law and the Assault Weapons Ban. We increased federal, state, and local gun crime prosecution by 22 percent since 1992. Now gun crime is down by 35 percent. Now we must do even more. We need mandatory child safety locks, to protect our children. We should require a photo license I.D., a full background check, and a gun safety test to buy a new handgun in America. We support more federal gun prosecutors, ATF agents and inspectors, and giving states and communities another 10,000 prosecutors to fight gun crime. Ending Racial Profiling. Good policing demands mutual trust and respect between the community and the police. We shouldn't let the acts of a few rogue officers undermine that trust or the reputation of the outstanding work of the vast majority of our dedicated men and women in blue. That is why we need to end the unjust practice of racial profiling in America - because it's not only unfair, it is inconsistent with America's community policing success, it is a violation of the basic American principle of innocent until proven guilty, it views Americans as members of groups instead of as individuals, and it is just plain shoddy policing. We believe that all law enforcement agencies in America should adopt a zero-tolerance policy toward racial profiling. Hate Crimes. The very purpose of hate crimes is to dehumanize and stigmatize - not only to wound the victim, but also to distort the American conscience. Every crime is a danger to Americans' lives and liberty. Hate crimes are more than assaults on people, they are assaults on the very idea of America. They should be punished with extra force. Protections should include hate violence based on gender, disability or sexual orientation. And the Republican Congress should stop standing in the way of this pro-civil rights, anti-crime legislation. Protecting Our Most Vulnerable Citizens. Our most vulnerable deserve special protections. We need tougher penalties against all sex offenders. We should raise the penalties for those who commit crimes against the elderly. We should give federal prosecutors new tools to fight fraud and abuse. We should move aggressively to shut down fraudulent telemarketers who target the elderly. We believe that we must overcome constitutional objections and reenact a strong new law to combat violence against women. And if you commit any violent crime in front of a child, you should pay an even higher price for it: more time in jail. Ending Domestic Violence. Violence in the home is an often silent terror in the lives of millions. We have to make sure that all battered women have the legal protection and the support they need to be safe in their own communities, and to keep their attackers away. By stopping domestic violence, we can also break the generational cycle of violence. We know that when children grow up in abusive families, they are more likely to become abusers themselves. Stopping Crime Before it Starts. Democrats also know that all Americans are better off if we stop crime before it claims new victims, rather than focusing single-mindedly on pursuing perpetrators after the harm is done. That is why we are firmly committed to sound and proven crime-prevention strategies that are good for all Americans. Solid investments in children and youth, in job creation, and in skills development are powerful antidotes to crime. Judges and the Supreme Court. We will fight to fill the vacancies on the federal bench to make sure we have enough judges to promptly decide all cases and to end Republican delays in the Senate that have kept qualified nominees, especially women and minorities, waiting literally for years for a Senate vote. Democrats oppose efforts to strip the federal courts of jurisdiction to decide critical issues affecting workers, immigrants, veterans and others of access to justice. And, unlike Republicans, Al Gore will appoint justices to the Supreme Court who have a demonstrated concern for and commitment to the individual rights protected by our Constitution, including the right to privacy. VALUING OUR FAMILIES Government does not raise children, families do. But government can help make the hardest job in the world - being a parent - a little easier. Today, families come in all different shapes and sizes, but they all face similar challenges. Government should be on the side of parents - making it easier for them to raise their children and pass down their values. With Democrats in the White House, we have passed the Family and Medical Leave law, which has been used by 20 million Americans to care for a newborn baby or a sick loved one. Al Gore led efforts to create the voluntary TV ratings system, to put the V-chip in all new TV sets sold in America so that parents can stop the assault of graphic images in their children's lives, and to insist on a quick and easy way for all Internet users to be able to make offensive web sites off limits to their children. Balancing Work and Family. If we are to value our families, we have to make much more progress. Strengthening America's families means helping parents make time for their children. We need to find new ways to help parents balance work and family so that they will have time to pass on the right values to their children. Already millions of Americans have benefitted from the Family and Medical Leave law, now we need to expand it so that it covers parent-teacher visits and children's routine medical appointments. And we will extend the law to cover more employers so that more working families enjoy this vital protection during times of family and medical need. We should urge employers to make workplaces more parent-friendly; explore strategies, including voluntary initiatives and policy reforms, that can provide income support for workers during periods of family and medical leave; call on parents to be more involved in their children's learning; and fix the \"marriage penalty\" so that parents can spend more time at home and less time trying to make ends meet. We should not penalize families by forcing couples to pay more in taxes just because they have made the sacred commitment of marriage to one another. We should also provide grants to community and faith-based organizations to help couples prepare for and strengthen their marriage and relationships, become better parents, and reduce domestic violence. Child Care and Early Childhood Education. Democrats believe in making child care more affordable through targeted tax cuts and other investments, in improving the safety and quality of child care centers, in requiring accountability so that federal monies and subsidy payments are effectively used to provide quality in child care, in ensuring that children start school ready to read, and in giving a helping hand to parents who decided to stay at home with their children. We need both higher pay and higher standards for child care workers - and they need to get training so that they can do their jobs well. It is a priority of the Democratic Party to fully fund Head Start. Eldercare. The Baby Boomers are the first generation with more parents than children. Many families are doing all they can to help for and care for their elderly parents. These families are doing the right thing - and America must be on their side. We must do more to support the families and individuals who are caring for relatives suffering from long-term illnesses at home or at institutions. We should provide Americans with long-term care needs and their caregivers a $3,000 tax credit. We should hold those who care for our nation's elderly to the highest standards and improve these workers' wages, benefits, training, and working conditions. We should make sure that every community in the country has a program to offer caregivers critical information, referrals, and respite from the difficult work of caring for a loved one. Fatherhood. Promoting responsible fatherhood is the critical next phase of welfare reform and one of the most important things we can do to reduce child poverty. Three times more men acknowledged paternity in 1998 than in 1993. This is a first step toward giving to a child the emotional and financial support a father must give to merit the name. Democrats believe in cracking down on deadbeats who abandon their children. So we must require all fathers who owe child support to pay or go to work; strengthen child support enforcement, including increasing the amount of child support that gets paid directly to poor families; and make it harder for parents who owe child support to get new credit cards. However, we also recognize that, in addition to dead beat dads there are dead broke dads. Thus Democrats support helping those men who want to reconnect with their families and who want to become a positive force in the lives of their children. Responsible Entertainment. Parents are struggling to pass on the right values in a culture that sometimes seems to practically scream that chaos and cruelty are cool. Democrats have worked to give parents the tools to have more control over the images their children are exposed to. Parents and the entertainment industry must accept more responsibility. Many parents are not aware of the resources available to them, such as the V-chip technology in television sets and Internet filtering devices, that can help them shield children from violent entertainment. The entertainment industry must accept more responsibility and exercise more self-restraint, by strictly enforcing movie ratings, by taking a close look at violence in its own advertising, and by determining whether the ratings systems are allowing too many children to be exposed to too much violence and cruelty. Democrats call for the reinstatement of the Fairness Doctrine by the Federal Communications Commission. We believe in public support for the arts, including the National Endowment for the Arts and the National Endowment for the Humanities. Public and private investment in creativity and cultural heritage - the arts and humanities - is an investment in the education of our children, in the well being of our communities, in the strength of our economy, and in spreading the dream of democracy throughout the world. ACCESSIBLE, AFFORDABLE, QUALITY HEALTH CARE For fifty years, the Democratic Party has been engaged in a battle to provide the kind of health care a great nation owes its people. We reaffirm our commitment to take concrete, specific, realistic steps to move toward the day when every American has affordable health coverage. And we will not rest until the job is done. During the past eight years, Democrats have helped Americans keep their doctor when they lose or change jobs. We passed the Child Health Insurance Program to help states provide health coverage to millions of uninsured children - the largest single investment in children's health in a 35 years. We kept solvent a Medicare system that was scheduled to go bankrupt this year. We brought immunization rates to an all-time high. In contrast, the Republican Party has refused to use one penny of the surplus to secure the solvency of Medicare and has supported plans that would increase Medicare premiums, force elderly patients into HMOs and raise the eligibility age for Medicare to 67. They have adamantly opposed the Patients' Bill of Rights and proposed instead a mirage \"Patient's Bill of Goods\" that would leave out a real guarantee of the right to see a specialist and assurances that you can go to the nearest emergency room - and leave out 135 million Americans in the cold. Instead of the guaranteed, universal prescription drug benefit that Democrats believe should be added to Medicare, Republicans are proposing to leave to insurance companies the decisions about whether and where a drug benefit might be offered, what it would include, and how much it would cost. Studies suggest that less than half of seniors will be able to use this benefit. Universal Health Coverage. There is much more left to do. We must redouble our efforts to bring the uninsured into coverage step-by-step and as soon as possible. We should guarantee access to affordable health care for every child in America. We should expand coverage to working families, including more Medicaid assistance to help with the transition from welfare to work. And we should also seek to ensure that dislocated workers are provided affordable health care. We should make health care accessible and affordable for small businesses. In addition, Americans aged 55 to 65 - the fastest growing group of uninsured - should be allowed to buy into the Medicare program to get the coverage they need. By taking these steps, we can move our nation closer to the goal of providing universal health coverage for all Americans. A Real Patients' Bill of Rights. Medical decisions should be made by patients and their doctors and nurses, not accountants and bureaucrats at the end of a phone line a thousand miles away. It is time we meaningfully addressed concerns about the quality of care and about the decline of patient, access, trust, and satisfaction. People need to get the health care they need, when they need it, without having to leap endless hurdles. Americans need a real, enforceable Patients' Bill of Rights with the right to see a specialist, the right to appeal decisions to an outside board, guaranteed coverage of emergency room care, and the right to sue when they are unfairly denied coverage. Al Gore will work with a wide range of stakeholders to develop a national strategy to reduce medical errors, including appropriate public reporting, analysis of root causes, and development of error prevention models. Democrats also believe that doctors, nurses, and other health care practitioners must be allowed to advocate freely on behalf of their patients. Protecting and Strengthening Medicare. It is time we ended the tragedy of elderly Americans being forced to choose between meals and medication. It is time we modernized Medicare with a new prescription drug benefit. This is an essential step in making sure that the best new cures and therapies are available to our seniors and disabled Americans. We cannot afford to permit our seniors to receive only part of the medical care they need. Democrats believe Medicare is worth fighting for - and worth saving. With the number of Americans on Medicare expected to double in the next 35 years, Al Gore has stepped up and taken responsibility by proposing a Medicare Lock Box that would insure Medicare surpluses are used for Medicare - and not for pork barrel spending or tax giveaways. We should also modernize Medicare by promoting competitive prices and remain vigilant against Medicare fraud. Fighting Diseases. Our newest medical miracles give us the chance to make significant progress in battling some of the most dreaded diseases. Democrats believe that we must invest in biomedical research and continue to fight and conquer everything from AIDS to Alzheimers to Diabetes to Parkinsons to spinal cord injuries. We must speed up the development of new drugs and get them to patients sooner while maintaining essential health and safety standards. We should allow stem cell research to make important new discoveries. We should expand prevention and widen access to clinical trials. And we should devote more resources to eliminating disease disparities among racial and ethnic groups. Our nation must do all it can to focus its efforts on fighting HIV and AIDS. A top priority for Democrats will be the continued investment in research, prevention, care, treatment, and we are deeply committed to the search for a cure. Democrats continue to support important programs such as the Ryan White CARE Act, the Housing Opportunities for People with AIDS program, and incentives to return Americans with HIV/AIDS to work. For a generation, America has been waging a war on cancer. Al Gore believes it is time we started winning it. Because of astonishing scientific breakthroughs, the day that America is cancer-free is within reach. With the completion of the draft of the Human Genome, we are on the verge of cracking cancer's secret code. Democrats believe in taking advantage of this progress by doubling federal cancer research. Fighting Teen Smoking. Al Gore is committed to dramatically reducing teen smoking in America. It is time we treated underage tobacco use like the health crisis it is. That's why we need to give the FDA full authority to keep cigarettes away from children. We must match the power of big tobacco's advertising dollars with a counter-campaign that tells kids the truth about the dangers of smoking and the risks of cancer to themselves and to others through second-hand smoke. And we should double our investment in efforts to prevent teen smoking and break the deadly grip of nicotine addiction. State attorneys general across America have recovered billions of dollars from the tobacco industry for damages caused by tobaccos' advertising directed at our children and for the death and disease created by cigarettes. Now Republicans are trying to stop the United States Justice Department from pursuing similar litigation to hold the tobacco companies accountable for the damages they have caused to American taxpayers. We believe it is wrong to insulate the tobacco companies from liability for their wrongdoing. Mental Health. Mental illness has long been concealed behind a shroud of silence and shame. Mental illness affects nearly one in five Americans each year, but nearly two-thirds of those Americans affected by mental disorders do not receive help. When mental illness goes untreated, undiagnosed, and unmentioned, people are denied the opportunity to live full lives and our nation is denied their full contribution. Democrats believe in supporting families caring for loved ones with mental illness by strengthening our community mental health system, providing access to full mental health coverage for every child in America, giving teachers and schools more mental health resources, and ensuring that mental illness and physical illness are treated equally by our nation's health plans. Disabilities. Democrats believe that we must fight to ensure that people with disabilities can meet their full potential and participate fully in the American dream. For people with disabilities accessing affordable health insurance is the greatest barrier to returning to work. That is why we fought to assure that people with disabilities do not lose their health care when they return to work. Democrats also support tax credits and grants to pay for rehabilitation and work-related expenses for people with disabilities. And we support all efforts to implement the Supreme Court's Olmstead decision and to make personal assistance services and supports available to people with disabilities in their homes and communities - because no one should be kept in a nursing home or institution if they prefer to live in the community with the necessary supports. CHOICE The Democratic Party stands behind the right of every woman to choose, consistent with Roe v. Wade, and regardless of ability to pay. We believe it is a fundamental constitutional liberty that individual Americans - not government - can best take responsibility for making the most difficult and intensely personal decisions regarding reproduction. This year's Supreme Court rulings show to us all that eliminating a woman's right to choose is only one justice away. That's why the stakes in this election are as high as ever. Our goal is to make abortion less necessary and more rare, not more difficult and more dangerous. We support contraceptive research, family planning, comprehensive family life education, and policies that support healthy childbearing. The abortion rate is dropping. Now we must continue to support efforts to reduce unintended pregnancies, and we call on all Americans to take personal responsibility to meet this important goal. The Democratic Party is a party of inclusion. We respect the individual conscience of each American on this difficult issue, and we welcome all our members to participate at every level of our party. This is why we are proud to put into our platform the very words which Republicans refused to let Bob Dole put into their 1996 platform and which they refused to even consider putting in their platform in 2000: \"While the party remains steadfast in its commitment to advancing its historic principles and ideals, we also recognize that members of our party have deeply held and sometimes differing views on issues of personal conscience like abortion and capital punishment. We view this diversity of views as a source of strength, not as a sign of weakness, and we welcome into our ranks all Americans who may hold differing positions on these and other issues. Recognizing that tolerance is a virtue, we are committed to resolving our differences in a spirit of civility, hope and mutual respect.\" PROTECTING OUR ENVIRONMENT Democrats know that for all of us there is no more solemn responsibility than that of stewards of God's creation. That is why we have worked for eight years to produce the cleanest environment in decades: with cleaner air, cleaner water, and a safer food supply; a record number of toxic waste dumps cleaned up; new smog and soot standards so that children with asthma and the elderly would be able to live better lives; and a strong international treaty to begin combating global warming - in a way that is market-based and realistic, and does not lead to economic cooling. From the Redwood forests to the Florida Everglades, from the Grand Canyon to Yellowstone to Yosemite, we have protected millions of acres of our precious natural lands. We stopped development in America's last wild places. Teddy Roosevelt saw our national parks as the playground of the people - there for average families to enjoy with camping and hiking. Today's Republicans see them as the playground of the powerful - there for big businesses to exploit with drilling and mining. The Republicans have tried to sell off national parks; gut air, water, and endangered species protections; let polluters off the hook; and put the special interests ahead of the people's interest. They are wrong. Out natural environment is too precious and too important to waste. Al Gore is committed to restoring the Everglades; protecting the coasts of California and Florida and the Arctic National Wildlife Refuge from oil and gas drilling; and preserving our untouched forests, including the Tongass, from logging and development. With regard to public lands, Democrats believe that communities, environmental interests, and government agencies should work together to protect our public resources, critical habitat areas, and wildlands while ensuring the vitality of local economies. We will work together to find land-based alternatives and decontamination technologies that will permanently end the ocean disposal of contaminated dredge spoils. Once Americans were led to believe they had to make a choice between the economy and the environment. They now know that this is a false choice. But there is a real choice to make in 2000: whether we will protect our environment in ways that are practical and achievable or go back to the policies that led to generations of environmental devastation and degradation. We have to do what's right for our Earth because it is the moral thing to do. It involves all of our lives - from the simple security of having clean safe, reliable, affordable electricity for your home; to America's ability to build and sell the best new clean cars, trucks, and technology to the world; to guarding our children from the summer smog that is made worse by global warming, and securing for our grandchildren the expectation of a joyful array of seasons that we took for granted when we grew up ourselves. Democrats believe we must give Americans incentives to invest in driving more fuel-efficient cars, trucks, and sport utility vehicles; living in more energy-efficient homes, and using more environmentally-sound appliances and equipment. We need to clean up aging power plants. We must invest in rebuilding and improving our transportation infrastructure and ensure that we adequately maintain these systems for the future. Americans need and rely on diverse transportation sources, and our public infrastructure priorities should reflect that diversity. We should invest in roads, bridges, light rail systems, cleaner buses, the aviation system, our national passenger railroad, Amtrak, and high-speed trains that would give Americans choices - freeing them from traffic, smog-choked cities, and being held hostage to foreign oil. We should ensure that urban communities affected by the presence of airports which create increased levels of noise and pollution be provided mitigation support to address these concerns. We must also ensure that we maintain adequate public funding and public administration of publicly operated and delivered transportation services, without gutting collective bargaining agreements or long-standing worker-protections. In these and other areas, we will encourage project labor agreements, fostering labor-management cooperation, quality development, and efficient use of public monies. Today, technology has advanced to the point that we can drive the kind of cars we like and live in the kind of houses we like - while being kind to the earth. We should use some of our budget surplus to help Americans take advantage of these new opportunities. With the right investments, these new environmentally-friendly technologies can create new jobs for American workers. America is blessed with abundant low-cost sources of coal, petroleum, and natural gas, but we must use them wisely and ensure that changes in the energy sector promote a workforce whose skills are expanded, utilized, and rewarded. Democrats believe that with the right incentives to encourage the development and deployment of clean energy technologies, we can make all our energy sources cleaner, safer, and healthier for our children. This responsibility includes disposing of nuclear waste in a scientifically-sound manner in accordance with standards designed to protect human health and the environment. And we must dramatically reduce climate-disrupting and health-threatening pollution in this country, while making sure that all nations of the world participate in this effort. Environmental standards should be raised throughout the world in order to preserve the Earth and to prevent a destructive race to the bottom wherein countries compete for production and jobs based on who can do the least to protect the environment. There will be no new bureaucracies, no new agencies, no new organizations. But there will be action and there will be progress. The Earth truly is in the balance - and we are the guardians of that harmony. Eight of the ten hottest years ever recorded have occurred during the past ten years. Scientists predict a daunting range of likely effects from global warming. Much of Florida and Louisiana submerged underwater. More record floods, droughts, heat waves, and wildfires. Diseases and pests spreading to new areas. Crop failures and famines. Melting glaciers, stronger storms, and rising seas. These are not Biblical plagues. They are the predicted result of human actions. They can be prevented only with a new set of human actions - big choices and new thinking. Working with the America's great automakers, Al Gore has led the Partnership for a New Generation of Vehicles which has helped spur the development of high-performing cars that get far better gas mileage while meeting emissions standards. Now we need to give Americans help in being able to afford these new cars - getting them out of the showrooms, onto the streets, and into our driveways. At the same time, we are committed to improving fuel economy in a way that preserves and creates jobs for American workers, and delivers products that consumers want to buy. To further this kind of progress, we now need the oil industry to join us in producing much cleaner fuels that will allow automotive environmental equipment to achieve the maximum possible reductions in emissions. We have also created a new 21st Century Truck Initiative to build highly-efficient heavy duty pick-up and delivery trucks, even long-haul 18-wheelers. Now we need to work in partnership with industry to create a new generation of mass transit and a new generation of cleaner, more reliable power systems. Al Gore wants to swap every dirty, smoke-belching city bus for a cleaner, less polluting one. RENEWING OUR DEMOCRACY AND CAMPAIGN FINANCE REFORM In the year 2000, along with all the other big choices they have to make, Americans will be making a choice about who's running their country: the people or the special interests, the voters or the lobbyists, the many or the few. We must restore American's faith in their own democracy by providing real and comprehensive campaign finance reform, creating fairer and more open elections, and breaking the link between special interests and political influence. The Republicans will have none of this. Instead of limiting the influence of the powerful on our politics, they want to raise contribution limits so even more special interest money can flow into campaigns. The big-time lobbyists and special interest were so eager to invest in George W. Bush and deliver campaign cash to him hand-over-fist that he became the first major party nominee to pull out of the primary election financing structure and refuse to abide by campaign spending limits. In this year's presidential primaries it became clear that the Republican establishment is violently opposed to John McCain's call for reforming our democracy. Al Gore supports John McCain's campaign for political reform. In fact, the McCain-Feingold bill is the very first piece of legislation that a President Al Gore will submit to Congress - and he will fight for it until it becomes the law of the land. Then he will go even further - much further. He will insist on tough new lobbying reform, publicly-guaranteed TV time for debates and advocacy by candidates, and a crackdown on special interest issue ads. Most boldly of all, Al Gore has proposed a public-private, non-partisan Democracy Endowment which will raise money from Americans and finance Congressional elections - with no other contributions allowed to candidates who accept the funding. This will let our politics be free from the influence of special interests and let Americans believe in their own democracy again. Just as our country has been the chief apostle of democracy in the world, we must lead by example at home. This begins with our nation's capital. The citizens of the District of Columbia are entitled to autonomy in the conduct of their civic affairs, full political representation as Americans who are fully taxed, and statehood. Puerto Rico has been under U.S. sovereignty for over a century and Puerto Ricans have been U.S. citizens since 1917, but the island's ultimate status still has not been determined and its 3.9 million residents still do not have voting representation in their national government. These disenfranchised citizens - who have contributed greatly to our country in war and peace - are entitled to the permanent and fully democratic status of their choice. Democrats will continue to work in the White House and Congress to clarify the options and enable them to choose and to obtain such a status from among all realistic options. Democrats believe the people of Guam, American Samoa, and the Virgin Islands have a right to be fully self-governing. We are committed to fair treatment in economic and social policies as well as improvement in federal-territorial relations in accordance with the needs of each area. Elected representatives of these areas will be regularly consulted on policies, laws, and treaties that affect the areas and we will ensure fair treatment for our fellow citizens in the territories. BUILDING ONE AMERICA Democrats believe that God has given the people of our nation not only a chance, but a mission to prove to men and women throughout this world that people of different racial and ethnic backgrounds, of all faiths and creeds, can not only work and live together, but can enrich and ennoble both themselves and our purpose. America's diversity is expanding, yet amidst important signs of progress, there is widespread evidence of persistent discrimination, growing racial segregation of our schools and neighborhoods, and dream-crushing barriers to opportunity. We cannot - we dare not - remain a nation divided. Our vision is of an America healed of hatreds and misunderstanding, with equality and opportunity so rich that legacies of discrimination and exclusion will be found only in history books, and not in our communities. To that end, Democrats support creation of a commission of distinguished scholars and civic leaders to examine the history of slavery, discrimination, and exclusion suffered by all minorities; to report on the continuing effects of those tragic chapters in our history; and to make appropriate recommendations on behalf of the American people. Welcoming Our Newest Americans. Immigrants enrich the tapestry of American life, making our economy more vibrant, our workplaces more productive, and our nation stronger. We believe that all levels of government, in partnership with the private and voluntary sectors, must devise and pursue a comprehensive immigrant integration agenda that will make the newest Americans full participants in the nation's mainstream. That's why Democrats support reforming the INS to provide better services, and investing the resources needed to reduce the backlog of citizenship applications from nearly two years to three months. Democrats also support increased resources for English language courses, which not only help newcomers learn our common language but also help us promote our common values. And, we believe that family reunification should continue to be the cornerstone of our legal immigration system. Democrats believe in an effective immigration system that balances a strong enforcement of our laws with fair and evenhanded treatment of immigrants and their families. The Clinton-Gore administration provided long overdue leadership in dramatically improving border management and law enforcement, including a major expansion of the Border Patrol and curbs on abuses of the asylum process. We also recognize that the current system fails to effectively control illegal immigration, has serious adverse impacts on state and local services, and on many communities and workers, and has led to an alarming number of deaths of migrants on the border. Democrats are committed to reexamining and fixing these failed policies. We must punish employers who engage in a pattern and practice of recruiting undocumented workers in order to intimidate and exploit them, and provide strengthened protections for immigrant workers, including whistleblower protections. Doing so enhances conditions for everyone in the workplace. We believe that any increases in H1-B visas must be temporary, must address only genuine shortages of highly skilled workers, and mist include worker protections. They must also be accompanied by other immigration fairness measures and by increased fees to train American workers for high skill jobs. The Democratic Party is committed to assuring an adequate, predictable supply of agricultural labor while protecting American farm workers who are among the poorest and more vulnerable in our society. We reject calls for guest worker programs that lead to exploitation, and instead call for adjusting the status of immigrants with deep roots in the country. We should have equitable asylum policies that treat people the same whether they have fled violence from the Right and Left. And we support restoration of basic due process protections and essential benefits for legal immigrants, so that immigrants are no longer subject to deportation for minor offenses, often committed decades ago without opportunity for any judicial review, and are eligible to receive safety net services supported by their tax dollars. Fighting for Civil Rights and Inclusion. Passage of the Civil Rights Act of 1964 was one of the proudest moments of our nation's history and a sterling testament to our aspirations as a people. Yet, despite undeniable progress over the last several decades, inequality and polarization nevertheless persist in far too many American workplaces, schools, and communities. Over the last eight years, we have fought hard to end discrimination. We have increased funding for civil rights enforcement - so that the laws on our books are not just pleasant words, but pledges of justice. Al Gore has strongly opposed efforts to roll back affirmative action programs. He knows that the way to lift this nation up is not by pulling the weakest down, but by continuing to expand opportunities for everyone who wants to achieve. The Clinton-Gore Administration has appointed the most diverse administration in American history, demonstrating that pursuing excellence means including the all of the best that our nation has to offer. Al Gore and the Democratic Party know that much remains to be done. We must remember we do not have an American to waste. We continue to lead the fight to end discrimination on the basis of race, gender, religion, age, ethnicity, disability, and sexual orientation. The Democratic Party has always supported the Equal Rights Amendment and will continue to do so, and we are committed to ensuring full equality for women and to vigorously enforcing the Americans with Disabilities Act. We support continuation of the White House initiative on Asian Americans and Pacific Islanders. Because every American counts, we will continue to work toward a census that counts every American. We support continued efforts, like the Employment Non-Discrimination Act, to end workplace discrimination against gay men and lesbians. We support the full inclusion of gay and lesbian families in the life of the nation. This would include an equitable alignment of benefits. We recognize the importance of new battles against forms of discrimination and disadvantage that stand as barriers to communities and families, such as environmental injustices and predatory lending practices. And we will fight for full funding and full staffing of the Equal Employment Opportunity Commission and other civil rights enforcement agencies so they can do their job of ensuring that America lives up to its creed of equal rights and equal opportunity for all. The Democratic Party proudly upholds its tradition of support for the first Americans. The sovereignty of the American Indians and Native Alaskans and a strong affirmation of the government-to-government relationship are basic to our approach to the tribal governments. As we move into the 21st century, we have to renew our trust obligations and work to improve the lives of the many Indians who live in terrible poverty. The Democratic Party pledges to continue our work to make a difference in the lives of those who occupied this land before us. We affirm the legal and political relationship between the United States and Native Hawaiians as an important step in the continuing process of reconciliation. We will work to pass legislation establishing a process for Native Hawaiians to reorganize a governing body, freely chosen, expressing their rights to self-determination. The justice we provide the first Americans is a measure of our nation's character, and Democrats believe we should build on the progress of the last eight years. Forging Common Ground. American citizenship entails both rights and responsibilities and we need to ask every American - from every walk of life - to give something back to their communities and their country. We are committed to expanding AmeriCorps so that more Americans both serve their country and further their educations. America will become much more diverse in the coming century. But while much is changing, much remains. Our common civic culture - one grounded in the values most Americans share: work, family, personal responsibility, individual liberty, and faith - ties us together. Our common ground - our shared civic institutions - makes us whole. In the years to come, we must celebrate our diversity and focus on strengthening the common values and beliefs that make us one America - one nation, under God, with liberty and justice for all. PEACE Eight years ago, Americans found themselves between two worlds. After half a century in which we stood up for peace and security all over the globe - taking on the forces of tyranny and terror that imperiled our interests and assaulted our values - the Cold War was over and a new Global Age was beginning. We needed new ideas and new leadership. Democrats have provided them. Under the leadership of Bill Clinton and Al Gore, the first light of the 21st Century finds America at peace. More of the world's citizens live in freedom than ever before, and our people and our values are protected by the greatest military force the world has ever known. Democratic leadership has brought peace and security to Americans and to millions of freedom-loving people around the globe. We achieved victory and ended ethnic cleansing in Kosovo - allowing hundreds of thousands of refugees to return to their homes in safety. We helped achieve historic breakthroughs in the Middle East peace process. We led the efforts that produced the Good Friday Accord in Northern Ireland - offering the best hope yet of ending decades of bloodshed. We are working to build a self-sustaining peace in Bosnia through the implementation of the Dayton Peace accords. We have ended the military dictatorship and given democracy a chance in Haiti. We have made Americans safer by reducing Russian nuclear arsenals. We strengthened and expanded NATO for a new century. But now is not the time to sound the trumpets of triumph. In the wake of the Cold War, America has entered a new Global Age that is altering our security challenges and creating entirely new issues. Globalization is transforming the international order that defined the 20th century. Today, for both good and ill, our destiny and the destinies of billions of people around the world are increasingly intertwined, and our domestic and international challenges are bound together as never before. The Democratic Party recognizes that globalization will continue shaping our future. We also believe that the United States has the means and the responsibility to shape globalization so that it reflects the needs and the values of the American people. Al Gore and the Democratic Party know that we must be able to meet any military challenge from a position of dominance. But Al Gore and the Democratic Party also recognize that there is a new security agenda - threats that affect the entire world and transcend political borders. During the past century, we have learned that if we wish to avoid war, we must be strong enough to deter aggression, but also farsighted enough to invest in peace. Now it is time to apply this lesson to the new global challenges we face - to shape a new strategy of Forward Engagement to guide our conduct around the world. Forward Engagement means addressing problems early in their development before they become crises, addressing them as close to the source of the problem as possible, and having the forces and resources to deal with these threats as soon after their emergence as possible. While we must always stand prepared to use our military power when all other options fail, Forward Engagement also means addressing societal and political problems before they evolve into threats to our national security and values - before armed conflict becomes the only way to achieve our goals. And Forward Engagement means drawing on all three main sources of American power - military strength, a vibrant, growing economy, and a free and democratic political system - to advance our objectives around the world. The Democratic Party believes that America's peace and security depend on our unflagging leadership and engagement in global affairs-and that Forward Engagement is the strategy that must guide us. We must maintain America's economic and military strength. We must also form partnerships to help solve global problems and take advantage of new global opportunities. That means we must deepen our key alliances, develop more constructive relationships with former enemies, and bring together diverse coalitions of nations to deal with new problems. America has a responsibility to lead - and should lead from within the international community. At a time when new conditions require new thinking, the Republican Party offers little more than outdated positions and a narrow worldview that lets international problems fester. Some Republicans believe America should turn away from the world. They oppose using our armed forces as part of international solutions, even when regional conflicts threaten our interests and our values. Other Republicans want America to act unilaterally. They attack the Anti-Ballistic Missile Treaty - even at the risk of precipitating a new nuclear arms race. They voted down the Comprehensive Test Ban Treaty, threatening both our security and our global leadership. They have attempted to sabotage the Clinton-Gore administration's efforts to negotiate with other nations by declaring that any arms control agreement - regardless of content - would be \"dead on arrival.\" Mired in the past, the Republican Party fails to realize that ensuring peace and security for Americans today does not just mean guarding against armies on the march. It means investing in building the global peace. It means addressing the fact that more than 1 billion of the Earth's inhabitants live on less than $1 a day - inviting social dislocation, violence, and war. It means meeting new challenges such as international crime and terrorism, environmental degradation, and pandemic diseases head-on. And it means that Forward Engagement must be the new pole-star of our global strategy. NEUTRALIZING THE FORCES THAT CAUSE CHAOS AND INSTABILITY The questions of war and peace among sovereign states are as important to our security as ever. But today America also faces a new set of international issues. Technology's unprecedented power means that lawlessness, diseases, and ecological disruptions - which once were localized - now land on America's doorstep even as they also threaten the stability and security of nations all over the world. Disruption of the World's Ecological System. The disruption of the world's ecological systems - from the rise of global warming and the consequent damage to our climate balance, to the loss of living species and the depletion of ocean fisheries and forest habitats - continues at a frightening rate. We must act now to protect our Earth while preserving and creating jobs for our people. In 1997, we negotiated the historic Kyoto Protocols, an international treaty that will establish a strong, realistic, and effective framework to reduce greenhouse emissions in an environmentally strong and economically sound way. We are working to develop a broad international effort to take action to meet this threat. Al Gore and the Democratic Party believe we must now ratify those Protocols. Global Epidemics. Global epidemics constitute another major security threat. Malaria is running out of control in Africa, and antibiotic-resistant strains of tuberculosis are ravaging Russia and other countries. But the most severe global epidemic is HIV/AIDS. It is more than a health tragedy, it is a threat to global security. AIDS now grips 20 million Africans. Fourteen million have already died, a quarter of them children. Each day, 11,000 more men, women, and children become infected. Diseases like AIDS threaten not just individual citizens, but the very institutions that define and defend the character of society. The Democratic Party believes we can and must do more to prevent transmission, care for those who are ill, and lead in knitting. together the scores of AIDS-fighting initiatives into a global campaign to defeat this threat. Fighting Drugs and Organized Crime. International drug networks and other organized crime syndicates represent a growing threat to the survival of democratic governance. They breed corruption and lawlessness and they erode the institutions that maintain societal order. Drug producing nations like Colombia have seen their societies torn apart by the intersection of criminal activity, political discord, and terrorism. And our nation is also afflicted with the violence and hopelessness of drugs. We must continue to combat narco-traffickers, increasing our budget to do so. We must continue to have a strong Drug Czar who can bring together the considerable resources of the U.S. Government in this effort. We must continue to fight those who make the financing of this effort possible such as the money launderers who facilitate the drug trade. We must continue to work with our friends and allies and international organizations to fight the blood money of the drug trade by getting a handle on those nations who turn a blind eye to the financial end of this problem. We must remember that the drug trade, like other criminal enterprises, fundamentally reflects the economics of hopelessness. Farmers have been drawn to cultivate these crops as a means for economic survival in the absence of other viable alternatives. Al Gore and the Democratic Party understand that no policy of interdiction and prosecution will succeed unless it is combined with robust investment in alternative ways to make a living. We must also build on our efforts to expand the rule of law, fight corruption, and improve democratic governance. TRANSFORMING OUR MILITARY A strong, flexible, and modern military force is the ultimate guarantor of our physical survival and the protection of our interests and values. Today, America's military is the best-trained, best-equipped, most capable, and most ready fighting force in the world. With Bill Clinton and Al Gore in the White House, Democrats reversed a decline in defense spending that began under President Bush, boosted pay and allowances, and provided the funding for a new generation of weapons. The Democratic Party understands that, good as they are, the armed forces must continue to evolve. They must not only remain prepared for conventional military action, but must sharpen their ability to deal with new missions and new kinds of threats. They must become more agile, more versatile, and must more completely incorporate the revolutionary implications and advantages of American supremacy in information technology. Recruiting, Training, and Retaining Our Troops. A high-tech fighting force must recruit, train, and retain a professional all-volunteer force of the highest caliber. The Democratic Party understands that in order to do this, military pay must continue to increase. We enacted the largest military pay increase in twenty years - and we must raise pay even more. We need to further reform the military retirement system and improve housing, health care, and childcare benefits to support the general competitiveness of military careers during a period of unprecedented prosperity in the civilian economy. While the number of soldiers and families on food stamps is down by two-thirds over the past decade, it is unacceptable that any member of our armed forces should have to rely on food stamps. Al Gore is committed to equal treatment of all service members and believes all patriotic Americans be allowed to serve their country without discrimination, persecution, and violence. The Democratic Party honors America's veterans for their selfless willingness to defend the United States and promote our values around the world. We must always remember the debt this nation owes its defenders. Al Gore will expand access to health care for all eligible veterans; pursue the causes of illness suffered by Vietnam and Gulf War veterans; press for more research on diseases caused by exposure to toxic battlefields and treat fairly veterans suffering from those ailments; back research efforts to screen and treat hepatitis C; and expand programs in the areas of mental health, spinal cord injury, and vision impairment. We will streamline the disability claims process to ensure that this nation continues to live up to its sacred commitment to the men and women who served in uniform. We support efforts of the Filipino American Veterans who fought in World War II to obtain equity. Deploying America's Technological Edge. It is imperative that aging weapons systems - which are now the backbone of our military - be replaced by the oncoming generation of advanced, high-tech weapons which are designed to make sure that our armed forces face any future conflict from a posture of dominance. Al Gore and the Democratic Party will make sure that the military has the most advanced weaponry, sophisticated intelligence, and information systems and, in addition, continues to invest in research and development for future supremacy. By contrast, George W. Bush has talked about \"skipping\" this generation of weapons - which could mean skipping our responsibility to give our fighting men and women the weapons they need. We must also ensure that investment in the infrastructure needed to support the military, including our maritime capability, is not ignored. And we must ensure a competitive workforce maintaining high-skilled workers and training programs that will ensure the capability to respond to national security emergencies and defense readiness. Protecting Our Interests and Securing Our Values. The lessons of the past eight years show that the nation must be prepared to use force when American interests and values are truly at stake. We cannot be the world's policeman, and we must be discriminating in our approach. But where the stakes are high, when we can assure ourselves that nothing short of military engagement can secure our national interest, when we know that we have the military forces available for the task, when we have made our best efforts to join with allies, and when the cost is proportionate to the objective, we must be ready to act. CLOSING THE GATES OF WAR In areas where conflict has raged, comprehensive peace agreements are the foundation for lasting security. Bill Clinton and Al Gore have actively pursued peaceful resolutions to conflicts across the world and have been prepared to go the extra mile on behalf of negotiators seeking peace. Al Gore and the Democratic Party are fundamentally committed to the security of our ally, Israel, and the creation of a comprehensive, just, and lasting peace between Israel and its neighbors. We helped broker the Israel-Jordan Peace Treaty, the Wye River accords, and the Sharm el-Sheik Memorandum, and will continue to work with all parties to make progress towards peace. Our special relationship with Israel is based on the unshakable foundation of shared values and a mutual commitment to democracy, and we will ensure that under all circumstances, Israel retains the qualitative military edge for its national security. Jerusalem is the capital of Israel and should remain an undivided city accessible to people of all faiths. In view of the government of Israel's courageous decision to withdraw from Lebanon, we believe special responsibility now resides with Syria to make a contribution toward peace. The recently-held Camp David summit, while failing to bridge all the gaps between Israel and the Palestinians, demonstrated President Clinton's resolve to do all the United States could do to bring an end to that long conflict. Al Gore, as president, will demonstrate the same resolve. We call on both parties to avoid unilateral actions, such as a unilateral declaration of Palestinian statehood, that will prejudge the outcome of negotiations, and we urge the parties to adhere to their joint pledge to resolve all differences only by good faith negotiations. In Northern Ireland, we helped facilitate multi-party talks and played an instrumental role in brokering the historic Good Friday Accord, which has greatly enhanced the prospect for peace. We will continue to work toward implementation of the Accord and provide continued political and economic support for the new institutions involving Northern Ireland, the Republic of Ireland, and Great Britain. Our goal is not merely the laying down of arms, but the joining together of hands in a new political relationship that enables former rivals to govern and thrive together. We have worked hard and successfully to calm dangerous tensions between our allies Greece and Turkey over issues in the Aegean, and we have never ceased our efforts to facilitate a resolution of tensions between the Greek and Turkish communities on Cyprus. This work must continue. In the Balkans, the Clinton-Gore Administration ended ethnic cleansing in Bosnia and Kosovo by the resolute use of military power and vigorous diplomacy. The Republican Party, having first opposed the Administration's efforts to restore peace in the region, now tries to impede the Administration's efforts to rebuild these shattered societies. We look forward to the day when Serbia will be free from the grip of Slobodan Milosevic, and we will work to make that happen. America did right in the Balkans, and now we must finish the job. Remembering the historic suffering of the people of Armenia, and recognizing the need of the modern Armenian state for security and economic growth, Al Gore and the Democratic Party are committed to continuing our efforts to bring a permanent end to tensions between Armenia and Azerbaijan over Nagorno-Karabakh, along with the restoration of diplomatic, commercial, and economic ties between Armenia and her neighbors, including Turkey. Al Gore helped bring about a special task force to intensify economic cooperation between the United States and Armenia. We have helped close the gates of war in other parts of the world as well, and our work continues. We helped settle the Peru-Ecuador border dispute and end the civil war in Guatemala. We have worked for peace in the Democratic Republic of Congo, the Central African Republic, Sierra Leone, and on the Ethiopia-Eritrea border. And we helped end the violence and protect democracy in East Timor by leading diplomatic efforts and supporting an international peacekeeping mission. We helped facilitate the dialogue between North and South Korea, without which the recent summit could not have occurred. We continue to work with China and Taiwan to resolve their differences by peaceful means. And we continue our work with India and Pakistan to dampen down a nuclear arms race on the sub-continent and continue to urge them to deal with their differences over their conflict in Kashmir with peaceful means. President Clinton's historic trip to India and Pakistan has created new possibilities for dialogue with these countries, and under a Gore Administration these will be continued vigorously. ENGAGING FORMER ENEMIES Democrats understand that we must engage former enemies. This Administration's efforts to design new relationships with the Russian Federation and China have been continuously subjected to every form of harassment and attack by the Republicans - but they have been in America's national interest and they have been the right thing to do. We recognize that Russia's historic transition to a market democracy is difficult - all the more reason we must continue to engage Russia. We recognize that Russian democracy is challenged by corruption that deeply penetrates her society - all the more reason to engage Russia on behalf of reform. We recognize that Russia has her own self-interest and concerns that can and do run contrary to ours - all the more reason to search for constructive forms of cooperation. We deeply disagree with what Russia is doing in Chechnya and remain concerned about signs of Russian efforts to intimidate the press - all the more reason to step up our discussions with them on those issues. The Democratic Party is prepared to pursue American objectives as needed even at the cost of friction with Russia. But it is also of tremendous potential benefit to us if we can nurture a sense of common purpose and trust. Al Gore and the Democratic Party will continue that effort. Similarly, we must continue to engage China - a nation with 1.3 billion people, a nuclear arsenal, and a role in the 21st Century that is destined to be one of the basic facts of international life. We must search out ways to cooperate across a broad range of issues, such as the environment and trade, while at the same time, insisting on adherence to international standards on human rights, freedom, the persecution of religions, the suppression of Tibet, and bellicose threats directed at Taiwan. China cannot be ignored, and these issues cannot - and must not - be marginalized. A deterioration of the U.S.-China relationship would harm, not help, American national security interests and the promotion of our values. A Gore Administration will fulfill its responsibilities under the Taiwan Relations Act. A Gore Administration will also remain committed to a \"One China\" policy. We support a resolution of cross-Straits issues that is both peaceful and consistent with the wishes of the people of Taiwan. ENHANCING EXISTING ALLIANCES The security and stability of Europe is critical to America's national security interests. We will continue to partner with the European Union to address global issues that could benefit from our combined capabilities. Under a Gore Administration, the U.S. will continue to work with our transatlantic allies to make the North Atlantic Treaty Organization (NATO) even stronger, thereby enhancing stability, promoting prosperity, and fostering democracy throughout Europe. The Democratic Party strongly supported the accession of Poland, the Czech Republic and Hungary as a milestone in building a stronger NATO and a more democratic and unified Europe. We look forward to bringing in additional qualified members in the future who share our values and are willing to take on the responsibilities of membership. A Gore Administration will ensure that the issue of NATO's future enlargement is part of the Alliance's agenda at the next summit in 2002 and that no non-NATO member has a veto over NATO decisions in this regard. We must strengthen our alliances and partnerships in Asia, with Japan and with South Korea. We must intensify our strategic cooperation with our ally Japan, building on our Joint Security Declaration, while finding more avenues to deal with Japan on a range of issues, from supporting democracy in Asia to promoting fair trade. And we remain committed to the defense of South Korea. The Democratic Party views our warm relationship with Australia as an anchor for our security interests in Southeast Asia, and we commend Australia for its leadership, and we applaud other nations for their participation with us in the peacekeeping operation in East Timor. We also are committed to enhancing our alliance with the countries of Latin America. We must build on the work that we began when we hosted the first Summit of the Americas, and we must accelerate implementation of the Plan of Action that will promote hemispheric cooperation on a full spectrum of political, economic, security and social issues. PREVENTING NEW PHYSICAL THREATS Preventing Proliferation. We must strengthen our defense against the proliferation of conventional and unconventional weapons that threaten America. Our first priority must be to continue the work we have begun in cutting stockpiles of weapons of mass destruction, halting testing, and ensuring that weapons and weapons-grade material do not fall into the wrong hands. Working with the government of the Russian Federation, we have helped safeguard nuclear material against the danger of theft. We have made it possible for thousands of Russia's nuclear scientists and weapons experts to find peaceful pursuits. And we have helped deactivate nearly 5,000 nuclear warheads. We are also equipping our military and continuously preparing our defenses for an unconventional attack. We have been an active player in international efforts to strengthen compliance with the Biological Weapons Convention. We renewed and made permanent the Non-Proliferation Treaty and ratified the Chemical Weapons Convention, but our effort to ratify the Comprehensive Test Ban Treaty was derailed by Senate Republicans. As President, Al Gore will promptly resubmit this treaty to the Senate with a demand from the American people for its ratification. Al Gore and the Democratic Party recognize the possibility of change in Iran, but we remain focused on the realities. Even as elements in Iran press for reform, the country still supports international terrorism, strives to acquire weapons of mass destruction, and represses its citizens, as evidenced by the immoral trial of 13 Jews in Shiraz. Ultimately, we must judge Iran by its actions. Al Gore will make an all-out effort to halt Iran's acquisition of weapons of mass destruction and delivery systems. In Iraq, we are committed to working with our international partners to keep Saddam Hussein boxed in, and we will work to see him out of power. Bill Clinton and Al Gore have stood up to Saddam Hussein time and time again. As President, Al Gore will not hesitate to use America's military might against Iraq when and where it is necessary. In light of the possibility that U.S. forces or our allies will have to contend with hostile tactical range ballistic missiles, we have been working rapidly to develop anti-tactical ballistic missile systems. We are working successfully with Israel on developing and deploying the Arrow anti-tactical ballistic missile system and the Tactical High Energy Laser. Our diplomacy has helped to halt North Korea's push for nuclear weapons. We got North Korea to stop testing long-range ballistic missiles and are also engaged in continuing negotiations regarding their testing and export of long-range ballistic missiles. The tight coordination between the United States, South Korea, and Japan is critical to our success, and we will maintain it as the two Koreas continue the dialogue began at the recent summit. We reject Republican plans to endanger our security with massive unilateral cuts in our arsenal and to construct an unproven, expensive, and ill-conceived missile defense system that would plunge us into a new arms race. Al Gore and the Democratic Party support the development of the technology for a limited national missile defense system that will be able to defend the U.S. against a missile attack from a state that has acquired weapons of mass destruction despite our efforts to block their proliferation. A decision to deploy such a system should be made based on four criteria: the nature of the threat, the feasability of the technology, the cost, and the overall impact on our national security, including arms control. The Democratic Party places a high value on ensuring that any such system is compatible with the Anti-Ballistic Missile Treaty. We also support continued work in significantly reducing strategic and other nuclear weapons, recognizing that the goal is strategic nuclear stability at progressively lower levels. Battling Terrorism. Whether terrorism is sponsored by a foreign nation or inspired by a single fanatic individual, such as Osama Bin Laden, Forward Engagement requires trying to disrupt terrorist networks, even before they are ready to attack. We must improve coordination internationally and domestically to share intelligence and develop operational plans. We must continue the comprehensive approach that has resulted in the development of a national counter-terrorism strategy involving all arms and levels of our government. We must continue to target terrorist finances, break up support cells, and disrupt training. And we must close avenues of cyber-attack by improving the security of the Internet and the computers upon which our digital economy exists. As President, Al Gore will tolerate no attack against American interests at home or abroad: terrorists must know that if they attack America, we will never forget. We will scour the world to hunt them down and bring them to justice. While fighting terrorism, we will protect the civil liberties of all Americans. Our justice system must guarantee fairness with procedures that protect the rights of the accused, even under the unusual circumstances of the investigation of threats to our national security. We must avoid stereotyping, for it defeats the highest purposes of our country if citizens feel automatically suspect by virtue of their ethnic origin. The purpose of terrorism is not only to intimidate, but also to divide and fracture, and we cannot permit that to happen. SEIZING OPPORTUNITIES Forward Engagement requires investment. But while international assistance and government aid are important - we should do more. There is no way to donate enough money to the parts of the world that are most deeply affected by war, lawlessness, disease, or disorder. What applies to us, applies to them: the only way for them to make real progress is to encourage investment by promoting growth that is sustainable and broadly shared. Latin America and the Caribbean must continue to be a focal point of our efforts. We believe that increased cooperation and trade with our partners in this hemisphere can reduce poverty and the reliance on the drug trade, and ultimately lead to economic development, stability, and prosperity. We have made great strides by helping avert a financial crisis in Mexico. Mexico's ongoing shift to a mature democracy, as demonstrated by her recent election, makes it increasingly possible for us to visualize even stronger relations and more effective relationships between ourselves, Mexico, and Canada, building on our growing economic ties to address environmental and social issues of common concern. A Gore Administration will build on this possibility in order to assure ourselves and the people of the Americas a future of democracy, prosperity, and security built on mutual trust and respect. At the same time, we should continue to safeguard environmental standards, food safety, and worker protections by refusing to allow cross-border trucking and bus operations until appropriate safety and worker fairness standards have been met. Prosperity and peace in Asia, the Middle East, and Africa will only be possible when those regions are fully integrated into the global economy. In Asia, we are working to promote fair trade with Japan and China. In the Middle East, we are promoting regional trade, particularly among Israel, Jordan, and Egypt. We must continue our work to reach out to moderate Arab states and we must intensify our effort to foster closer ties to the Islamic World. With respect to sub-Saharan Africa, the Democratic Party believes in supporting what South African President Thabo Mbeki has called \"an African renaissance.\" Notwithstanding this region's many problems, we see the example of South Africa as a great beacon of hope. We are encouraged by the restoration of democracy in Nigeria, the long-term continuation of a stable democratic system in Botswana, and Mozambique's courageous efforts of recovery after years of civil war. Even in the midst of her continuing problems, we see in Zimbabwe's recent election hope for the survival of the ideal of a multi-ethnic society. We regard the recently enacted African Growth and Opportunity Act as a major contribution toward the future. We believe that the United Nations can play an integral role in our policy of Forward Engagement. We understand that the institution needs both resources and reform if it is to play that role, and we pledge to take the lead on both fronts. Prosperity Abroad. Globalization must be a tide that lifts all boats, not a wave that overwhelms the most vulnerable among us. We support increasing our investment in the International Labor Organization and expanding the use of trade preferences that are tied to improvement in core labor standards. We also want to reverse the widening gap between rich and poor and nations, which is why Al Gore and the Democratic Party back debt forgiveness for the world's poorest nations. We must seek to reform international institutions such as the World Trade Organization, the International Monetary Fund, and the World Bank so that core labor standards, human rights, and protections of the environment are integral to their policies and practices. These institutions must also improve their transparency, accountability, and level of consultation with civil society so that citizens around the world can both understand the basis for their decisions and contribute to them. We should use our influence in multilateral development institutions to not only provide emergency assistance for stabilizing economies and to create social safety nets, including unemployment insurance and health care, but also to give people the skills, education, and training they need to compete in the New Economy. We must make a special effort to help women and children in societies that are devastated by war, disease and poverty. Women are traditionally the backbone of the family. We must also make a special effort to hear women when they rise up courageously to resist or end war in their communities. They are in a sense the front lines - the first affected - by the horrors of war and the misery of disease and poverty. We demand the United States Congress pass the Convention to Eliminate all forms of Discrimination Against Women which has been consistently blocked by the Republican Senate. And children represent the future. When we lose our children, we lose the promise of a future. Our investment programs must be more targeted toward women. And we must end the scourge of child labor by helping societies create educational opportunities for children and, more importantly, economic alternatives to employing the young. Promoting Democracy, Human Rights, Rule of Law, and Civil Society. American values and freedoms are a beacon unto nations, and we should use the power of our ideals to foster democracy, human rights, rule of law, and civil society throughout the world. The Democratic Party believes that America must continue to work closely with other nations, as well as non-governmental organizations to promote these goals. We aim to rededicate ourselves to the defense of democracy in the Americas at a moment when it is being brought into question in Peru and absent on the island of Cuba. We will continue to work with Haiti to deepen the roots of democracy that we helped replant. We will continue to press for human rights, the rule of law, and political freedom. We will continue to support the spread of democracy across Africa, Asia, and the Middle East and the development of judiciary, legal systems, media and civil society organizations. To accomplish this, we need the right tools. Al Gore and the Democratic Party support continued funding for the National Endowment for Democracy, Radio Liberty, Radio Free Europe, Radio Free Asia, Radio Marti, and other efforts to promote democracy and the free flow of ideas. We will build on our successful Reinventing Government program, led by Al Gore, to help other nations make their governments more responsive, more open, and more effective. We strongly support international educational exchanges. The students who come to America to study here - at the best academic institutions in the world - learn about our democratic values and institutions, our entrepreneurial skills, and our culture. They learn that Americans are noble dreamers remaining ever inclusive. * * * * * * Forty years ago, John F. Kennedy came to Los Angeles to accept the Democratic Party's nomination for president. In doing so, he pointed America towards new frontiers at home and abroad. In the year 2000, Al Gore comes to Los Angeles to accept that same nomination and renew our party's determination to accept big challenges and make bold choices. At the edge of a new century, Democrats stand united in our determination to offer prosperity to all who are willing to work for it, to provide progress to all who are willing to live by the values that have made America great, and to bring peace to all those willing to embrace democracy all over the world. For eight years, the Democratic Party's new thinking has helped America reach unparalleled heights of prosperity, progress, and peace. Now, we say that this is the time to move forward - not to go back. Now, we say that Democrats have just yet begun to fight for a better America and a brighter future. Now, we say to America, \"You ain't seen nothing yet.\""
\end{verbatim}

\begin{Shaded}
\begin{Highlighting}[]
\FunctionTok{table}\NormalTok{(}\FunctionTok{codes}\NormalTok{(corpus)) }\CommentTok{\# count codes of all manifestos}
\end{Highlighting}
\end{Shaded}

\begin{verbatim}
## 
##   000   101   102   103 103.1 103.2   104   105   106   107   108   109   110 
##   218   319    66     6     4     4   927   203   132   695     4   196     2 
##   201 201.1 201.2   202 202.1 202.2 202.3   203   204   301   302   303   304 
##   180   310   194   178   302     4    74   187    11   281     5   255    94 
##   305 305.1 305.2 305.3   401   402   403   404   405   406   407   408   409 
##   364   101    86     2   638   403   434    69    21    67    81   150     7 
##   410   411   412   413   414   415   416 416.2   501   502   503   504   505 
##   285   640    20     6   132     1    33    63   412    37   967   832   146 
##   506   507   601 601.1 601.2 602.1 602.2   603   604   605 605.1 605.2   606 
##   505    48   206   169    58    31   107   698    62   490   229   134    39 
## 606.1 606.2   607 607.1 607.2 607.3   608 608.1 608.2   701   702   703 703.1 
##    50     8    56    32    12   152    14     2     5   490    61    83   163 
## 703.2   704   705   706     H 
##     2    80    92   289   380
\end{verbatim}

What years, countries and parties are included in the dataset? How many
texts do you have for each of these?

\begin{quote}
\end{quote}

Prepare your data for topic modelling by creating a document feature
matrix. Describe the choices you make here, and comment on how these
might affect your final result.

\hypertarget{research-question}{%
\subsection{2. Research question}\label{research-question}}

Describe a research question you want to explore with topic modelling.
Comment on how answerable this is with the methods and data at your
disposal.

\begin{quote}
\end{quote}

\hypertarget{topic-model-development}{%
\subsection{3. Topic model development}\label{topic-model-development}}

Create a topic model using your data. Explain to a non-specialist what
the topic model does. Comment on the choices you make here in terms of
hyperparameter selection and model choice. How might these affect your
results and the ability to answer your research question?

\begin{quote}
\end{quote}

\hypertarget{topic-model-description}{%
\subsection{4. Topic model description}\label{topic-model-description}}

Describe the topic model. What topics does it contain? How are these
distributed across the data?

\begin{quote}
\end{quote}

\hypertarget{answering-your-research-question}{%
\subsection{5. Answering your research
question}\label{answering-your-research-question}}

Use your topic model to answer your research question by showing plots
or statistical results. Discuss the implications of what you find, and
any limitations inherent in your approach. Discuss how the work could be
improved upon in future research.

\begin{quote}
\end{quote}

\hypertarget{sources}{%
\section{Sources}\label{sources}}

\begin{itemize}
\tightlist
\item
  Lehmann, Pola / Franzmann, Simon / Burst, Tobias / Regel, Sven /
  Riethmüller, Felicia / Volkens, Andrea / Weßels, Bernhard / Zehnter,
  Lisa (2023): The Manifesto Data Collection. Manifesto Project
  (MRG/CMP/MARPOR). Version 2023a. Berlin: Wissenschaftszentrum Berlin
  für Sozialforschung (WZB) / Göttingen: Institut für
  Demokratieforschung (IfDem).
  \url{https://doi.org/10.25522/manifesto.mpds.2023a}
\end{itemize}

\hypertarget{resources}{%
\section{Resources}\label{resources}}

\begin{itemize}
\tightlist
\item
  \href{https://cran.r-project.org/web/packages/manifestoR/vignettes/manifestoRworkflow.pdf}{\texttt{manifestoR}
  vignette}
\end{itemize}

  \bibliography{../presentation-resources/MyLibrary.bib}

\end{document}
