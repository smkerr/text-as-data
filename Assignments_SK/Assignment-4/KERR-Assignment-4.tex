% Options for packages loaded elsewhere
\PassOptionsToPackage{unicode}{hyperref}
\PassOptionsToPackage{hyphens}{url}
\PassOptionsToPackage{dvipsnames,svgnames,x11names}{xcolor}
%
\documentclass[
]{article}
\usepackage{amsmath,amssymb}
\usepackage{iftex}
\ifPDFTeX
  \usepackage[T1]{fontenc}
  \usepackage[utf8]{inputenc}
  \usepackage{textcomp} % provide euro and other symbols
\else % if luatex or xetex
  \usepackage{unicode-math} % this also loads fontspec
  \defaultfontfeatures{Scale=MatchLowercase}
  \defaultfontfeatures[\rmfamily]{Ligatures=TeX,Scale=1}
\fi
\usepackage{lmodern}
\ifPDFTeX\else
  % xetex/luatex font selection
\fi
% Use upquote if available, for straight quotes in verbatim environments
\IfFileExists{upquote.sty}{\usepackage{upquote}}{}
\IfFileExists{microtype.sty}{% use microtype if available
  \usepackage[]{microtype}
  \UseMicrotypeSet[protrusion]{basicmath} % disable protrusion for tt fonts
}{}
\makeatletter
\@ifundefined{KOMAClassName}{% if non-KOMA class
  \IfFileExists{parskip.sty}{%
    \usepackage{parskip}
  }{% else
    \setlength{\parindent}{0pt}
    \setlength{\parskip}{6pt plus 2pt minus 1pt}}
}{% if KOMA class
  \KOMAoptions{parskip=half}}
\makeatother
\usepackage{xcolor}
\usepackage[margin=1in]{geometry}
\usepackage{color}
\usepackage{fancyvrb}
\newcommand{\VerbBar}{|}
\newcommand{\VERB}{\Verb[commandchars=\\\{\}]}
\DefineVerbatimEnvironment{Highlighting}{Verbatim}{commandchars=\\\{\}}
% Add ',fontsize=\small' for more characters per line
\usepackage{framed}
\definecolor{shadecolor}{RGB}{248,248,248}
\newenvironment{Shaded}{\begin{snugshade}}{\end{snugshade}}
\newcommand{\AlertTok}[1]{\textcolor[rgb]{0.94,0.16,0.16}{#1}}
\newcommand{\AnnotationTok}[1]{\textcolor[rgb]{0.56,0.35,0.01}{\textbf{\textit{#1}}}}
\newcommand{\AttributeTok}[1]{\textcolor[rgb]{0.13,0.29,0.53}{#1}}
\newcommand{\BaseNTok}[1]{\textcolor[rgb]{0.00,0.00,0.81}{#1}}
\newcommand{\BuiltInTok}[1]{#1}
\newcommand{\CharTok}[1]{\textcolor[rgb]{0.31,0.60,0.02}{#1}}
\newcommand{\CommentTok}[1]{\textcolor[rgb]{0.56,0.35,0.01}{\textit{#1}}}
\newcommand{\CommentVarTok}[1]{\textcolor[rgb]{0.56,0.35,0.01}{\textbf{\textit{#1}}}}
\newcommand{\ConstantTok}[1]{\textcolor[rgb]{0.56,0.35,0.01}{#1}}
\newcommand{\ControlFlowTok}[1]{\textcolor[rgb]{0.13,0.29,0.53}{\textbf{#1}}}
\newcommand{\DataTypeTok}[1]{\textcolor[rgb]{0.13,0.29,0.53}{#1}}
\newcommand{\DecValTok}[1]{\textcolor[rgb]{0.00,0.00,0.81}{#1}}
\newcommand{\DocumentationTok}[1]{\textcolor[rgb]{0.56,0.35,0.01}{\textbf{\textit{#1}}}}
\newcommand{\ErrorTok}[1]{\textcolor[rgb]{0.64,0.00,0.00}{\textbf{#1}}}
\newcommand{\ExtensionTok}[1]{#1}
\newcommand{\FloatTok}[1]{\textcolor[rgb]{0.00,0.00,0.81}{#1}}
\newcommand{\FunctionTok}[1]{\textcolor[rgb]{0.13,0.29,0.53}{\textbf{#1}}}
\newcommand{\ImportTok}[1]{#1}
\newcommand{\InformationTok}[1]{\textcolor[rgb]{0.56,0.35,0.01}{\textbf{\textit{#1}}}}
\newcommand{\KeywordTok}[1]{\textcolor[rgb]{0.13,0.29,0.53}{\textbf{#1}}}
\newcommand{\NormalTok}[1]{#1}
\newcommand{\OperatorTok}[1]{\textcolor[rgb]{0.81,0.36,0.00}{\textbf{#1}}}
\newcommand{\OtherTok}[1]{\textcolor[rgb]{0.56,0.35,0.01}{#1}}
\newcommand{\PreprocessorTok}[1]{\textcolor[rgb]{0.56,0.35,0.01}{\textit{#1}}}
\newcommand{\RegionMarkerTok}[1]{#1}
\newcommand{\SpecialCharTok}[1]{\textcolor[rgb]{0.81,0.36,0.00}{\textbf{#1}}}
\newcommand{\SpecialStringTok}[1]{\textcolor[rgb]{0.31,0.60,0.02}{#1}}
\newcommand{\StringTok}[1]{\textcolor[rgb]{0.31,0.60,0.02}{#1}}
\newcommand{\VariableTok}[1]{\textcolor[rgb]{0.00,0.00,0.00}{#1}}
\newcommand{\VerbatimStringTok}[1]{\textcolor[rgb]{0.31,0.60,0.02}{#1}}
\newcommand{\WarningTok}[1]{\textcolor[rgb]{0.56,0.35,0.01}{\textbf{\textit{#1}}}}
\usepackage{graphicx}
\makeatletter
\def\maxwidth{\ifdim\Gin@nat@width>\linewidth\linewidth\else\Gin@nat@width\fi}
\def\maxheight{\ifdim\Gin@nat@height>\textheight\textheight\else\Gin@nat@height\fi}
\makeatother
% Scale images if necessary, so that they will not overflow the page
% margins by default, and it is still possible to overwrite the defaults
% using explicit options in \includegraphics[width, height, ...]{}
\setkeys{Gin}{width=\maxwidth,height=\maxheight,keepaspectratio}
% Set default figure placement to htbp
\makeatletter
\def\fps@figure{htbp}
\makeatother
\setlength{\emergencystretch}{3em} % prevent overfull lines
\providecommand{\tightlist}{%
  \setlength{\itemsep}{0pt}\setlength{\parskip}{0pt}}
\setcounter{secnumdepth}{-\maxdimen} % remove section numbering
\usepackage{booktabs}
\usepackage{xcolor}
\ifLuaTeX
  \usepackage{selnolig}  % disable illegal ligatures
\fi
\usepackage[]{natbib}
\bibliographystyle{plainnat}
\IfFileExists{bookmark.sty}{\usepackage{bookmark}}{\usepackage{hyperref}}
\IfFileExists{xurl.sty}{\usepackage{xurl}}{} % add URL line breaks if available
\urlstyle{same}
\hypersetup{
  pdftitle={Assignment 4},
  pdfauthor={Steve Kerr (211924)},
  colorlinks=true,
  linkcolor={Maroon},
  filecolor={Maroon},
  citecolor={Blue},
  urlcolor={blue},
  pdfcreator={LaTeX via pandoc}}

\title{Assignment 4}
\author{Steve Kerr (211924)}
\date{2023-10-23}

\begin{document}
\maketitle

\hypertarget{introduction}{%
\section{Introduction}\label{introduction}}

In this assignment, you are asked to use topic modelling to investigate
manifestos from the manifesto project maintained by
\href{https://manifesto-project.wzb.eu/}{WZB}. You can either use the UK
manifestos we looked at together in class, or collect your own set of
manifestos by choosing the country/countries, year/years and
party/parties you are interested in. You should produce a report which
includes your code, that addresses the following aspects of creating a
topic model, making sure to answer the questions below.

This time, you will be assessed not only on whether the code gets the
right result, but on how you understand and communicate your
understanding of the modelling process and how this can answer your
research question. The best research question is one that is interesting
and answerable, but the most important thing is that the research
question is answerable with the methods you choose.

You will also be assessed on the presentation of your results, and on
the concision and readability of your code.

\begin{Shaded}
\begin{Highlighting}[]
\CommentTok{\# load packages}
\NormalTok{pacman}\SpecialCharTok{::}\FunctionTok{p\_load}\NormalTok{(}
\NormalTok{  here,}
\NormalTok{  manifestoR}
\NormalTok{  )}

\CommentTok{\# set API key}
\FunctionTok{mp\_setapikey}\NormalTok{(}\FunctionTok{here}\NormalTok{(}\StringTok{"Assignments\_SK/Assignment{-}4/manifesto\_apikey.txt"}\NormalTok{))}
\end{Highlighting}
\end{Shaded}

\hypertarget{data-acquisition-description-and-preparation}{%
\subsection{1. Data acquisition, description, and
preparation}\label{data-acquisition-description-and-preparation}}

Bring together a dataset from the WZB.

\begin{Shaded}
\begin{Highlighting}[]
\CommentTok{\# load data}
\DocumentationTok{\#\# download election programmes texts and codings}
\NormalTok{corpus }\OtherTok{\textless{}{-}} \FunctionTok{mp\_corpus}\NormalTok{(countryname }\SpecialCharTok{==} \StringTok{"United States"}\NormalTok{)}
\end{Highlighting}
\end{Shaded}

\begin{verbatim}
## Connecting to Manifesto Project DB API... 
## Connecting to Manifesto Project DB API... corpus version: 2023-1
\end{verbatim}

\begin{verbatim}
## Warning in formatmpds(data.frame(fromJSON(jsonstr), stringsAsFactors = FALSE)):
## NAs introduced by coercion

## Warning in formatmpds(data.frame(fromJSON(jsonstr), stringsAsFactors = FALSE)):
## NAs introduced by coercion

## Warning in formatmpds(data.frame(fromJSON(jsonstr), stringsAsFactors = FALSE)):
## NAs introduced by coercion

## Warning in formatmpds(data.frame(fromJSON(jsonstr), stringsAsFactors = FALSE)):
## NAs introduced by coercion

## Warning in formatmpds(data.frame(fromJSON(jsonstr), stringsAsFactors = FALSE)):
## NAs introduced by coercion

## Warning in formatmpds(data.frame(fromJSON(jsonstr), stringsAsFactors = FALSE)):
## NAs introduced by coercion

## Warning in formatmpds(data.frame(fromJSON(jsonstr), stringsAsFactors = FALSE)):
## NAs introduced by coercion

## Warning in formatmpds(data.frame(fromJSON(jsonstr), stringsAsFactors = FALSE)):
## NAs introduced by coercion

## Warning in formatmpds(data.frame(fromJSON(jsonstr), stringsAsFactors = FALSE)):
## NAs introduced by coercion

## Warning in formatmpds(data.frame(fromJSON(jsonstr), stringsAsFactors = FALSE)):
## NAs introduced by coercion

## Warning in formatmpds(data.frame(fromJSON(jsonstr), stringsAsFactors = FALSE)):
## NAs introduced by coercion

## Warning in formatmpds(data.frame(fromJSON(jsonstr), stringsAsFactors = FALSE)):
## NAs introduced by coercion

## Warning in formatmpds(data.frame(fromJSON(jsonstr), stringsAsFactors = FALSE)):
## NAs introduced by coercion

## Warning in formatmpds(data.frame(fromJSON(jsonstr), stringsAsFactors = FALSE)):
## NAs introduced by coercion

## Warning in formatmpds(data.frame(fromJSON(jsonstr), stringsAsFactors = FALSE)):
## NAs introduced by coercion

## Warning in formatmpds(data.frame(fromJSON(jsonstr), stringsAsFactors = FALSE)):
## NAs introduced by coercion

## Warning in formatmpds(data.frame(fromJSON(jsonstr), stringsAsFactors = FALSE)):
## NAs introduced by coercion

## Warning in formatmpds(data.frame(fromJSON(jsonstr), stringsAsFactors = FALSE)):
## NAs introduced by coercion

## Warning in formatmpds(data.frame(fromJSON(jsonstr), stringsAsFactors = FALSE)):
## NAs introduced by coercion

## Warning in formatmpds(data.frame(fromJSON(jsonstr), stringsAsFactors = FALSE)):
## NAs introduced by coercion

## Warning in formatmpds(data.frame(fromJSON(jsonstr), stringsAsFactors = FALSE)):
## NAs introduced by coercion

## Warning in formatmpds(data.frame(fromJSON(jsonstr), stringsAsFactors = FALSE)):
## NAs introduced by coercion

## Warning in formatmpds(data.frame(fromJSON(jsonstr), stringsAsFactors = FALSE)):
## NAs introduced by coercion

## Warning in formatmpds(data.frame(fromJSON(jsonstr), stringsAsFactors = FALSE)):
## NAs introduced by coercion

## Warning in formatmpds(data.frame(fromJSON(jsonstr), stringsAsFactors = FALSE)):
## NAs introduced by coercion

## Warning in formatmpds(data.frame(fromJSON(jsonstr), stringsAsFactors = FALSE)):
## NAs introduced by coercion

## Warning in formatmpds(data.frame(fromJSON(jsonstr), stringsAsFactors = FALSE)):
## NAs introduced by coercion

## Warning in formatmpds(data.frame(fromJSON(jsonstr), stringsAsFactors = FALSE)):
## NAs introduced by coercion

## Warning in formatmpds(data.frame(fromJSON(jsonstr), stringsAsFactors = FALSE)):
## NAs introduced by coercion

## Warning in formatmpds(data.frame(fromJSON(jsonstr), stringsAsFactors = FALSE)):
## NAs introduced by coercion

## Warning in formatmpds(data.frame(fromJSON(jsonstr), stringsAsFactors = FALSE)):
## NAs introduced by coercion

## Warning in formatmpds(data.frame(fromJSON(jsonstr), stringsAsFactors = FALSE)):
## NAs introduced by coercion

## Warning in formatmpds(data.frame(fromJSON(jsonstr), stringsAsFactors = FALSE)):
## NAs introduced by coercion

## Warning in formatmpds(data.frame(fromJSON(jsonstr), stringsAsFactors = FALSE)):
## NAs introduced by coercion

## Warning in formatmpds(data.frame(fromJSON(jsonstr), stringsAsFactors = FALSE)):
## NAs introduced by coercion

## Warning in formatmpds(data.frame(fromJSON(jsonstr), stringsAsFactors = FALSE)):
## NAs introduced by coercion

## Warning in formatmpds(data.frame(fromJSON(jsonstr), stringsAsFactors = FALSE)):
## NAs introduced by coercion

## Warning in formatmpds(data.frame(fromJSON(jsonstr), stringsAsFactors = FALSE)):
## NAs introduced by coercion

## Warning in formatmpds(data.frame(fromJSON(jsonstr), stringsAsFactors = FALSE)):
## NAs introduced by coercion

## Warning in formatmpds(data.frame(fromJSON(jsonstr), stringsAsFactors = FALSE)):
## NAs introduced by coercion

## Warning in formatmpds(data.frame(fromJSON(jsonstr), stringsAsFactors = FALSE)):
## NAs introduced by coercion

## Warning in formatmpds(data.frame(fromJSON(jsonstr), stringsAsFactors = FALSE)):
## NAs introduced by coercion

## Warning in formatmpds(data.frame(fromJSON(jsonstr), stringsAsFactors = FALSE)):
## NAs introduced by coercion

## Warning in formatmpds(data.frame(fromJSON(jsonstr), stringsAsFactors = FALSE)):
## NAs introduced by coercion

## Warning in formatmpds(data.frame(fromJSON(jsonstr), stringsAsFactors = FALSE)):
## NAs introduced by coercion

## Warning in formatmpds(data.frame(fromJSON(jsonstr), stringsAsFactors = FALSE)):
## NAs introduced by coercion

## Warning in formatmpds(data.frame(fromJSON(jsonstr), stringsAsFactors = FALSE)):
## NAs introduced by coercion

## Warning in formatmpds(data.frame(fromJSON(jsonstr), stringsAsFactors = FALSE)):
## NAs introduced by coercion

## Warning in formatmpds(data.frame(fromJSON(jsonstr), stringsAsFactors = FALSE)):
## NAs introduced by coercion

## Warning in formatmpds(data.frame(fromJSON(jsonstr), stringsAsFactors = FALSE)):
## NAs introduced by coercion

## Warning in formatmpds(data.frame(fromJSON(jsonstr), stringsAsFactors = FALSE)):
## NAs introduced by coercion

## Warning in formatmpds(data.frame(fromJSON(jsonstr), stringsAsFactors = FALSE)):
## NAs introduced by coercion

## Warning in formatmpds(data.frame(fromJSON(jsonstr), stringsAsFactors = FALSE)):
## NAs introduced by coercion

## Warning in formatmpds(data.frame(fromJSON(jsonstr), stringsAsFactors = FALSE)):
## NAs introduced by coercion

## Warning in formatmpds(data.frame(fromJSON(jsonstr), stringsAsFactors = FALSE)):
## NAs introduced by coercion

## Warning in formatmpds(data.frame(fromJSON(jsonstr), stringsAsFactors = FALSE)):
## NAs introduced by coercion

## Warning in formatmpds(data.frame(fromJSON(jsonstr), stringsAsFactors = FALSE)):
## NAs introduced by coercion

## Warning in formatmpds(data.frame(fromJSON(jsonstr), stringsAsFactors = FALSE)):
## NAs introduced by coercion

## Warning in formatmpds(data.frame(fromJSON(jsonstr), stringsAsFactors = FALSE)):
## NAs introduced by coercion

## Warning in formatmpds(data.frame(fromJSON(jsonstr), stringsAsFactors = FALSE)):
## NAs introduced by coercion

## Warning in formatmpds(data.frame(fromJSON(jsonstr), stringsAsFactors = FALSE)):
## NAs introduced by coercion

## Warning in formatmpds(data.frame(fromJSON(jsonstr), stringsAsFactors = FALSE)):
## NAs introduced by coercion

## Warning in formatmpds(data.frame(fromJSON(jsonstr), stringsAsFactors = FALSE)):
## NAs introduced by coercion

## Warning in formatmpds(data.frame(fromJSON(jsonstr), stringsAsFactors = FALSE)):
## NAs introduced by coercion

## Warning in formatmpds(data.frame(fromJSON(jsonstr), stringsAsFactors = FALSE)):
## NAs introduced by coercion

## Warning in formatmpds(data.frame(fromJSON(jsonstr), stringsAsFactors = FALSE)):
## NAs introduced by coercion

## Warning in formatmpds(data.frame(fromJSON(jsonstr), stringsAsFactors = FALSE)):
## NAs introduced by coercion

## Warning in formatmpds(data.frame(fromJSON(jsonstr), stringsAsFactors = FALSE)):
## NAs introduced by coercion

## Warning in formatmpds(data.frame(fromJSON(jsonstr), stringsAsFactors = FALSE)):
## NAs introduced by coercion

## Warning in formatmpds(data.frame(fromJSON(jsonstr), stringsAsFactors = FALSE)):
## NAs introduced by coercion

## Warning in formatmpds(data.frame(fromJSON(jsonstr), stringsAsFactors = FALSE)):
## NAs introduced by coercion

## Warning in formatmpds(data.frame(fromJSON(jsonstr), stringsAsFactors = FALSE)):
## NAs introduced by coercion

## Warning in formatmpds(data.frame(fromJSON(jsonstr), stringsAsFactors = FALSE)):
## NAs introduced by coercion

## Warning in formatmpds(data.frame(fromJSON(jsonstr), stringsAsFactors = FALSE)):
## NAs introduced by coercion

## Warning in formatmpds(data.frame(fromJSON(jsonstr), stringsAsFactors = FALSE)):
## NAs introduced by coercion

## Warning in formatmpds(data.frame(fromJSON(jsonstr), stringsAsFactors = FALSE)):
## NAs introduced by coercion

## Warning in formatmpds(data.frame(fromJSON(jsonstr), stringsAsFactors = FALSE)):
## NAs introduced by coercion

## Warning in formatmpds(data.frame(fromJSON(jsonstr), stringsAsFactors = FALSE)):
## NAs introduced by coercion

## Warning in formatmpds(data.frame(fromJSON(jsonstr), stringsAsFactors = FALSE)):
## NAs introduced by coercion

## Warning in formatmpds(data.frame(fromJSON(jsonstr), stringsAsFactors = FALSE)):
## NAs introduced by coercion

## Warning in formatmpds(data.frame(fromJSON(jsonstr), stringsAsFactors = FALSE)):
## NAs introduced by coercion

## Warning in formatmpds(data.frame(fromJSON(jsonstr), stringsAsFactors = FALSE)):
## NAs introduced by coercion

## Warning in formatmpds(data.frame(fromJSON(jsonstr), stringsAsFactors = FALSE)):
## NAs introduced by coercion

## Warning in formatmpds(data.frame(fromJSON(jsonstr), stringsAsFactors = FALSE)):
## NAs introduced by coercion

## Warning in formatmpds(data.frame(fromJSON(jsonstr), stringsAsFactors = FALSE)):
## NAs introduced by coercion

## Warning in formatmpds(data.frame(fromJSON(jsonstr), stringsAsFactors = FALSE)):
## NAs introduced by coercion

## Warning in formatmpds(data.frame(fromJSON(jsonstr), stringsAsFactors = FALSE)):
## NAs introduced by coercion

## Warning in formatmpds(data.frame(fromJSON(jsonstr), stringsAsFactors = FALSE)):
## NAs introduced by coercion

## Warning in formatmpds(data.frame(fromJSON(jsonstr), stringsAsFactors = FALSE)):
## NAs introduced by coercion

## Warning in formatmpds(data.frame(fromJSON(jsonstr), stringsAsFactors = FALSE)):
## NAs introduced by coercion

## Warning in formatmpds(data.frame(fromJSON(jsonstr), stringsAsFactors = FALSE)):
## NAs introduced by coercion

## Warning in formatmpds(data.frame(fromJSON(jsonstr), stringsAsFactors = FALSE)):
## NAs introduced by coercion

## Warning in formatmpds(data.frame(fromJSON(jsonstr), stringsAsFactors = FALSE)):
## NAs introduced by coercion

## Warning in formatmpds(data.frame(fromJSON(jsonstr), stringsAsFactors = FALSE)):
## NAs introduced by coercion

## Warning in formatmpds(data.frame(fromJSON(jsonstr), stringsAsFactors = FALSE)):
## NAs introduced by coercion

## Warning in formatmpds(data.frame(fromJSON(jsonstr), stringsAsFactors = FALSE)):
## NAs introduced by coercion

## Warning in formatmpds(data.frame(fromJSON(jsonstr), stringsAsFactors = FALSE)):
## NAs introduced by coercion

## Warning in formatmpds(data.frame(fromJSON(jsonstr), stringsAsFactors = FALSE)):
## NAs introduced by coercion

## Warning in formatmpds(data.frame(fromJSON(jsonstr), stringsAsFactors = FALSE)):
## NAs introduced by coercion

## Warning in formatmpds(data.frame(fromJSON(jsonstr), stringsAsFactors = FALSE)):
## NAs introduced by coercion

## Warning in formatmpds(data.frame(fromJSON(jsonstr), stringsAsFactors = FALSE)):
## NAs introduced by coercion

## Warning in formatmpds(data.frame(fromJSON(jsonstr), stringsAsFactors = FALSE)):
## NAs introduced by coercion

## Warning in formatmpds(data.frame(fromJSON(jsonstr), stringsAsFactors = FALSE)):
## NAs introduced by coercion

## Warning in formatmpds(data.frame(fromJSON(jsonstr), stringsAsFactors = FALSE)):
## NAs introduced by coercion

## Warning in formatmpds(data.frame(fromJSON(jsonstr), stringsAsFactors = FALSE)):
## NAs introduced by coercion

## Warning in formatmpds(data.frame(fromJSON(jsonstr), stringsAsFactors = FALSE)):
## NAs introduced by coercion

## Warning in formatmpds(data.frame(fromJSON(jsonstr), stringsAsFactors = FALSE)):
## NAs introduced by coercion

## Warning in formatmpds(data.frame(fromJSON(jsonstr), stringsAsFactors = FALSE)):
## NAs introduced by coercion

## Warning in formatmpds(data.frame(fromJSON(jsonstr), stringsAsFactors = FALSE)):
## NAs introduced by coercion

## Warning in formatmpds(data.frame(fromJSON(jsonstr), stringsAsFactors = FALSE)):
## NAs introduced by coercion

## Warning in formatmpds(data.frame(fromJSON(jsonstr), stringsAsFactors = FALSE)):
## NAs introduced by coercion

## Warning in formatmpds(data.frame(fromJSON(jsonstr), stringsAsFactors = FALSE)):
## NAs introduced by coercion

## Warning in formatmpds(data.frame(fromJSON(jsonstr), stringsAsFactors = FALSE)):
## NAs introduced by coercion

## Warning in formatmpds(data.frame(fromJSON(jsonstr), stringsAsFactors = FALSE)):
## NAs introduced by coercion

## Warning in formatmpds(data.frame(fromJSON(jsonstr), stringsAsFactors = FALSE)):
## NAs introduced by coercion

## Warning in formatmpds(data.frame(fromJSON(jsonstr), stringsAsFactors = FALSE)):
## NAs introduced by coercion

## Warning in formatmpds(data.frame(fromJSON(jsonstr), stringsAsFactors = FALSE)):
## NAs introduced by coercion

## Warning in formatmpds(data.frame(fromJSON(jsonstr), stringsAsFactors = FALSE)):
## NAs introduced by coercion

## Warning in formatmpds(data.frame(fromJSON(jsonstr), stringsAsFactors = FALSE)):
## NAs introduced by coercion

## Warning in formatmpds(data.frame(fromJSON(jsonstr), stringsAsFactors = FALSE)):
## NAs introduced by coercion

## Warning in formatmpds(data.frame(fromJSON(jsonstr), stringsAsFactors = FALSE)):
## NAs introduced by coercion

## Warning in formatmpds(data.frame(fromJSON(jsonstr), stringsAsFactors = FALSE)):
## NAs introduced by coercion

## Warning in formatmpds(data.frame(fromJSON(jsonstr), stringsAsFactors = FALSE)):
## NAs introduced by coercion

## Warning in formatmpds(data.frame(fromJSON(jsonstr), stringsAsFactors = FALSE)):
## NAs introduced by coercion

## Warning in formatmpds(data.frame(fromJSON(jsonstr), stringsAsFactors = FALSE)):
## NAs introduced by coercion

## Warning in formatmpds(data.frame(fromJSON(jsonstr), stringsAsFactors = FALSE)):
## NAs introduced by coercion

## Warning in formatmpds(data.frame(fromJSON(jsonstr), stringsAsFactors = FALSE)):
## NAs introduced by coercion

## Warning in formatmpds(data.frame(fromJSON(jsonstr), stringsAsFactors = FALSE)):
## NAs introduced by coercion

## Warning in formatmpds(data.frame(fromJSON(jsonstr), stringsAsFactors = FALSE)):
## NAs introduced by coercion

## Warning in formatmpds(data.frame(fromJSON(jsonstr), stringsAsFactors = FALSE)):
## NAs introduced by coercion

## Warning in formatmpds(data.frame(fromJSON(jsonstr), stringsAsFactors = FALSE)):
## NAs introduced by coercion

## Warning in formatmpds(data.frame(fromJSON(jsonstr), stringsAsFactors = FALSE)):
## NAs introduced by coercion

## Warning in formatmpds(data.frame(fromJSON(jsonstr), stringsAsFactors = FALSE)):
## NAs introduced by coercion

## Warning in formatmpds(data.frame(fromJSON(jsonstr), stringsAsFactors = FALSE)):
## NAs introduced by coercion

## Warning in formatmpds(data.frame(fromJSON(jsonstr), stringsAsFactors = FALSE)):
## NAs introduced by coercion

## Warning in formatmpds(data.frame(fromJSON(jsonstr), stringsAsFactors = FALSE)):
## NAs introduced by coercion

## Warning in formatmpds(data.frame(fromJSON(jsonstr), stringsAsFactors = FALSE)):
## NAs introduced by coercion

## Warning in formatmpds(data.frame(fromJSON(jsonstr), stringsAsFactors = FALSE)):
## NAs introduced by coercion

## Warning in formatmpds(data.frame(fromJSON(jsonstr), stringsAsFactors = FALSE)):
## NAs introduced by coercion

## Warning in formatmpds(data.frame(fromJSON(jsonstr), stringsAsFactors = FALSE)):
## NAs introduced by coercion

## Warning in formatmpds(data.frame(fromJSON(jsonstr), stringsAsFactors = FALSE)):
## NAs introduced by coercion

## Warning in formatmpds(data.frame(fromJSON(jsonstr), stringsAsFactors = FALSE)):
## NAs introduced by coercion

## Warning in formatmpds(data.frame(fromJSON(jsonstr), stringsAsFactors = FALSE)):
## NAs introduced by coercion

## Warning in formatmpds(data.frame(fromJSON(jsonstr), stringsAsFactors = FALSE)):
## NAs introduced by coercion

## Warning in formatmpds(data.frame(fromJSON(jsonstr), stringsAsFactors = FALSE)):
## NAs introduced by coercion

## Warning in formatmpds(data.frame(fromJSON(jsonstr), stringsAsFactors = FALSE)):
## NAs introduced by coercion

## Warning in formatmpds(data.frame(fromJSON(jsonstr), stringsAsFactors = FALSE)):
## NAs introduced by coercion

## Warning in formatmpds(data.frame(fromJSON(jsonstr), stringsAsFactors = FALSE)):
## NAs introduced by coercion

## Warning in formatmpds(data.frame(fromJSON(jsonstr), stringsAsFactors = FALSE)):
## NAs introduced by coercion

## Warning in formatmpds(data.frame(fromJSON(jsonstr), stringsAsFactors = FALSE)):
## NAs introduced by coercion

## Warning in formatmpds(data.frame(fromJSON(jsonstr), stringsAsFactors = FALSE)):
## NAs introduced by coercion

## Warning in formatmpds(data.frame(fromJSON(jsonstr), stringsAsFactors = FALSE)):
## NAs introduced by coercion

## Warning in formatmpds(data.frame(fromJSON(jsonstr), stringsAsFactors = FALSE)):
## NAs introduced by coercion

## Warning in formatmpds(data.frame(fromJSON(jsonstr), stringsAsFactors = FALSE)):
## NAs introduced by coercion

## Warning in formatmpds(data.frame(fromJSON(jsonstr), stringsAsFactors = FALSE)):
## NAs introduced by coercion

## Warning in formatmpds(data.frame(fromJSON(jsonstr), stringsAsFactors = FALSE)):
## NAs introduced by coercion

## Warning in formatmpds(data.frame(fromJSON(jsonstr), stringsAsFactors = FALSE)):
## NAs introduced by coercion

## Warning in formatmpds(data.frame(fromJSON(jsonstr), stringsAsFactors = FALSE)):
## NAs introduced by coercion
\end{verbatim}

\begin{verbatim}
## Connecting to Manifesto Project DB API... 
## Connecting to Manifesto Project DB API... corpus version: 2023-1 
## Connecting to Manifesto Project DB API... corpus version: 2023-1 
## Connecting to Manifesto Project DB API... corpus version: 2023-1
\end{verbatim}

\begin{Shaded}
\begin{Highlighting}[]
\NormalTok{corpus}
\end{Highlighting}
\end{Shaded}

\begin{verbatim}
## <<ManifestoCorpus>>
## Metadata:  corpus specific: 0, document level (indexed): 0
## Content:  documents: 32
\end{verbatim}

\begin{Shaded}
\begin{Highlighting}[]
\CommentTok{\# explore data}
\FunctionTok{head}\NormalTok{(}\FunctionTok{content}\NormalTok{(corpus[[}\DecValTok{1}\NormalTok{]])) }\DocumentationTok{\#\# view beginning of text of first manifesto}
\end{Highlighting}
\end{Shaded}

\begin{verbatim}
## [1] "In 1796, in America's first contested national election, our Party, under the leadership of Thomas Jefferson, campaigned on the principles of \"The Rights of Man\". Ever since, these four words have underscored our identity with the plain people of America and the world. In periods of national crises we Democrats have returned to these words for renewed strength. We return to them today. In 1960, \"The Rights of Man\" are still the issue. It is our continuing responsibility to provide an effective instrument of political action for every American who seeks to strengthen these rights everywhere here in America, and everywhere in our 20th Century world. The common danger of mankind is war and the threat of war. Today, three billion human beings live in fear that some rash act or blunder may plunge us all into a nuclear holocaust which will leave only ruined cities, blasted homes, and a poisoned earth and sky. Our objective, however, is not the right to coexist in armed camps on the same planet with totalitarian ideologies it is the creation of an enduring peace in which the universal values of human dignity, truth, and justice under law are finally secured for all men everywhere on earth. If America is to work effectively for such a peace, we must first restore our national strength--military, political, economic, and moral. NATIONAL DEFENSE The new Democratic Administration will recast our military capacity in order to provide forces and weapons of a diversity, balance, and mobility sufficient in quantity and quality to deter both limited and general aggressions. When the Democratic Administration left office in 1953 the United States was the pre-eminent power in the world. Most free nations had confidence in our will and our ability to carry out our commitments to the common defense. Even those who wished us ill respected our power and influence. The Republican Administration has lost that position of pre eminence. Over the past seven and one-half years, our military power has steadily declined relative to that of the Russians and the Chinese and their satellites. This is not a partisan election-year charge. it has been persistently made by high officials of the Republican Administration itself. Before Congressional committees they have testified that the Communists will have a dangerous lead in intercontinental missiles through 1963 and that the Republican Administration has no plans to catch up. They have admitted that the Soviet Union leads in the space race and that they have no plans to catch up. They have also admitted that our conventional military forces, on which we depend for defense in any non-nuclear war, have been dangerously slashed for reasons of \"economy\" and that they have no plans to reverse this trend. As a result, our military position today is measured in terms of gaps, missile gap, space gap, limited-war gap. To recover from the errors of the past seven and one-half years will not be easy. This is the strength that must be erected: 1. Deterrent military power such that the Soviet and Chinese leaders will have no doubt that an attack on the United States would surely be followed by their own destruction. 2. Balanced conventional military forces which will permit a response graded to the intensity of any threats of aggressive force. 3. Continuous modernization of these forces through intensified research and development, including essential programs now slowed down, terminated, suspended, or neglected for lack of budgetary support. A first order of business of a Democratic Administration will be a complete re-examination of the organization of our armed forces. A military organization structure, conceived before the revolution in weapons technology, cannot be suitable for the strategic deterrent, continental defense, limited war, and military alliance requirements of the nineteen sixties. We believe that our armed forces should be organized more nearly on the basis of function, not only to produce greater military strength, but also to eliminate duplication and save substantial sums. We pledge our will, energies and resources to oppose Communist aggression. Since World War II it has been clear that our own security must be pursued in concert with that of many other nations. The Democratic Administrations which in World War II led in forging a mighty and victorious alliance, took the initiative after the war in creating the North Atlantic Treaty Organization-- the greatest peacetime alliance in history. This alliance has made it possible to keep Western Europe and the Atlantic Community secure against Communist pressures. Our present system of alliances was begun in a time of an earlier weapons technology when our ability to retaliate against Communist attack required bases all around the periphery of the Soviet Union. Today, because of our continuing weakness in mobile weapons systems and intercontinental missiles, our defenses still depend in part on bases beyond our borders for planes and shorter range missiles. If an alliance is to be maintained in vigor, its unity must be reflected in shared purposes. Some of our allies have contributed neither devotion to the cause of freedom nor any real military strength. The new Democratic Administration will review our system of pacts and alliances. We shall continue to adhere to our treaty obligations, including the commitment of the UN Charter to resist aggression. But we shall also seek to shift the emphasis of our cooperation from military aid to economic development, wherever this is possible. Civil Defense We commend the work of the civil defense groups throughout the nation. A strong and effective civil defense is an essential element in our nation's defense. The new Democratic Administration will undertake a full review and analysis of the programs that should be adopted if the protection possible is to be provided to the civilian population of our nation. ARMS CONTROL A fragile power balance sustained by mutual nuclear terror does not, however, constitute peace. We must regain the initiative on the entire international front with effective new policies to create the conditions for peace. There are no simple solutions to the infinitely complex challenges which face us. Mankind's eternal dream, a world of peace, can only be built slowly and patiently. A primary task is to develop responsible proposals that will help break the deadlock on arms control. Such proposals should include means for ending nuclear tests under workable safeguards, cutting back nuclear weapons, reducing conventional forces, preserving outer space for peaceful purposes, preventing surprise attack, and limiting the risk of accidental war. This requires a national peace agency for disarmament planning and research to muster the scientific ingenuity, coordination, continuity, and seriousness of purpose which are now lacking in our arms control efforts. The national peace agency would develop the technical and scientific data necessary for serious disarmament negotiations, would conduct research in cooperation with the Defense Department and Atomic Energy Commission on methods of inspection and monitoring arms control agreements particularly agreements to control nuclear testing, and would provide continuous technical advice to our disarmament negotiators. As with armaments, so with disarmaments, the Republican Administration has provided us with much talk but little constructive action. Representatives of the United States have gone to conferences without plans or preparation. The Administration has played opportunistic politics, both at home and abroad. Even during the recent important negotiations at Geneva and Paris, only a handful of people were devoting full time to work on the highly complex problem of disarmament. More than 100 billion dollars of the world's production now goes each year into armaments. To the extent that we can secure the adoption of effective arms control agreements, vast resources will be freed for peaceful use. The new Democratic Administration will plan for an orderly shift of our expenditures. Long-delayed reductions in excise, corporation, and individual income taxes will then be possible. We can also step-up the pace in meeting our backlog of public needs and in pursuing the promise of atomic and space science in a peaceful age. As world-wide disarmament proceeds, it will free vast resources for a new international attack on the problem of world poverty. THE INSTRUMENTS OF FOREIGN POLICY American foreign policy in all its aspects must be attuned to our world of change. We will recruit officials whose experience, humanity, and dedication fit them for the task of effectively representing America abroad. We will provide a more sensitive and creative direction to our overseas information program. And we will overhaul our administrative machinery so that America may avoid diplomatic embarrassments and at long last speak with a single confident voice in world affairs. The \"Image\" of America First, those men and women selected to represent us abroad must be chosen for their sensitive understanding of the peoples with whom they will live. We can no longer afford representatives who are ignorant of the language and culture and politics of the nations in which they represent us. Our information programs must be more than news broadcasts and boastful recitals of our accomplishments and our material riches. We must find ways to show the people of the world that we share the same goals dignity health freedom schools for children a place in the sun and that we will work together to achieve them. Our program of visits between Americans and people of other nations will be expanded, with special emphasis upon students and younger leaders. We will encourage study of foreign languages. We favor continued support and extension of such programs as the East-West cultural center established at the University of Hawaii. We shall study a similar center for Latin America with due consideration of the existing facilities now available in the Canal Zone. National Policy Machinery In the present administration, the National Security Council has been used not to focus issues for decision by the responsible leaders of Government, but to paper over problems of policy with \"agreed solutions\" which avoid decisions. The mishandling of the U-2 espionage flights, the sorry spectacle of official denial, retraction, and contradiction and the admitted misjudging of japanese public opinion are only two recent examples of the breakdown of the Administration's machinery for assembling facts, making decisions, and coordinating action. The Democratic Party welcomes the study now being made by the Senate Subcommittee on National Policy Machinery. The new Democratic Administration will revamp and simplify this cumbersome machinery. WORLD TRADE World trade is more than ever essential to world peace. In the tradition of Cordell Hull, we shall expand world trade in every responsible way. Since all Americans share the benefits of this policy, its costs should not be the burden of a few. We shall support practical measures to ease the necessary adjustments of industries and communities which may be unavoidably hurt by increases in imports. World trade raises living standards, widens markets, reduces costs, increases profits, and builds political stability and international economic cooperation. However, the increase in foreign imports involves costly adjustment and damage to some domestic industries and communities. The burden has been heavier recently because of the Republican failure to maintain an adequate rate of economic growth, and the refusal to use public programs to ease necessary adjustments. The Democratic Administration will help industries affected by foreign trade with measures favorable to economic growth, orderly transition, fair competition, and the long-run economic strength of all parts of our nation. Industries and communities affected by foreign trade need and deserve appropriate help through trade adjustment measures such as direct loans, tax incentives, defense contracts priority, and retraining assistance. Our Government should press for reduction of foreign barriers to the sale of the products of American industry and agriculture. These are particularly severe in the case of fruit products. The present balance-of-payments situation provides a favorable opportunity for such action. The new Democratic Administration will seek international agreements to assure fair competition and fair labor standards to protect our own workers and to improve the lot of workers elsewhere. Our domestic economic policies and our essential foreign policies must be harmonious. To sell, we must buy. We therefore must resist the temptation to accept remedies that deny American producers and consumers access to world markets and destroy the prosperity of our friends in the non-Communist world. IMMIGRATION We shall adjust our immigration, nationality and refugee policies to eliminate discrimination and to enable members of scattered families abroad to be united with relatives already in our midst. The national-origins quota system of limiting immigration contradicts the founding principles of this nation. It is inconsistent with our belief in the rights of man. This system was instituted after World War I as a policy of deliberate discrimination by a Republican Administration and Congress. The revision of immigration and nationality laws we seek will implement our belief that enlightened immigration, naturalization and refugee policies and humane administration of them are important aspects of our foreign policy. These laws will bring greater skills to our land reunite families permit the United States to meet its fair share of world programs of rescue and rehabilitation, and take advantage of immigration as an important factor in the growth of the American economy. In this World Refugee Year it is our hope to achieve admission of our fair share of refugees. We will institute policies to alleviate suffering among the homeless wherever we are able to extend our aid. We must remove the distinctions between native born and naturalized citizens to assure full protection of our laws to all. There is no place in the United States for \"second-class citizenship\". The protections provided by due process, right of appeal, and statutes of limitation, can be extended to non-citizens without hampering the security of our nation. We commend the Democratic Congress for the initial steps that have recently been taken toward liberalizing changes in immigration law. However, this should not be a piecemeal project and we are confident that a Democratic President in cooperation with Democratic Congresses will again implant a humanitarian and liberal spirit in our nation's immigration and citizenship policies. To the peoples and governments beyond our shores we offer the following pledges: THE UNDERDEVELOPED WORLD To the non-Communist nations of Asia, Africa, and Latin America: We shall create with you working partnerships based on mutual respect and understanding. In the Jeffersonian tradition, we recognize and welcome the irresistible momentum of the world revolution of rising expectations for a better life. We shall identify American policy with the values and objectives of this revolution. To this end the new Democratic Administration will revamp and refocus the objectives, emphasis and allocation of our foreign assistance programs. The proper purpose of these programs is not to buy gratitude or to recruit mercenaries, but to enable the peoples of these awakening, developing nations to make their own free choices. As they achieve a sense of belonging, of dignity, and of justice, freedom will become meaningful for them, and therefore worth defending. Where military assistance remains essential for the common defense, we shall see that the requirements are fully met. But as rapidly as security considerations permit, we will replace tanks with tractors, bombers with bulldozers, and tacticians with technicians. We shall place our programs of international cooperation on a long-term basis to permit more effective planning. We shall seek to associate other capital-exporting countries with us in promoting the orderly economic growth of the underdeveloped world. We recognize India and Pakistan as major tests of the capacity of free men in a difficult environment to master the age old problems of illiteracy, poverty, and disease. We will support their efforts in every practical way. We welcome the emerging new nations of Africa to the world community. Here again we shall strive to write a new chapter of fruitful cooperation. In Latin America we shall restore the Good Neighbor Policy based on far closer economic cooperation and increased respect and understanding. In the Middle East we will work for guarantees to insure independence for all states. We will encourage direct Arab Israeli peace negotiations, the resettlement of Arab refugees in lands where there is room and opportunity for them, an end to boycotts and blockades, and unrestricted use of the Suez Canal by all nations. A billion and a half people in Asia, Africa, and Latin America are engaged in an unprecedented attempt to propel themselves into the 20th Century. They are striving to create or reaffirm their national identity. But they want much more than independence. They want an end to grinding poverty. They want more food, health for themselves and their children, and other benefits that a modern industrial civilization can provide. Communist strategy has sought to divert these aspirations into narrowly nationalistic channels, or external troublemaking, or authoritarianism. The Republican Administration has played into the hands of this strategy by concerning itself almost exclusively with the military problem of Communist invasion. The Democratic programs of economic cooperation will be aimed at making it as easy as possible for the political leadership in these countries to turn the energy, talent and resources of their peoples to orderly economic growth. History and current experience show that an annual per capita growth rate of at least 2 percent is feasible in these countries. The Democratic Administration's assistance program, in concert with the aid forthcoming from our partners in Western Europe, Japan, and the British Commonwealth, will be geared to facilitating this objective. The Democratic Administration will recognize that assistance to these countries is not an emergency or short-term matter. Through the Development Loan Fund and otherwise, we shall seek to assure continuity in our aid programs for periods of at least five years, in order to permit more effective allocation on our part and better planning by the governments of the countries receiving aid. More effective use of aid and a greater confidence in us and our motives will be the result. We shall establish priorities for foreign aid which will channel it to those countries abroad which, by their own willingness to help themselves, show themselves most capable of using it effectively. We shall use our own agricultural productivity as an effective tool of foreign aid, and also as a vital form of working capital for economic development. We shall seek new approaches which will provide assistance without disrupting normal world markets for food and fiber. We shall give attention to the problem of stabilizing world prices of agricultural commodities and basic raw materials on which many underdeveloped countries depend for needed foreign exchange. We shall explore the feasibility of shipping and storing a substantial part of our food abundance in a system of \"food banks\" located at distribution centers in the underdeveloped world. Such a system would be an effective means of alleviating famine and suffering in times of natural disaster, and of cushioning the effect of bad harvests. It would also have a helpful anti-inflationary influence as economic development gets under way. Although basic development requirements like transport, housing, schools, and river development may be financed by Government, these projects are usually built and sometimes managed by private enterprise. Moreover, outside this public sector a large and increasing role remains for private investment. The Republican Administration has done little to summon American business to play its part in this, one of the most creative tasks of our generation. The Democratic Administration will take steps to recruit and organize effectively the best business talent in America for foreign economic development. We urge continued economic assistance to Israel and the Arab peoples to help them raise their living standards. We pledge our best efforts for peace in the Middle East by seeking to prevent an arms race while guarding against the dangers of a military imbalance resulting from Soviet arms shipments. THE ATLANTIC COMMUNITY To our friends and associates in the Atlantic Community: We propose a broader partnership that goes beyond our common fears to recognize the depth and sweep of our common political, economic, and cultural interests. We welcome the recent heartening advances toward European unity. In every appropriate way, we shall encourage their further growth within the broader framework of the Atlantic Community. After World War II Democratic statesmen saw that an orderly, peaceful world was impossible with Europe shattered and exhausted. They fashioned the great programs which bear their names, the Truman Doctrine and the Marshall Plan by which the economies of Europe were revived. Then in NATO they renewed for the common defense the ties of alliance forged in war. In these endeavors, the Democratic Administrations invited leading Republicans to full participation as equal partners. But the Republican Administration has rejected this principle of bi partisanship. We have already seen how the mutual trust and confidence created abroad under Democratic leadership have been eroded by arrogance, clumsiness, and lack of understanding in the Republican Administration. The new Democratic Administration will restore the former high levels of cooperation within the Atlantic Community envisaged from the beginning by the NATO treaty in political and economic spheres as well as military affairs. We welcome the progress towards European unity expressed in the Coal And Steel Community Euratom the European Economic Community the European Free Trade Association and the European Assembly. We shall conduct our relations with the nations of the Common Market so as to encourage the opportunities for freer and more expanded trade, and to avert the possibilities of discrimination that are inherent in it. We shall encourage adjustment with the so-called \"Outer Seven\" nations so as to enlarge further the area of freer trade. THE COMMUNIST WORLD To the rulers of the communist world: We confidently accept your challenge to competition in every field of human effort. We recognize this contest as one between two radically different approaches to the meaning of life, our open society which places its highest value upon individual dignity, and your closed society in which the rights of men are sacrificed to the state. We believe your Communist ideology to be sterile, unsound, and doomed to failure. We believe that your children will reject the intellectual prison in which you seek to confine them, and that ultimately they will choose the eternal principles of freedom. In the meantime, we are prepared to negotiate with you whenever and wherever there is a realistic possibility of progress without sacrifice of principle. If negotiations through diplomatic channels provide opportunities we will negotiate. If debate before the United Nations holds promise, we will debate. If meetings at high level offer prospects of success, we will be there. But we will use all the power, resources, and energy at our command to resist the further encroachment of Communism on freedom, whether at Berlin, Formosa, or new points of pressure as yet undisclosed. We will keep open the lines of communication with our opponents. Despite difficulties in the way of peaceful agreements, every useful avenue will be energetically explored and pursued. However, we will never surrender positions which are essential to the defense of freedom nor will we abandon peoples who are now behind the Iron Curtain through any formal approval of the status quo. Everyone proclaims \"firmness\" in support of Berlin. The issue is not the desire to be firm, but the capability to be firm. This the Democratic Party will provide as it has done before. The ultimate solution of the situation in berlin must be approached in the broader context of settlement of the tensions and divisions of Europe. The good faith of the United States is pledged likewise to defending Formosa. We will carry out that pledge. The new Democratic Administration will also reaffirm our historic policy of opposition to the establishment anywhere in the Americas of governments dominated by foreign powers, a policy now being undermined by Soviet threats to the freedom and independence of Cuba. The Government of the United States under a Democratic Administration will not be deterred from fulfilling its obligations and solemn responsibilities under its treaties and agreements with the nations of the Western Hemisphere. Nor will the United States in conformity with its treaty obligations permit the establishment of a regime dominated by international, atheistic Communism in the Western Hemisphere. To the people who live in the Communist world and its captive nations: We proclaim an enduring friendship which goes beyond governments and ideologies to our common human interest in a better world. Through exchanges of persons, cultural contacts, trade in non-strategic areas, and other non-governmental activities, we will endeavor to preserve and improve opportunities for human relationships which no Iron Curtain can permanently sever. No political platform promise in history was more cruelly cynical than the Republican effort to buy votes in 1952 with false promises of painless liberation for the captive nations. The blood of heroic freedom fighters in Hungary tragically proved this promise a fraud. We Democrats will never be party to such cruel cultivation of false hopes. We look forward to the day when the men and women of Albania, Bulgaria, Czechoslovakia, East Germany, Estonia, Hungary, Latvia, Lithuania, Poland, Rumania, and the other captive nations will stand again in freedom and justice. We will hasten, by every honorable and responsible means, the arrival of the day. We shall never accept any deal or arrangement which acquiesces in the present subjugation of these peoples. We deeply regret that the policies and actions of the Government of Communist China have interrupted the generations of friendship between the Chinese and American peoples. We reaffirm our pledge of determined opposition to the present admission of Communist China to the United Nations. Although normal diplomatic relations between our Governments are impossible under present conditions, we shall welcome any evidence that the Chinese Communist Government is genuinely prepared to create a new relationship based on respect for international obligations, including the release of american prisoners. We will continue to make every effort to effect the release of American citizens and servicemen now injustly imprisoned in Communist China and elsewhere in the Communist empire. THE UNITED NATIONS To all our fellow members of the United Nations: We shall strengthen our commitments in this, our great continuing institution for conciliation and the growth of a world community. Through the machinery of the United Nations we shall work for disarmament the establishment of an international police force, the strengthening of the World Court, and the establishment of world law. We shall propose the bolder and more effective use of the specialized agencies to promote the world's economic and social development. Great Democratic Presidents have taken the lead in the effort to unite the nations of the world in an international organization to assure world peace with justice under law. The League of Nations conceived by Woodrow Wilson was doomed by Repubican defeat of United States participation. The United Nations sponsored by Franklin Roosevelt, has become the one place where representatives of the rival systems and interests which divide the world can and do maintain continuous contact. The United States adherence to the World Court contains a so called \"self-judging reservation\" which, in effect, permits us to prevent a Court decision in any particular case in which we are involved. The Democratic Party proposes its repeal. To all these endeavors so essential to world peace, we, the members of the Democratic Party, will bring a new urgency, persistence, and determination, born of the conviction that in our thermonuclear century all of the other Rights of Man hinge on our ability to assure man's right to peace. The pursuit of peace, our contribution to the stability of the new nations of the world, our hopes for progress and well being at home, all these depend in large measure on our ability to release the full potential of our American economy for employment, production, and growth. Our generation of Americans has achieved a historic technological breakthrough. Today we are capable of creating an abundance in goods and services beyond the dreams of our parents. Yet on the threshold of plenty the Republican Administration hesitates, confused and afraid. As a result, massive human needs now exist side by side with idle workers, idle capital, and idle machines. The Republican failure in the economic field has been virtually complete. Their years of power have consisted of two recessions, in 1953-1954 and 1957-1960, separated by the most severe peacetime inflation in history. They have shown themselves incapable of checking inflation. In their efforts to do so, they have brought on recessions that have thrown millions of Americans out of work. Yet even in these slumps, the cost of living has continued to climb, and it is now at an all-time high. They have slowed down the rate of growth of the economy to about one-third the rate of the Soviet Union. Over the past 7 and one-half year period, the Republicans have failed to balance the budget or reduce the national debt. Responsible fiscal policy requires surpluses in good times to more than offset the deficits which may occur in recessions, in order to reduce the national debt over the long run. The Republican Administration has produced the deficits, in fact, the greatest deficit in any peacetime year in history, in 1958-1959, but only occasional and meager surpluses. Their first seven years produced a total deficit of nearly 19 billion dollars. While reducing outlays for essential public services which directly benefit our people, they have raised the annual interest charge on the national debt to a level 3 billion dollars higher than when they took office. In the eight fiscal years of the Republican Administration, these useless higher interest payments will have cost the taxpayer 9 billion dollars. They have mismanaged the public debt not only by increasing interest rates, but also by failing to lengthen the average maturity of Government obligations when they had a clear opportunity to do so. ECONOMIC GROWTH The new Democratic Administration will confidently proceed to unshackle American enterprise and to free American labor, industrial leadership, and capital, to create an abundance that will outstrip any other system. Free competitive enterprise is the most creative and productive form of economic order that the world has seen. The recent slow pace of American growth is due not to the failure of our free economy but to the failure of our national leadership. We Democrats believe that our economy can and must grow at an average rate of 5 percent annually, almost twice as fast as our average annual rate since 1953. We pledge ourselves to policies that will achieve this goal without inflation. Economic growth is the means whereby we improve the American standard of living and produce added tax resources for national security and essential public services. Our economy must grow more swiftly in order to absorb two groups of workers, the much larger number of young people who will be reaching working age in the 1960s, and the workers displaced by the rapid pace of technological advances, including automation. Republican policies which have stifled growth could only mean increasingly severe unemployment, particularly of youth and older workers. AN END TO TIGHT MONEY As the first step in speeding economic growth, a Democratic President will put an end to the present high interest, tight money policy. This policy has failed in its stated purpose to keep prices down. It has given us two recessions within five years, bankrupted many of our farmers, produced a record number of business failures, and added billions of dollars in unnecessary higher interest charges to Government budgets and the cost of living. A new Democratic Administration will reject this philosophy of economic slowdown. We are committed to maximum employment, at decent wages and with fair profits, in a far more productive, expanding economy. The Repblican high-interest policy has extracted a costly toll from every American who has financed a home, an automobile, a refrigerator, or a television set. It has foisted added burdens on taxpayers of state and local governments which must borrow for schools and other public services. It has added to the cost of many goods and services, and hence has been itself a factor in inflation. It has created windfalls for many financial institutions. The nine billion dollars of added interest charges on the national debt would have been even higher but for the prudent insistence of the Democratic Congress that the ceiling on interest rates for long-term Government bonds be maintained. CONTROL OF INFLATION The American consumer has a right to fair prices. We are determined to secure that right. Inflation has its roots in a variety of causes, its cure lies in a variety of remedies. Among those remedies are monetary and credit policies properly applied, budget surpluses in times of full employment, and action to restrain \"administered price\" increases in industries where economic power rests in the hands of a few. A fair share of the gains from increasing productivity in many industries should be passed on to the consumer through price reductions. The agenda which a new Democratic Administration will face next January is crowded with urgent needs on which action has been delayed, deferred, or denied by the present Administration. A new Democratic Administration will undertake to meet those needs. It will reaffirm the Economic Bill of Rights which Franklin Roosevelt wrote into our national conscience sixteen years ago. It will reaffirm these rights for all Americans of whatever race, place of residence, or station in life, 1. \"The right to a useful and remunerative job in the industries or shops or farms or mines of the nation.\" FULL EMPLOYMENT The Democratic Party reaffirms its support of full employment as a paramount objective of national policy. For nearly 30 months the rate of unemployment has been between 5 and 7 and one-half percent of the labor force. A pool of three to four million citizens, able and willing to work but unable to find jobs, has been written off by the Republican Administration as a \"normal\" readjustment of the economic system. The policies of a Democratic Administration to restore economic growth will reduce current unemployment to a minimum. Thereafter, if recessionary trends appear, we will act promptly with counter measures, such as public works or temporary tax cuts. We will not stand idly by and permit recessions to run their course as the Republican Administration has done. AID TO DEPRESSED AREAS The right to a job requires action to create new industry in America's depressed areas of chronic unemployment. General economic measures will not alone solve the problem of localities which suffer some special disadvantage. To bring prosperity to these depressed areas and to enable them to make their full contribution to the national welfare, specially directed action is needed. Areas of heavy and persistent unemployment result from depletion of natural resources, technological change, shifting defense requirements, or trade imbalances which have caused the decline of major industries. Whole communities, urban and rural, have been left stranded in distress and despair, through no fault of their own. These communities have undertaken valiant efforts of self help. But mutual aid, as well as self-help, is part of the American tradition. Stricken communities deserve the help of the whole nation. The Democratic Congress twice passed bills to provide this help. The Republican President twice vetoed them. These bills proposed low-interest loans to private enterprise to create new industry and new jobs in depressed communities, assistance to the communities to provide public facilities necessary to encourage the new industry, and retraining of workers for the new jobs. The Democratic Congress will again pass, and the Democratic President will sign, such a bill. DISCRIMINATION IN EMPLOYMENT The right to a job requires action to break down artificial and arbitrary barriers to employment based on age race sex religion or national origin. Unemployment strikes hardest at workers over 40, minority groups, young people, and women. We will not achieve full employment until prejudice against these workers is wiped out. COLLECTIVE BARGAINING The right to a job requires the restoration of full support for collective-bargaining and the repeal of the anti-labor excesses which have been written into our labor laws. Under Democratic leadership a sound national policy was developed, expressed particularly by the Wagner National Labor Relations Act, which guaranteed the rights of workers to organize and to bargain collectively. But the Republican Administration has replaced this sound policy with a national anti-labor policy.  The Republican Taft-Hartley Act seriously weakened unions in their efforts to bring economic justice to the millions of American workers who remain unorganized. By administrative action anti-labor personnel appointed by the Republicans to the National Labor Relations Board have made the Taft-Hartley Act even more restrictive in its application than in its language. Thus the traditional goal of the Democratic Party, to give all workers the right to organize and bargain collectively, has still not been achieved. We pledge the enactment of an affirmative labor policy which will encourage free collective-bargaining through the growth and development of free and responsible unions. Millions of workers just now seeking to organize are blocked by Federally authorized \"right-to-work\" laws, unreasonable limitations on the right to picket, and other hampering legislative and administrative provisions. Again, in the new Labor-Management Reporting and Disclosure Act, the Republican Administration perverted the constructive effort of the Democratic Congress to deal with improper activities of a few in labor and management by turning that Act into a means of restricting the legitimate rights of the vast majority of working men and women in honest labor unions. This law likewise strikes hardest at the weak or poorly organized, and it fails to deal with abuses of management as vigorously as with those of labor. We will repeal the authorization for \"right-to-work\" laws, limitations on the rights to strike, to picket peacefully and to tell the public the facts of a labor dispute, and other anti-labor features of the Taft-Hartley Act and the 1959 Act. This unequivocal pledge for the repeal of the anti-labor and restrictive provisions of those laws will encourage collective bargaining and strengthen and support the free and honest labor movement. The Railroad Retirement Act and the Railroad Unemployment Insurance Act are in need of improvement. We strongly oppose Republican attempts to weaken the Railway Labor Act. We shall strengthen and modernize the Walsh-Healey and Davis Bacon Acts, which protect the wage standards of workers employed by Government contractors. Basic to the achievement of stable labor-management relations is leadership from the White House. The Republican Administration has failed to provide such leadership. It failed to foresee the deterioration of labor-management relations in the steel industry last year. When a national emergency was obviously developing, it failed to forestall it. When the emergency came, the Administration's only solution was government by injunction. A Democratic President, through his leadership and concern, will produce a better climate for continuing constructive relationships between labor and management. He will have periodic White House conferences between labor and management to consider their mutual problems before then reach the critical stage. A Democratic President will use the vast fact-finding facilities that are available to inform himself, and the public, in exercising his leadership in labor disputes for the benefit of the nation as a whole. If he needs more such facilities, or authority, we will provide them. We further pledge that in the administration of all labor legislation we will restore the level of integrity, competence and sympathetic understanding required to carry out the intent of such legislation. PLANNING FOR AUTOMATION The right to a job requires planning for automation, so that men and women will be trained and available to meet shifting employment needs. We will conduct a continuing analysis of the nation's manpower resources and of measures which may be required to assure their fullest development and use. We will provide the Government leadership necessary to insure that the blessings of automation do not become burdens of widespread unemployment. For the young and the technologically displaced workers, we will provide the opportunity for training and retraining that equips them for jobs to be filled. MINIMUM WAGES 2. \"The right to earn enough to provide adequate food and clothing and recreation.\" At the bottom of the income scale are some eight million families whose earnings are too low to provide even basic necessities of food, shelter, and clothing. We pledge to raise the minimum wage to one dollar and twenty five cents per hour and to extend coverage to several million workers not now protected. We pledge further improvements in the wage, hour and coverage standards of the Fair Labor Standards Act so as to extend its benefits to all workers employed in industries engaged in or affecting interstate commerce and to raise its standards to keep up with our general economic progress and needs. We shall seek to bring the two million men, women and children who work for wages on the farms of the United States under the protection of existing labor and social legislation, and to assure migrant labor, perhaps the most underprivileged of all, of a comprehensive program to bring them not only decent wages but also adequate standards of health housing Social Security protection education and welfare services. AGRICULTURE 3. \"The right of every farmer to raise and sell his products at a return which will give him and his family a decent living.\" We shall take positive action to raise farm income to full parity levels and to preserve family farming as a way of life. We shall put behind us once and for all the timidity with which our Government has viewed our abundance of food and fiber. We will set new high levels of food and consumption both at home and abroad. As long as many Americans and hundreds of millions of people in other countries remain underfed, we shall regard these agricultural riches, and the family farmers who produce them, not as a liability but as a national asset. Using Our Abundance The Democratic Administration will inaugurate a national food and fiber policy for expanded use of our agricultural abundance. We will no longer view food stockpiles with alarm but will use them as powerful instruments for peace and plenty. We will increase consumption at home. A vigorous, expanding economy will enable many American families to eat more and better food. We will use the food stamp programs authorized to feed needy children, the aged and the unemployed. We will expand and improve the school lunch and milk programs. We will establish and maintain food reserves for national defense purposes near important population centers in order to preserve lives in event of national disaster, and will operate them so as not to depress farm prices. We will expand research into new industrial uses of agricultural products. We will increase consumption abroad. The Democratic Party believes our nation's capacity to produce food and fiber is one of the great weapons for waging war against hunger and want throughout the world. With wise management of our food abundance we will expand trade between nations, support economic and human development programs, and combat famine. Unimaginative, outmoded Republican policies which fail to use these productive capacities of our farms have been immensely costly to our nation. They can and will be changed. Achieving Income Parity While farmers have raised their productive efficiency to record levels, Republican farm policies have forced their income to drop by 30 percent. Tens of thousands of farm families have been bankrupted and forced off the land. This has happened despite the fact that the Secretary of Agriculture has spent more on farm programs than all previous Secretaries in history combined. Farmers acting individually or in small groups are helpless to protect their incomes from sharp declines. Their only recourse is to produce more, throwing production still further out of balance with demand and driving prices down further. This disastrous downward cycle can be stopped only by effective farm programs sympathetically administered with the assistance of demo elected farmer committees. The Democratic Administration will work to bring about full parity income for farmers in all segments of agriculture by helping them to balance farm production with the expanding needs of the nation and the world. Measures to this end include production and marketing quotas measured in terms of barrels, bushels and bales, loans on basic commodities at not less than 90 percent of parity, production payments, commodity purchases, and marketing orders and agreements. We repudiate the Republican Administration of the Soil Bank Program, which has emphasized the retirement of whole farm units, and we pledge an orderly land retirement and conservation program. We are convinced that a successful combination of these approaches will cost considerably less than present Republican programs which have failed. We will encourage agricultural cooperatives by expanding and liberalizing existing credit facilities and developing new facilities if necessary to assist them in extending their marketing and purchasing activities, and we will protect cooperatives from punitive taxation. The Democratic Administration will improve the marketing practices of the family-type dairy farm to reduce risk of loss. To protect farmers' incomes in times of natural disaster, the Federal Crop Insurance Program, created and developed experimentally under Democratic Administrations, should be invigorated and expanded nationwide. Improving Working and Living on Farms Garm families have been among those victimized most severely by Republican tight-money policies. Young people have been barred from entering agriculture. Giant corporations and other non-farmers, with readier access to credit and through vertical integration methods, have supplanted hundreds of farm families and caused the bankruptcy of many others. The Democratic Party is committed by tradition and conviction to preservation of family agriculture. To this end, we will expand and liberalize farm credit facilities, especially to meet the needs of family farm agriculture and to assist beginning farmers. Many families in America's rural counties are still living in poverty because of inadequate resources and opportunity. This blight and personal desperation should have received national priority attention long ago. The new Democratic Administration will begin at once to eradicate long neglected rural blight. We will help people help themselves with extended and supervised credit for farm improvement, local industrial development, improved vocational training and other assistance to those wishing to change to non farm employment, and with the fullest development of commercial and recreational possibilities. This is one of the major objectives of the area redevelopment program, twice vetoed by the Republican President. The rural electric cooperatives celebrate this year the twenty-fifth anniversary of the creation of the Rural Electrification Administration under President Franklin D. Roosevelt. The Democratic Congress has successfully fought the efforts of the Republican Administration to cut off REA loans and force high interest rate policies on this great rural enterprise. We will maintain interest rates for REA co-ops and public power districts at the levels provided in present law. We deplore the Administration's failure to provide the dynamic leadership necessary for encouraging loans to rural users for generation of power where necessary. We promise the co-ops active support in meeting the ever growing demand for electric power and telephone service, to be filled on a complete area coverage basis without requiring benefits for special interest power groups. In every way we will seek to help the men, women, and children whose livelihood comes from the soil to achieve better housing, education, health, and decent earnings and working conditions. All these goals demand the leadership of a Secretary of Agriculture who is conversant with the technological and economic aspects of farm problems, and who is sympathetic with the objectives of effective farm legislation not only for farmers but for the best interest of the nation as a whole. SMALL BUSINESS 4. \"The right of every businessman, large and small, to trade in an atmosphere of freedom from unfair competition and domination by monopolies at home and abroad.\" The new Democratic Administration will act to make our free economy really free, free from the oppression of monopolistic power, and free from the suffocating impact of high interest rates. We will help create an economy in which small businesses can take root, grow and flourish. We Democrats pledge: 1. Action to aid small business in obtaining credit and equity capital at reasonable rates. Small business which must borrow to stay alive has been a particular victim of the high interest policies of the Republican administration. The loan program of the small Business Administration should be accelerated, and the independence of that agency preserved. the Small Business Investment Act of 1958 must be administered with a greater sense of its importance and possibilities. 2. Protection of the public against the growth of monopoly. The last seven and one-half years of Republican government has been the greatest period of merger and amalgamation in industry and banking in American history. Democratic Congresses have enacted numerous important measures to strengthen our anti trust laws. Since 1950 the four Democratic Congresses have enacted laws like the Celler-Kefauver Anti-Merger Act, and improved the laws against price discriminations and tie in sales. When the Republicans were in control of the 80th and 83rd Congresses they failed to enact a single measure to strengthen or improve the anti-trust laws. The democratic party opposes this trend to monopoly. We pledge vigorous enforcement of the anti-trust laws. We favor requiring corporations to file advance notice of mergers with the anti-trust enforcement agencies. We favor permitting all firms to have access at reasonable rates to patented inventions resulting from Government financed research and development contracts. We favor strengthening the Robinson-Patman Act to protect small business against price discrimination. We favor authorizing the Federal Trade Commission to obtain temporary injunctions during the pendency of administrative proceedings. 3. A more equitable share of Government contracts to small and independent business. We will move from almost complete reliance on negotiation in the award of government contracts toward open, competitive bidding. HOUSING 5. \"The right of every family to a decent home.\" Today our rate of home building is less than that of ten years ago. A healthy, expanding economy will enable us to build two million homes a year, in wholesome neighborhoods, for people of all incomes. At this rate, within a single decade we can clear away our slums and assure every American family a decent place to live. Republican policies have led to a decline of the home building industry and the production of fewer homes. Republican high-interest policies have forced the cost of decent housing beyond the range of many families. Republican indifference has perpetuated slums. We record the unpleasant fact that in 1960 at least 40 million Americans live in substandard housing. One million new families are formed each year and need housing, and 300,000 existing homes are lost through demolition or other causes and need to be replaced. At present, construction does not even meet these requirements, much less permit reduction of the backlog of slum units. We support a housing construction goal of more than two million homes a year. Most of the increased construction will be priced to meet the housing needs of middle-income and low-income families who now live in substandard housing and are priced out of the market for decent homes. Our housing programs will provide for rental as well as sales housing. They will permit expanded cooperative housing programs and sharply stepped up rehabilitation of existing homes. To make possible the building of two million homes a year in wholesome neighborhoods, the home building industry should be aided by special mortgage assistance, with low interest rates, long-term mortgage periods and reduced down payments. Where necessary, direct Government loans should be provided. Even with this new and flexible approach, there will still be need for a substantial low-rent public housing program authorizing as many units as local communities require and are prepared to build. HEALTH 6. \"The right to adequate medical care and the opportunity to achieve and enjoy good health.\" Illness is expensive. Many Americans have neither incomes nor insurance protection to enable them to pay for modern health care. The problem is particularly acute with our older citizens, among whom serious illness strikes most often. We shall provide medical care benefits for the aged as part of the time-tested Social Security insurance system. We reject any proposal which would require such citizens to submit to the indignity of a means test, a \"pauper's oath\". For young and old alike, we need more medical schools, more hospitals, more research laboratories to speed the final conquest of major killers. Medical Care for Older Persons Fifty million Americans, more than a fourth of our people, have no insurance protection against the high cost of illness. For the rest, private health insurance pays, on the average, only about one-third of the cost of medical care. The problem is particularly acute among the 16 million Americans over 65 years old, and among disabled workers, widows and orphans. Most of these have low-incomes and the elderly among them suffer two to three times as much illness as the rest of the population. The Republican Administration refused to acknowledge any national responsibility for health care for elder citizens until forced to do so by an increasingly outraged demand. Then, its belated proposal was a cynical sham built around a degrading test based on means or income, a \"pauper's oath\". The most practicable way to provide health protection for older people is to use the contributory machinery of the Social Security system for insurance covering hospital bills and other high-cost medical services. For those relatively few of our older people who have never been eligible for Social Security coverage we shall provide corresponding benefits by appropriations from the general revenue.  Research We will step up medical research on the major killers and crippling diseases, cancer, heart disease, arthritis, mental illness. Expenditures for these purposes should be limited only by the availability of personnel and promising lines of research. Today such illness costs us 35 billion dollars annually, much of which could be avoided. Federal appropriations for medical research are barely one percent of this amount. Heart disease and cancer together account for two out of every three deaths in this country. The Democratic President will summon to a White House conference the nation's most distinguished scientists in these fields to map a coordinated long run program for the prevention and control of these diseases. We will also support a cooperative program with other nations on international health research. Hospitals We will expand and improve the Hill-Burton hospital construction program. Health Manpower To ease the growing shortage of doctors and other medical personnel we propose Federal aid for constructing, expanding and modernizing schools of medicine, dentistry, nursing and public health. We are deeply concerned that the high cost of medical education is putting this profession beyond the means of most American families. We will provide scholarships and other assistance to break through the financial barriers to medical education. Mental Health Mental patients fill more than half the hospital beds in the country today. We will provide greatly increased Federal support for psychiatric research and training, and community mental health programs, to help bring back thousands of our hospitalized mentally ill to full and useful lives in the community. 7. \"The right to adequate protection from the economic fears of old age, sickness, accidents, and unemployment.\" A PROGRAM FOR THE AGING The Democratic Administration will end the neglect of our older citizens. They deserve lives of usefulness, dignity, independence, and participation. We shall assure them not only health care but employment for those who want work, decent housing, and recreation. Already 16 million Americans, about one in ten, are over 65, with the prospect of 26 million by 1980. Health As stated, we will provide an effective system for paid up medical insurance upon retirement financed during working years through the Social Security mechanism and available to all retired persons without a means test. This has first priority. Income Half of the people over 65 have incomes inadequate for basic nutrition, decent housing, minimum recreation and medical care. Older people who do not want to retire need employment opportunity and those of retirement age who no longer wish to or cannot work need better retirement benefits. We pledge a campaign to eliminate discrimination in employment due to age. As a first step we will prohibit such discrimination by Government contractors and subcontractors. We will amend the Social Security Act to increase the retirement benefit for each additional year of work after 65, thus encouraging workers to continue on the job full time. To encourage part time work by others, we favor raising the 1200 dollar-a-year ceiling on what a worker may earn while still drawing Social Security benefits. Retirement benefits must be increased generally, and minimum benefits raised from 33 dollars a month to 50 dollars. Housing We shall provide decent and suitable housing which older persons can afford. Specifically we shall move ahead with the program of direct Government loans for housing for older people initiated in the Housing Act of 1959, a program which the Republican Administration has sought to kill. Special Services We shall take Federal action in support of state efforts to bring standards of care in nursing homes and other institutions for the aged up to desirable minimums. We shall support demonstration and training programs to translate proven research into action in such fields as health, nutritional guidance, home care, counseling, recreational activity. Taken together, these measures will affirm a new charter of rights for the older citizens among us, the right to a life of usefulness, health, dignity, independence and participation. WELFARE Disability Insurance We shall permit workers who are totally and permanently disabled to retire at any age, removing the arbitrary requirement that the worker be 50 years of age. We shall also amend the law so that after six months of total disability, a worker will be eligible for disability benefits, with restorative services to enable him to return to work. Physically Handicapped We pledge continued support of legislation for the rehabilitation of physically handicapped persons and improvement of employment opportunity for them. Public Assistance Persons in need who are inadequately protected by social insurance are cared for by the states and local communities under public assistance programs. The Federal Government, which now shares the cost of aid to some of these, should share in all, and benefits should be made available without regard to residence. Unemployment Benefits We will establish uniform minimum standards throughout the nation for coverage, duration, and amount of unemployment insurance benefits. Equality for Women We support legislation which will guarantee to women equality of rights under the law, including equal pay for equal work. Child Welfare The Child Welfare Program and other services already established under the Social Security Act should be expanded. Federal leadership is required in the nationwide campaign to prevent and control juvenile delinquency. Intergroup Relations We propose a Federal bureau of intergroup relations to help solve problems of discrimination in housing, education, employment, and community opportunities in general. The bureau would assist in the solution of problems arising from the resettlement of immigrants and migrants within our own country, and in resolving religious, social and other tensions where they arise. EDUCATION 8. \"The right to a good education.\" America's young people are our greatest resource for the future. Each of them deserves the education which will best develop his potentialities. We shall act at once to help in building the classrooms and employing the teachers that are essential if the right to a good education is to have genuine meaning for all the youth of America in the decade ahead. As a national investment in our future we propose a program of loans and scholarship grants to assure that qualified young Americans will have full opportunity for higher education, at the institutions of their choice, regardless of the income of their parents. The new Democratic Administration will end eight years of official neglect of our educational system. America's education faces a financial crisis. The tremendous increase in the number of children of school and college age has far outrun the available supply of educational facilities and qualified teachers. The classroom shortage alone is interfering with the education of ten million students. America's teachers, parents and school administrators have striven courageously to keep up with the increased challenge of education. So have states and local communities. Education absorbs two fifths of all their revenue. with limited resources, private educational institutions have shouldered their share of the burden. Only the Federal Government is not doing its part. For eight years, measures for the relief of the educational crisis have been held up by the cynical maneuvers of the Republican Party in Congress and the White House. We believe that America can meet its educational obligations only with generous Federal financial support, within the traditional framework of local control. The assistance will take the form of Federal grants to states for educational purposes they deem most pressing including classroom construction and teachers salaries. It will include aid for the construction of academic facilities as well as dormitories at colleges and universities. We pledge further Federal support for all phases of vocational education for youth and adults, for libraries and adult education for realizing the potential of educational television, and for exchange of students and teachers with other nations. As part of a broader concern for young people we recommend establishment of a Youth Conservation Corps, to give underprivileged young people a rewarding experience in a healthful environment. The pledges contained in this Economic Bill of Rights point the way to a better life for every family in America. They are the means to a goal that is now within our reach, the final eradication in America of the age-old evil of poverty. Yet there are other pressing needs on our national agenda. NATURAL RESOURCES A thin layer of earth, a few inches of rain, and a blanket of air make human life possible on our planet. Sound public policy must assure that these essential resources will be available to provide the good life for our children and future generations. Water, timber and grazing lands, recreational areas in our parks, shores, forests and wildernesses, energy, minerals, even pure air, all are feeling the press of enormously increased demands of a rapidly growing population. Natural resources are the birthright of all the people. The new Democratic Administration, with the vision that built a TVA and a Grand Coulee, will develop and conserve that heritage for the use of this and future generations. We will reverse Republican policies under which America's resources have been wasted depleted underdeveloped and recklessly given away. We favor the best use of our natural resources, which generally means adoption of the multiple purpose principle to achieve full development for all the many functions they can serve. Water and Soil An abundant supply of pure water is essential to our economy. this is a national problem. Water must serve domestic, industrial and irrigation needs and inland navigation. It must provide habitat for fish and wildlife, supply the base for much outdoor recreation, and generate electricity. Water must also be controlled to prevent floods, pollution, salinity and silt. The new Democratic Administration will develop a comprehensive national water resource policy. In cooperation with state and local governments, and interested private groups, the Democratic Administration will will develop a balanced, multiple purpose plan for each major river basin, to be revised periodically to meet changing needs. We will erase the Republican slogan of \"no new starts\" and will begin again to build multiple purpose dams, hydroelectric facilities, flood-control works, navigation facilities, and reclamation projects to meet mounting and urgent needs. We will renew the drive to protect every acre of farm land under a soil and water conservation plan, and we will speed up the small watershed program. We will support and intensify the research effort to find an economical way to convert salt and brackish water. The Republicans discouraged this research, which holds untold possibilities for the whole world. Water and Air Pollution America can no longer take pure water and air for granted. Polluted rivers carry their dangers to everyone living along their courses, impure air does not respect boundaries. Federal action is needed in planning, coordinating and helping to finance pollution control. The states and local communities cannot go it alone. Yet President Eisenhower vetoed a Democratic bill to give them more financial help in building sewage treatment plants. A Democratic President will sign such a bill. Democrats will step up research on pollution control giving special attention to: 1. the rapidly growing problem of air pollution from industrial plants, automobile exhausts, and other sources, and 2. disposal of chemical and radioactive wastes, some of which are now being dumped off our coasts without adequate knowledge of the potential consequences. Outdoor Recreation As population grows and the work week shortens and transportation becomes easier and speedier, the need for outdoor recreation facilities mounts. We must act quickly to retain public access to the oceans gulfs rivers streams lakes and reservoirs areas while there is yet time. Areas near major population centers are and their shorelines, and to reserve adequate camping and recreational particularly needed. The new Democratic Administration will work to improve and extend recreation opportunities in national parks and monuments, forests, and river development projects, and near metropolitan areas. Emphasis will be on attractive low-cost facilities for all the people and on preventing undue commercialization. The National Park System is still incomplete, in particular, the few remaining suitable shorelines must be included in it. A national wilderness system should be created for areas already set aside as wildernesses. The system should be extended but only after careful consideration by the Congress of the value of areas for competing uses. Recreational needs of the surrounding area should be given important consideration in disposing of Federally owned lands. We will protect fish and game habitats from commercial exploitation and require military installations to conform to sound conservation practices. Energy The Republican Administration would turn the clock back to the days before the New Deal, in an effort to divert the benefits of the great natural energy resources from all the people to a favored few. It has followed for many years a \"no new starts\" policy. It has stalled atomic energy development, it has sought to cripple rural electrification. It has closed the pilot plant on getting oil from shale. It has harrassed and hampered the TVA. We reject this philosophy and these policies. The people are entitled to use profitably what they already own. The Democratic Administration instead will foster the development of efficient regional giant power systems from all sources, including water, tidal, and nuclear power, to supply low cost electricity to all retail electric systems, public, private, and cooperative. The Democratic Administration will continue to develop \"yardsticks\" for measuring the rates of private utility systems. This means meeting the needs of rural electric cooperatives for low interest loans for distribution, transmission and generation facilities, Federal transmission facilities, where appropriate, to provide efficient low-cost power supply, and strict enforcement of the public preference clauses in power marketing. The Democratic Administration will support continued study and research on energy fuel resources, including new sources in wind and sun. It will push forward with the Passamaquoddy tidal power project with its great promise of cheaper power and expanded prosperity for the people of New England. We support the establishment of a national fuels policy. The 15 billion dollar national investment in atomic energy should be protected as a part of the public domain. Federal Lands and Forests The record of the Republican Administration in handling the public domain is one of complete lethargy. It has failed to secure existing assets. In some cases, it has given away priceless resources for plunder by private corporations, as in the Al Sarena mining incident and the secret leasing of game refuges to favored oil interests. The new Democratic Administration will develop balanced land and forest policies suited to the needs of a growing America. This means intensive forest management in a multiple-use and sustained-yield basis, reforestation of burnt-over lands, building public access roads, range reseeding and improvement, intensive work in watershed management, concern for small business operations, and insuring free public access to public lands for recreational uses. Minerals America uses half the minerals produced in the entire Free World. Yet our mining industry is in what may be the initial phase of a serious long-term depression. Sound policy requires that we strengthen the domestic mining industry without interfering with adequate supplies of needed materials at reasonable costs. We pledge immediate efforts toward the establishment of a realistic long-range minerals policy. The new Democratic Administration will begin intensive research on scientific prospecting for mineral deposits. We will speed up the geologic mapping of the country, with emphasis on Alaska. We will resume research and development work on use of low grade mineral reserves, especially oil shale, lignites, iron ore taconite, and radio-active minerals. These efforts have been halted or cut back by the Republican Administration. The Democratic Party favors a study of the problem of non uniform seaward boundaries of the costal states. Government Machinery for Managing Resources Long-range programming of the nation's resource development is essential. We favor creation of a council of advisors on resources and conservation, which will evaluate and report annually upon our resource needs and progress. We shall put budgeting for resources on a businesslike basis, distinguishing between operating expense and capital investment, so that the country can have an accurate picture of the costs and returns. We propose the incremental method in determining the economic justification of our river basin programs. Charges for commercial use of public lands will be brought into line with benefits received. CITIES AND THEIR SUBURBS A new Democratic Administration will expand Federal programs to help urban communities clear their slums, dispose of their sewage, educate their children, transport suburban commuters to and from their jobs, and combat juvenile delinquency. We will give the city dweller a voice at the Cabinet table by bringing together within a single department programs concerned with urban and metropolitan problems. The United States is now predominantly an urban nation. The efficiency, comfort, and beauty of our cities and suburbs influence the lives of all Americans. Local governments have found increasing difficulty in coping with such fundamental public problems as urban renewal, slum clearance, water supply, mass transportation recreation, health, welfare, education and metropolitan planning. These problems are, in many cases, interstate and regional in scope. Yet the Republican Administration has turned its back on urban and suburban America. The list of Republican vetoes includes housing, urban renewal and slum clearance, area redevelopment, public works, airports and steam pollution control. It has proposed severe cutbacks in aid for hospital construction, public assistance, vocational education, community facilities and sewage disposal. The result has been to force communities to thrust an ever greater tax load upon the already overburdened property taxpayer and to forgo needed public services. The Democratic Party believes that state and local governments are strengthened, not weakened by financial assistance from the Federal Government. We will extend such aid without impairing local administration through unnecessary Federal interference or red tape. We propose a ten year action program to restore our cities and provide for balanced suburban development, including the following: 1. The elimination of slums and blight and the restoration of cities and depressed areas within the next ten years. 2. Federal aid for metropolitan area planning and community facility programs. 3. Federal aid for comprehensive metropolitan transportation programs, including bus and rail mass transit, commuter railroads as well as highway programs, and construction of civil airports. 4. Federal aid in combating air and water pollution. 5. Expansion of park systems to meet the recreation needs of our growing population. The Federal Government must recognize the financial burdens placed on local governments, urban and rural alike, by Federal installations and land holdings. TRANSPORTATION Over the past seven years, we have watched the steady weakening of the nation's transportation system. Railroads are in distress. Highways are congested. Airports and airways lag far behind the needs of the jet age. To meet this challenge we will establish a national transportation policy, designed to coordinate and modernize our facilities for transportation by road, rail, water, and air. Air The jet age has made rapid improvement in air safety imperative. Rather than \"an orderly withdrawal\" from the airport grant programs as proposed by the Republican Administration, we pledge to expand the program to accommodate growing air traffic. Water Development of our inland waterways, our harbors, and Great Lakes commerce has been held back by the Republican President. We pledge the improvement of our rivers and harbors by new starts and adequate maintenance. A strong and efficient American-flag merchant marine is essential to peacetime commerce and defense emergencies. Continued aid for ship construction and operation to offset cost differentials favoring foreign shipping is essential to these goals. Roads The Republican Administration has slowed down, stretched out and greatly increased the costs of the interstate highway program. The Democratic Party supports the highway program embodied in the Acts of 1956 and 1958 and the principle of Federal-state partnership in highway construction. We commend the Democratic Congress for establishing a special committee which has launched an extensive investigation of this highway program. Continued scrutiny of this multi-billion-dollar highway program can prevent waste, inefficiency and graft and maintain the public's confidence. Rail The nation's railroads are in particular need of freedom from burdensome regulation to enable them to compete effectively with other forms of transportation. We also support Federal assistance in meeting certain capital needs, particularly for urban mass transportation. SCIENCE We will recognize the special role of our Federal Government in support of basic and applied research. Space The Republican Administration has remained incredibly blind to the prospects of space exploration. It has failed to pursue space programs with a sense of urgency at all close to their importance to the future of the world. It has allowed the Communists to hit the moon first, and to launch substantially greater payloads. The Republican program is a catchall of assorted projects with no clearly defined, long range plan of research. The new Democratic Administration will press forward with our national space program in full realization of the importance of space accomplishments to our national security and our international prestige. We shall reorganize the program to achieve both efficiency and speedy execution. We shall bring top scientists into positions of responsibility. We shall undertake long-term basic research in space science and propulsion. We shall initiate negotiations leading toward the international regulation of space. Atomic Energy The United States became pre-eminent in the development of atomic energy under Democratic Administrations. The Republican Administration, despite its glowing promises of \"Atoms for Peace\" has permitted the gradual deterioration of United States leadership in atomic development both at home and abroad. In order to restore United States leadership in atomic development, the new Democratic Administration will: 1. Restore truly nonpartisan and vigorous administration of the vital atomic energy program. 2. Continue the development of the various promising experimental and prototype atomic power plants which show promise, and provide increasing support for longer-range projects at the frontiers of atomic energy application. 3. Continue to preserve and support national laboratories and other Federal atomic installations as the foundation of technical progress and a bulwark of national defense. 4. Accelerate the Rover nuclear rocket project and auxiliary power facilities so as to achieve world leadership in peaceful outer space exploration. 5. Give reality to the United States international atoms-for peace programs and continue and expand technological assistance to underdeveloped countries. 6. Consider measures for improved organization and procedure for radiation protection and reactor safety, including a strengthening of the role of the Federal Radiation Council, and the separation of quasi-judicial functions in reactor safety regulations. 7. Provide a balanced and flexible nuclear defense capability, including the augmentation of the nuclear submarine fleet. Oceanography Oceanographic research is needed to advance such important programs as food and minerals from our Great Lakes and the sea. the present Administration has neglected this new scientific frontier. GOVERNMENT OPERATIONS We shall reform the processes of Government in all branches, Executive, Legislative, and Judicial. We will clean out corruption and conflicts of interest, and improve Government services. The Federal Service Two weeks before this Platform was adopted, the difference between the Democratic and Republican attitudes toward Government employees was dramatically illustrated. The Democratic Congress passed a fully justified pay increase to bring Government pay scales more nearly into line with those of private industry. The Republican President vetoed the pay raise. The Democratic Congress decisively overrode the veto. The heavy responsibilities of modern government require a Federal service characterized by devotion to duty, honesty of purpose and highest competence. We pledge the modernization and strengthening of our Civil Service system. We shall extend and improve the employees appeals system and improve programs for recognizing the outstanding merits of individual employees. Ethics in Government We reject totally the concept of dual or triple loyalty on the part of Federal officials in high places. The conflict-of-interest statutes should be revised and strengthened to assure the Federal service of maximum security against unethical practices on the part of public officials. The Democratic Administration will establish and enforce a Code of Ethics to maintain the full dignity and integrity of the Federal service and to make it more attractive to the ablest men and women. Regulatory Agencies The Democratic Party promises to clean up the federal regulatory agencies. The acceptance by Republican appointees to these agencies of gifts, hospitality, and bribes from interests under their jurisdiction has been a particularly flagrant abuse of public trust. We shall bring all contacts with commissioners into the open, and will protect them from any form of improper pressure. We shall appoint to these agencies men of ability and independent judgment who understand that their function is to regulate these industries in the public interest. We promise a thorough review of existing agency practices, with an eye toward speedier decisions, and a clearer definition of what constitutes the public interest. The Democratic Party condemns the usurpation by the Executive of the powers and functions of any of the independent agencies and pledges the restoration of the independence of such agencies and the protection of their integrity of action. The Postal Service The Republican policy has been to treat the United States postal service as a liability instead of a great investment in national enlightenment, social efficiency and economic betterment. Constant curtailment of service has inconvenienced every citizen. A program must be undertaken to establish the Post Office Department as a model of efficiency and service. We pledge ourselves to: 1. Restore the principle that the postal service is a public service. 2. Separate the public service costs from those to be borne by the users of the mails. 3. Continue steady improvement in working conditions and wage scales, reflecting increasing productivity. 4. Establish a long range program for research and capital improvements compatible with the highest standards of business efficiency. Law Enforcement In recent years, we have been faced with a shocking increase in crimes of all kinds. Organized criminals have even infiltrated into legitimate business enterprises and labor unions. The Republican Administration, particularly the Attorney General's office, has failed lamentably to deal with this problem despite the growing power of the underworld. The new Democratic Administration will take vigorous corrective action. Freedom of Information We reject the Republican contention that the workings of Government are the special private preserve of the Executive. The massive wall of secrecy erected between the Executive branch and the Congress as well as the citizen must be torn down. Information must flow freely, save in those areas in which the national security is involved. Clean Elections The Democratic Party favors realistic and effective limitations on contributions and expenditures, and full disclosure of campaign financing in Federal elections. We further propose a tax credit to encourage small contributions to political parties. The Democratic Party affirms that every candidate for public office has a moral obligation to observe and uphold traditional American principles of decency, honesty and fair play in his campaign for election. We deplore efforts to divide the United States into regional and ethnic groups. We denounce and repudiate campaign tactics that substitute smear and slander, bigotry and false accusations of bigotry, for truth and reasoned argument. District of Columbia The capital city of our nation should be a symbol of democracy to people throughout the world. The Democratic Party reaffirms its long-standing support of home rule for the District of Columbia, and pledges to enact legislation permitting voters of the District to elect their own local government. We urge the legislatures of the 50 states to ratify the 23rd Amendment passed by the Democratic Congress to give Distict citizens the right to participate in Presidential elections. We also support a Constitutional amendment giving the District voting representation in Congress. Virgin Islands We believe that the voters of the Virgin Islands should have the right to elect their own Governor, to have a delegate in the Congress of the United States and to have the right to vote in national elections for a President and Vice President of the United States. Puerto Rico The social, economic, and political progress of the Commonwealth of Puerto Rico is a testimonial to the sound enabling legislation, and to the sincerity and understanding with which the people of the 50 states and Puerto Rico are meeting their joint problems. The Democratic Party, under whose administration the Commonwealth status was established, is entitled to great credit for providing the opportunity which the people of Puerto Rico have used so successfully. Puerto Rico has become a show place of world-wide interest, a tribute to the benefits of the principles of self-determination. Further benefits for Puerto Rico under these principles are certain to follow. CONGRESSIONAL PROCEDURES In order that the will of the American people may be expressed upon all legislative proposals, we urge that action be taken at the beginning of the 87th Congress to improve Congressional procedures so that majority rule prevails and decisions can be made after reasonable debate without being blocked by a minority in either House. The rules of the House of Representatives should be so amended as to make sure that bills reported by legislative committees reach the floor for consideration without undue delay. CONSUMERS In an age of mass production, distribution, and advertising, consumers require effective Government representation and protection. The Republican Administration has allowed the Food and Drug Administration to be weakened. Recent Senate hearings on the drug industry have revealed how flagrant profiteering can be when essential facts on costs, prices, and profits are hidden from scrutiny. The new Democratic Administration will provide the money and the authority to strengthen this agency for its task. We propose a consumer counsel, backed by a suitable staff, to speak for consumers in the formulation of Government policies and represent consumers in administrative proceedings. The consumer also has a right to know the cost of credit when he borrows money. We shall enact Federal legislation requiring to vendors of credit to provide a statement of specific credit charges and what these charges cost in terms of true annual interest. VETERANS AFFAIRS We adhere to the American tradition dating from the Plymouth Conony in New England in 1636: \"any soldier injured in defense of the colony shall be maintained competently by the colony for the remainder of his life.\" We pledge adequate compensation for those with service connected disabilities and for the survivors of those who died in service or from service-connected disabilities. We pledge pensions adequate for a full and dignified life for disabled and distressed veterans and for needy survivors of deceased veterans. Veterans of World War I whose Federal benefits have not matched those of veterans of subsequent service, will receive the special attention of the Democratic Party looking toward equitable adjustments. We endorse expanded programs of vocational rehabilitation for disabled veterans, and education for orphans of servicemen. The quality of medical care furnished to the disabled veterans has deteriorated under the Republican Administration. We shall work for an increased availability of facilities for all veterans in need and we shall move with particular urgency to fulfull the need for expanded domiciliary and nursing-home facilities. We shall continue the veterans home loan guarantee and direct loan programs and educational benefits patterned after the G.I. Bill of Rights. AMERICAN INDIANS We recognize the unique legal and moral responsibility of the Federal Government for Indians in restitution for the injustice that has sometimes been done them. We therefore pledge prompt adoption of a program to assist Indian tribes in the full development of their human and natural resources and to advance the health, education, and economic well-being of Indian citizens while preserving their cultural heritage. Free consent of the Indian tribes concerned shall be required before the Federal Government makes any change in any Federal Indian treaty or other contractual relationship. The new Democratic Administration will bring competent, sympathetic, and dedicated leadership to the administration of Indian affairs which will end practices that have eroded Indian rights and resources, reduced the Indians land base and repudiated Federal responsibility. Indian claims against the United States can and will be settled promptly, whether by negotiation or other means, in the best interests of both parties. THE ARTS The arts flourish where there is freedom and where individual initiative and imagination are encouraged. We enjoy the blessings of such an atmosphere. The nation should begin to evaluate the possibilities for encouraging and expanding participation in and appreciation of our cultural life. We propose a federal advisory agency to assist in the evaluation, development, and expansion of cultural resources of the united-states. we shall support legislation needed to provide incentives for those endowed with extraordinary talent, as a worthy supplement to existing scholarship programs. CIVIL LIBERTIES With democratic values threatened today by Communist tyranny, we reaffirm our dedication to the Bill of Rights. Freedom and civil liberties, far from being incompatible with security, are vital to our national strength. Unfortunately, those high in the Republican Administration have all too often sullied the name and honor of loyal and faithful American citizens in and out of Government. The Democratic Party will strive to improve Congressional investigating and hearing procedures. We shall abolish useless disclaimer affidavits such as those for student educational loans. We shall provide a full and fair hearing, including confrontation of the accuser, to any person whose public or private employment or reputation is jeopardized by a loyalty or security proceeding. Protection of rights of American citizens to travel, to pursue lawful trade and to engage in other lawful activities abroad without distinction as to race or religion is a cardinal function of the national sovereignty. We will oppose any international agreement or treaty which by its terms or practices differentiates among American citizens on grounds of race or religion. The list of unfinished business for America is long. The accumulated neglect of nearly a decade cannot be wiped out overnight. Many of the objectives which we seek will require our best efforts over a period of years. Although the task is far-reaching, we will tackle it with vigor and confidence. We will substitute planning for confusion, purpose for indifference, direction for drift and apathy. We will organize the policymaking machinery of the Executive branch to provide vigor and leadership in establishing our national goals and achieving them. The new Democratic President will sign, not veto, the efforts of a Democratic Congress to create more jobs, to build more homes, to save family farms, to clean up polluted streams and rivers, to help depressed areas, and to provide full employment for our people. FISCAL RESPONSIBILITY We vigorously reject the notion that America, with a half trillion-dollar gross national product, and nearly half of the world's industrial resources, cannot afford to meet our needs at home and in our world relationships. We believe, moreover, that except in periods of recession or national emergency, these needs can be met with a balanced budget, with no increase in present tax rates, and with some surplus for the gradual reduction of our national debt. To assure such a balance we shall pursue a four-point program of fiscal responsibility. First, we shall end the gross waste in Federal expenditures which needlessly raises the budgets of many Government agencies. The most conspicuous unnecessary item is, of course, the excessive cost of interest on the national debt. Courageous action to end duplication and competition among the armed services will achieve large savings. The cost of the agricultural program can be reduced while at the same time prosperity is being restored to the nation's farmers. Second, we shall collect the billions in taxes which are owed to the Federal Government but not now collected. The Internal Revenue Service is still suffering from the cuts inflicted upon its enforcement staff by the Republican Administration and the Republican Congress in 1953. The Administration's own Commissioner of Internal Revenue has testified that billions of dollars in revenue are lost each year because of the Service does not have sufficient agents to follow up on tax evasion. We will add enforcement personnel, and develop new techniques of enforcement, to collect tax revenue which is now being lost through evasion. Third, we shall close the loopholes in the tax laws by which certain privileged groups legally escape their fair share of taxation. Among the more conspicuous loopholes are depletion allowances which are inequitable, special consideration for recipients of dividend income, and deductions for extravagant \"business expenses\" which have reached scandalous proportions. Tax reform can raise additional revenue and at the same time increase legitimate incentives for growth, and make it possible to ease the burden on the general taxpayer who now pays an unfair share of taxes because of special favors to the few. Fourth, we shall bring in added Federal tax revenues by expanding the economy itself. Each dollar of additional production puts an additional 18 cents in tax revenue in the national treasury. A five percent growth rate, therefore, will mean that at the end of four years the Federal Government will have had a total of nearly 50 billion dollars in additional tax revenues above those presently received. By these four methods we can sharply increase the Government funds available for needed services, for correction of tax inequities, and for debt or tax reduction. Much of the challenge of the 1960s, however, remains unforeseen and unforeseeable. If, therefore, the unfolding demands of the new decade at home or abroad should impose clear national responsibilities that cannot be fulfilled without higher taxes, we will not allow political disadvantage to deter us from doing what is required. As we proceed with the urgent task of restoring America's productivity, confidence, and power we will never forget that our national interest is more than the sum total of all the group interests in America. When group interests conflict with the national interest, it will be the national interest which we serve. On its values and goals the quality of American life depends. Here above all our national interest and our devotion to the Rights of Man coincide. Democratic Administrations under Wilson, Roosevelt, and Truman led the way in pressing for economic justice for all Americans. But man does not live by bread alone. A new Democratic Administration, like its predecessors, will once again look beyond material goals to the spiritual meaning of American society. We have drifted into a national mood that accepts payola and quiz scandals, tax evasion and false expense accounts, soaring crime rates, influence peddling in high Government circles, and the exploitation of sadistic violence as popular entertainment. For eight long critical years our present national leadership has made no effective effort to reverse this mood. The new Democratic Administration will help create a sense of national purpose and higher standards of public behavior. CIVIL RIGHTS We shall also seek to create an affirmative new atmosphere in which to deal with racial divisions and inequalities which threaten both the integrity of our democratic faith and the proposition on which our nation was founded, that all men are created equal. It is our faith in human dignity that distinguishes our open free society from the closed totalitarian society of the Communists. The Constitution of the United States rejects the notion that the Rights of Man means the rights of some men only. We reject it too. The right to vote is the first principle of self-government. The Constitution also guarantees to all Americans the equal protection of the laws. It is the duty of the Congress to enact the laws necessary and proper to protect and promote these constitutional rights. The Supreme Court has the power to interpret these rights and the laws thus enacted. It is the duty of the President to see that these rights are respected and that the Constitution and laws as interpreted by the Supreme Court are faithfully executed. What is now required is effective moral and political leadership by the whole Executive branch of our Government to make equal opportunity a living reality for all Americans. As the party of Jefferson, we shall provide that leadership. In every city and state in greater or lesser degree there is discrimination based on color, race, religion, or national origin. If discrimination in voting, education, the administration of justice or segregated lunch counters are the issues in one area, discrimination in housing and employment may be pressing questions elsewhere. The peaceful demonstrations for first-class citizenship which have recently taken place in many parts of this country are a signal to all of us to make good at long last the guarantees of our constitution. The time has come to assure equal access for all americans to all areas of community life, including voting booths, schoolrooms, jobs, housing, and public facilities. The Democratic Administration which takes office next January will therefore use the full powers provided in the Civil Rights Acts of 1957 and 1960 to secure for all Americans the right to vote. If these powers, vigorously invoked by a new Attorney General and backed by a strong and imaginative Democratic President prove inadequate, further powers will be sought. We will support whatever action is necessary to eliminate literacy tests and the payment of poll taxes as requirements for voting. A new Democratic Administration will also use its full powers, legal and moral, to ensure the beginning of good faith compliance with the Constitutional requirement that racial discrimination be ended in public education. We believe that every school district affected by the Supreme Court's school desegregation decision should submit a plan providing for at least first step compliance by 1963, the 100th anniversary of the Emancipation Proclamation. To facilitate compliance, technical and financial assistance should be given to school districts facing special problems of transition. For this and for the protection of all other Constitutional rights of Americans, the Attorney General should be empowered and directed to file civil injunction suits in Federal courts to prevent the denial of any civil right on grounds of race, creed, or color. The new Democratic Administration will support Federal legislation establishing a Fair Employment Practices Commission to secure effectively for everyone the right of equal opportunity for employment. In 1949 the President's Committee on Civil Rights recommended a permanent Commission on Civil Rights. The new Democratic Administration will broaden the scope and strengthen the powers of the present commission and make it permanent.  Its functions will be to provide assistance to communities, industries, or individuals in the implementation of Constitutional rights in education, housing, employment, transportation, and the administration of justice. In addition, the Democratic Administration will use its full executive powers to assure equal employment opportunities and to terminate racial segregation throughout Federal services and institutions, and on all Government contracts. The sucessful desegregation of the armed services took place through such decisive executive action under President Truman. Similarly the new Democratic Administration will take action to end discrimination in federal housing programs, including Federally assisted housing. To accomplish these goals will require executive orders, legal actions brought by the Attorney General, legislation, and improved Congressional procedures to safeguard majority rule. Above all, it will require the strong, active, persuasive, and inventive leadership of the President of the United States. The Democratic President who takes office next January will face unprecedented challenges. His Administration will present a new face to the world. It will be a bold, confident, affirmative face. We will draw new strength from the universal truths which the founder of our Party asserted in the Declaration of Independence to be \"self evident\". Emerson once spoke of an unending contest in human affairs, a contest between the Party of Hope and the Party of Memory. For seven and one-half years America, governed by the Party of Memory, has taken a holiday from history. As the Party of Hope it is our responsibility and opportunity to call forth the greatness of the American people. In this spirit, we hereby rededicate ourselves to the continuing service of the Rights of Man everywhere in America and everywhere else on God's earth."
\end{verbatim}

\begin{Shaded}
\begin{Highlighting}[]
\FunctionTok{table}\NormalTok{(}\FunctionTok{codes}\NormalTok{(corpus)) }\DocumentationTok{\#\# count codes of all manifestos}
\end{Highlighting}
\end{Shaded}

\begin{verbatim}
## 
##   000   101   102   103 103.1 103.2   104   105   106   107   108   109   110 
##   219   325    66     6     4     4   941   205   136   726     4   196     2 
##   201 201.1 201.2   202 202.1 202.2 202.3   203   204   301   302   303   304 
##   192   310   194   194   302     4    74   187    11   292     5   263    94 
##   305 305.1 305.2 305.3   401   402   403   404   405   406   407   408   409 
##   394   101    86     2   642   407   440    72    23    71    81   166     7 
##   410   411   412   413   414   415   416 416.2   501   502   503   504   505 
##   302   655    22     6   138     1    33    63   431    39   986   848   148 
##   506   507   601 601.1 601.2 602.1 602.2   603   604   605 605.1 605.2   606 
##   511    48   211   169    58    31   107   717    63   516   229   134    61 
## 606.1 606.2   607 607.1 607.2 607.3   608 608.1 608.2   701   702   703 703.1 
##    50     8    61    32    12   152    14     2     5   496    61    89   163 
## 703.2   704   705   706     H 
##     2    80    96   301   380
\end{verbatim}

What years, countries and parties are included in the dataset? How many
texts do you have for each of these? Prepare your data for topic
modelling by creating a document feature matrix. Describe the choices
you make here, and comment on how these might affect your final result.

\hypertarget{research-question}{%
\subsection{2. Research question}\label{research-question}}

Describe a research question you want to explore with topic modelling.
Comment on how answerable this is with the methods and data at your
disposal.

\hypertarget{topic-model-development}{%
\subsection{3. Topic model development}\label{topic-model-development}}

Create a topic model using your data. Explain to a non-specialist what
the topic model does. Comment on the choices you make here in terms of
hyperparameter selection and model choice. How might these affect your
results and the ability to answer your research question?

\hypertarget{topic-model-description}{%
\subsection{4. Topic model description}\label{topic-model-description}}

Describe the topic model. What topics does it contain? How are these
distributed across the data?

\hypertarget{answering-your-research-question}{%
\subsection{5. Answering your research
question}\label{answering-your-research-question}}

Use your topic model to answer your research question by showing plots
or statistical results. Discuss the implications of what you find, and
any limitations inherent in your approach. Discuss how the work could be
improved upon in future research.

\hypertarget{sources}{%
\section{Sources}\label{sources}}

Lehmann, Pola / Franzmann, Simon / Burst, Tobias / Regel, Sven /
Riethmüller, Felicia / Volkens, Andrea / Weßels, Bernhard / Zehnter,
Lisa (2023): The Manifesto Data Collection. Manifesto Project
(MRG/CMP/MARPOR). Version 2023a. Berlin: Wissenschaftszentrum Berlin für
Sozialforschung (WZB) / Göttingen: Institut für Demokratieforschung
(IfDem). \url{https://doi.org/10.25522/manifesto.mpds.2023a}

\hypertarget{resources}{%
\section{Resources}\label{resources}}

\begin{verbatim}
* [manifestoR vignette](https://cran.r-project.org/web/packages/manifestoR/vignettes/manifestoRworkflow.pdf)
\end{verbatim}

  \bibliography{../presentation-resources/MyLibrary.bib}

\end{document}
